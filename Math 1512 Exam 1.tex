\documentclass[11pt,reqno]{article}
%\input{17F-137-HomeworkTemplate}
\title{Homework \#2}
%\maketitle
%\due{Tuesday 9/12}

\headheight -10pt

%%\voffset 0.3 cm

\textheight 26cm% 25.5cm

\textwidth 18cm

\topmargin -2cm

\parindent 0pt

\oddsidemargin -1.2cm \columnsep 18pt

\usepackage{amsmath}
\usepackage{amsthm}
\usepackage{amssymb}
\usepackage{amsfonts, mathdots}
\usepackage{graphicx}
\usepackage{latexsym}
\usepackage{times}
\usepackage{fancyhdr}
\usepackage{url}
\usepackage{multicol}
\usepackage{cite}
\usepackage{hyperref}
\usepackage{mathrsfs}
\usepackage[usenames]{color}
\usepackage{enumitem}
%\usepackage{verbatim}

\usepackage{subcaption}
\usepackage{setspace}


\usepackage{tikz}
\usetikzlibrary{decorations.pathreplacing}
\usetikzlibrary{positioning}
\usetikzlibrary{calc}
\usetikzlibrary{arrows,shapes,backgrounds,3d}
\usepackage{fix-cm}
\usetikzlibrary{decorations.pathreplacing,shapes,snakes}

%%
\newcommand{\coursenumber}{Math 137}
\newcommand{\coursename}{Real and Functional Analysis}


\renewcommand{\Re}[1]{\operatorname{Re} #1 }
\renewcommand{\Im}[1]{\operatorname{Im} #1}
\newcommand{\diam}{\operatorname{diam}}

%% New Commands
\newcommand{\D}{\mathbb{D}}
\newcommand{\Z}{\mathbb{Z}}
\newcommand{\R}{\mathbb{R}}
\newcommand{\Q}{\mathbb{Q}}
\newcommand{\C}{\mathbb{C}}
\newcommand{\F}{\mathbb{F}}
\newcommand{\G}{\mathcal{G}}
\newcommand{\M}{\mathcal{M}}
\newcommand{\K}{\mathbb{K}}
\newcommand{\U}{\mathcal{U}}
\newcommand{\UU}{\mathscr{U}}
\newcommand{\s}{\mathscr{S}}
\newcommand{\A}{\mathcal{A}}
\renewcommand{\L}{\mathfrak{L}}
\newcommand{\B}{\mathcal{B}}
\renewcommand{\P}{\mathcal{P}}
\newcommand{\N}{\mathbb{N}}

\DeclareMathOperator{\Aut}{Aut}



\renewcommand{\ker}{\operatorname{kernel}}
\newcommand{\ran}{\operatorname{range}}

\newcommand{\diag}{\operatorname{diag}}
\newcommand{\norm}[1]{\| #1 \|}
\newcommand{\inner}[1]{\langle #1 \rangle}
\newcommand{\E}{\mathcal{E}}
\newcommand{\V}{\mathcal{V}}
\newcommand{\W}{\mathcal{W}}
\newcommand{\WW}{\mathscr{W}}
\newcommand{\T}{\mathbb{T}}
\newcommand{\vecspan}{\operatorname{span}}
\newcommand{\interior}{\operatorname{int}}
\newcommand{\tr}{\operatorname{tr}}
\newcommand{\rank}{\operatorname{rank}}
\newcommand{\nullity}{\operatorname{nullity}}

\newcommand{\colspace}{\operatorname{colspace}}
\newcommand{\rowspace}{\operatorname{rowspace}}
\newcommand{\nullspace}{\operatorname{nullspace}}

\linespread{1}
\setlength{\parskip}{0.5ex plus 0.5ex minus 0.2ex}

%%  Matrices
\newcommand{\minimatrix}[4]{\begin{bmatrix} #1 & #2 \\ #3 & #4 \end{bmatrix}  }
\newcommand{\megamatrix}[9]{\begin{bmatrix} #1 & #2 & #3 \\ #4 & #5 & #6 \\ #7 & #8 & #9\end{bmatrix}  }

\renewcommand{\vec}[1]{{\bf #1}}
\newcommand{\ovec}{\operatorname{vec}}
\renewcommand{\labelenumi}{(\roman{enumi})}
\renewcommand{\hat}{\widehat}

\newcommand{\tworowvector}[2]{[#1\,\,#2]}
\newcommand{\threerowvector}[3]{[#1\,\,#2\,\,#3]}
\newcommand{\fourrowvector}[4]{[#1\,\,#2\,\,#3\,\,#4]}
\newcommand{\fiverowvector}[5]{[#1\,\,#2\,\,#3\,\,#4\,\,#5]}
\newcommand{\sixrowvector}[6]{[#1\,\,#2\,\,#3\,\,#4\,\,#5\,\,#6]}

\newcommand{\twovector}[2]{\begin{bmatrix} #1\\#2 \end{bmatrix} }
\newcommand{\threevector}[3]{\begin{bmatrix} #1\\#2\\#3 \end{bmatrix} }
\newcommand{\fourvector}[4]{\begin{bmatrix} #1\\#2\\#3\\#4 \end{bmatrix} }
\newcommand{\fivevector}[5]{\begin{bmatrix} #1\\#2\\#3\\#4\\#5 \end{bmatrix} }

\newcommand{\comment}[1]{\marginpar{\scriptsize\color{gray}$\bullet$\,#1}}
\newcommand{\highlight}[1]{\color{blue}#1\color{black}}

\renewcommand{\labelenumi}{\theenumi}
\renewcommand{\theenumi}{(\alph{enumi})}%
\renewcommand{\labelenumii}{\theenumii}
\renewcommand{\theenumii}{(\roman{enumii})}%
\newcommand{\due}[1]{\vspace{-0.2in}\begin{center}\textsc{due at the beginning of class \underline{#1}} \end{center}\medskip }


%%%
%%% Theorem Styles
%%%
\newtheorem{Proposition}{Proposition}
\newtheorem{Corollary}{Corollary}
\newtheorem{Theorem}{Theorem}
\newtheorem*{Thm}{Theorem}
\newtheorem{Postulate}{Postulate}
\newtheorem{Lemma}{Lemma}
\theoremstyle{definition}
\newtheorem*{Definition}{Definition}
\newtheorem*{Example}{Example}
\newtheorem*{Remark}{Remark}
\newtheorem{problem}{Exercise}
\newtheorem*{Question}{Question}

\let\oldenumerate=\enumerate
\def\enumerate{
	\oldenumerate
	\setlength{\itemsep}{1pt}
}
\let\olditemize=\itemize
\def\itemize{
	\olditemize
	\setlength{\itemsep}{5pt}
}

\allowdisplaybreaks
\pagenumbering{gobble}
\begin{document}
\centerline{\textbf{\Large{Math 1512 Exam 1}}}

\vspace{0.12in}

\textbf{NAME:}\hrulefill

\vspace{0.2in}


\textbf{INSTRUCTIONS:}

SHOW ALL OF YOUR WORK. Unsupported and illegible answers will not receive credit. Use\textbf{ proper mathematical notation} to receive full credit.
Absolutely NO electronic devices or notes are allowed during this test. May the Force be with you...
	
	\begin{enumerate}
		\item[1.] (10 pts) Compute the following limits. If a limit does not exist explain why. 
		\begin{enumerate}
			\item[a.] $\lim\limits_{x \to 1} \frac{\sqrt{x + 8} - 3}{x - 1}$
			\vspace{8cm}
			\item[b.] $\lim\limits_{x \to 4} \frac{\frac{1}{x} - \frac{1}{4}}{x - 4}$
		\end{enumerate}
	\newpage
		\item[2.] (20 pts) For what value of the constant $c$ is the function $f$ continuous on $(-\infty,\infty)$?. Once you have found $c$, sketch a graph of $f$. Be sure to label key points.
		\[f(x)=\begin{cases} cx^2+4, & x< -2\\ -x-3c, & x\geq -2 \end{cases}\]
		
		\vspace{2in}
		
		
		\begin{tikzpicture}[>=latex]
			%x axis
			\draw[->] (-7,0) -- (7,0) node[below] {$x$};
			
			
			%y axis
			\draw[->] (0,-5) -- (0,5) node[left] {$y$};
			
			\node[below left] at (0,0) {\footnotesize $0$};
		\end{tikzpicture}
		
		
		
		
		%\vspace{1in}
		
		%\item For what values of $a$ does $f(x)$ fail to be differe
		
		
		Write a formula for $f'(x)$ as a piece-wise function and determine if $f$ is differentiable on $(-\infty,\infty)$. Justify your answer carefully.
		\newpage 
		\item[3.] (20 pts) The gravitational force exerted by the planet Earth on a unit mass at a distance $r$ from the center of the planet is
		
		\[F(r)=\begin{cases} \frac{GMr}{R^3}, &0\leq r<R \\\frac{GM}{r^2}, & r\geq R  \end{cases}\]  where $M$ is the mass of the Earth, R is its radius, and $G$ is the gravitational constant. 
		
		\begin{enumerate}
			
			\item Is $F(r)$ continuous? Explain. 
			
			\vspace{2in}
			
			\item Sketch a graph of $F(r)$. Label the key points.
			
			\begin{tikzpicture}[>=latex]
				%x axis
				\draw[->] (-7,0) -- (7,0) node[below] {$r$};
				
				
				%y axis
				\draw[->] (0,-5) -- (0,5) node[left] {$F(r)$};
				
				\node[below left] at (0,0) {\footnotesize $0$};
			\end{tikzpicture}
			
			\item Compute $F'(2R)$. What does the sign of the derivative indicate?
		\end{enumerate}
		\newpage
		\item[4.]  (20 pts)Let $f(x) = - \frac{10}{\sqrt{1 - x}}$. Use the limit definition of the derivative to find $f'(x)$.
		\newpage
		\item[5.] (10 pts) Evaluate and simplify $y'$ 
		\begin{enumerate}
			%\item[a.] $y = \frac{4u^2 + u}{8u + 1}$ 
			%\vspace{6cm}
			\item[a.] $y = \left(1 + \frac{1}{x^2\sqrt{x}}\right)(x^2 + 1)$
			\vspace{12cm}
			\item[b.] $ y  = 4t^2 - \frac{2t}{5t + 1}$ 
			%\vspace{6cm}
			%\item[d.] $ y = 5 x^3 - 3 x + \sqrt{x} - \frac{4}{x^2 \sqrt{x}}$ 
		\end{enumerate} 
		\newpage
		\item[6.] (20 pts) Consider $f(x) = \frac{x^2 - 9}{x(x - 3)}$. 
		\begin{enumerate}
			\item[a.] Find the vertical asymptotes of $f$. 
			\vspace{3cm}
			\item[b.] Find any horizontal asymptotes of $f$ 
			\vspace{5cm}
			\item[c.] Find the tangent line to $f$ at the point $(2, \frac{5}{2})$ 
		\end{enumerate}
	\end{enumerate}
	
	
\end{document}
