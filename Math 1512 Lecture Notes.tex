\documentclass[12pt,reqno]{article}
%\input{17F-137-HomeworkTemplate}
\title{Homework \#2}
%\maketitle
%\due{Tuesday 9/12}

\usepackage[margin=0.5in]{geometry}
\usepackage{amsmath}
\usepackage{amsthm}
\usepackage{amssymb}
\usepackage{amsfonts}
%\usepackage{graphicx}
\usepackage{latexsym}
\usepackage{times}
\usepackage{fancyhdr}
\usepackage{url}
\usepackage{multicol}
\usepackage{cite}
\usepackage{hyperref}
\usepackage{mathrsfs}
\usepackage[usenames]{color}
\usepackage{enumitem}
%\usepackage{verbatim}
\usepackage{tikz-cd}
\usepackage{pgfplots}

\usepackage{subcaption}
\usepackage{setspace}


\usepackage{tikz}
\usetikzlibrary{decorations.pathreplacing}
\usetikzlibrary{positioning}
\usetikzlibrary{calc}
\usetikzlibrary{arrows,shapes,backgrounds,3d}
\usepackage{fix-cm}
\usepackage{tikz-cd}
\usetikzlibrary{decorations.pathreplacing,shapes,snakes}

%%
\newcommand{\coursenumber}{Math 137}
\newcommand{\coursename}{Real and Functional Analysis}


\renewcommand{\Re}[1]{\operatorname{Re} #1 }
\renewcommand{\Im}[1]{\operatorname{Im} #1}
\newcommand{\diam}{\operatorname{diam}}

%% New Commands
\newcommand{\D}{\mathbb{D}}
\newcommand{\Z}{\mathbb{Z}}
\newcommand{\R}{\mathbb{R}}
\newcommand{\Q}{\mathbb{Q}}
\newcommand{\C}{\mathbb{C}}
\newcommand{\F}{\mathbb{F}}
\newcommand{\G}{\mathcal{G}}
\newcommand{\M}{\mathcal{M}}
\newcommand{\K}{\mathbb{K}}
\newcommand{\U}{\mathcal{U}}
\newcommand{\UU}{\mathscr{U}}
\newcommand{\s}{\mathscr{S}}
\newcommand{\A}{\mathcal{A}}
\renewcommand{\L}{\mathfrak{L}}
\newcommand{\B}{\mathcal{B}}
\renewcommand{\P}{\mathcal{P}}
\newcommand{\N}{\mathbb{N}}



\renewcommand{\ker}{\operatorname{kernel}}
\newcommand{\ran}{\operatorname{range}}

\newcommand{\diag}{\operatorname{diag}}
\newcommand{\norm}[1]{\| #1 \|}
\newcommand{\inner}[1]{\langle #1 \rangle}
\newcommand{\E}{\mathcal{E}}
\newcommand{\V}{\mathcal{V}}
\newcommand{\W}{\mathcal{W}}
\newcommand{\WW}{\mathscr{W}}
\newcommand{\T}{\mathbb{T}}
\newcommand{\vecspan}{\operatorname{span}}
\newcommand{\interior}{\operatorname{int}}
\newcommand{\tr}{\operatorname{tr}}
\newcommand{\rank}{\operatorname{rank}}
\newcommand{\nullity}{\operatorname{nullity}}

\newcommand{\colspace}{\operatorname{colspace}}
\newcommand{\rowspace}{\operatorname{rowspace}}
\newcommand{\nullspace}{\operatorname{nullspace}}

\linespread{1.1}
\setlength{\parskip}{0.5ex plus 0.5ex minus 0.2ex}

%%  Matrices
\newcommand{\minimatrix}[4]{\begin{bmatrix} #1 & #2 \\ #3 & #4 \end{bmatrix}  }
\newcommand{\megamatrix}[9]{\begin{bmatrix} #1 & #2 & #3 \\ #4 & #5 & #6 \\ #7 & #8 & #9\end{bmatrix}  }

\renewcommand{\vec}[1]{{\bf #1}}
\newcommand{\ovec}{\operatorname{vec}}
\renewcommand{\labelenumi}{(\roman{enumi})}
\renewcommand{\hat}{\widehat}

\newcommand{\tworowvector}[2]{[#1\,\,#2]}
\newcommand{\threerowvector}[3]{[#1\,\,#2\,\,#3]}
\newcommand{\fourrowvector}[4]{[#1\,\,#2\,\,#3\,\,#4]}
\newcommand{\fiverowvector}[5]{[#1\,\,#2\,\,#3\,\,#4\,\,#5]}
\newcommand{\sixrowvector}[6]{[#1\,\,#2\,\,#3\,\,#4\,\,#5\,\,#6]}

\newcommand{\twovector}[2]{\begin{bmatrix} #1\\#2 \end{bmatrix} }
\newcommand{\threevector}[3]{\begin{bmatrix} #1\\#2\\#3 \end{bmatrix} }
\newcommand{\fourvector}[4]{\begin{bmatrix} #1\\#2\\#3\\#4 \end{bmatrix} }
\newcommand{\fivevector}[5]{\begin{bmatrix} #1\\#2\\#3\\#4\\#5 \end{bmatrix} }

\newcommand{\comment}[1]{\marginpar{\scriptsize\color{gray}$\bullet$\,#1}}
\newcommand{\highlight}[1]{\color{blue}#1\color{black}}

\renewcommand{\labelenumi}{\theenumi}
\renewcommand{\theenumi}{(\alph{enumi})}%
\renewcommand{\labelenumii}{\theenumii}
\renewcommand{\theenumii}{(\roman{enumii})}%
\newcommand{\due}[1]{\vspace{-0.2in}\begin{center}\textsc{due at the beginning of class \underline{#1}} \end{center}\medskip }


%%%
%%% Theorem Styles
%%%
\newtheorem{Proposition}{Proposition}
\newtheorem{Corollary}{Corollary}
\newtheorem{Theorem}{Theorem}
\newtheorem*{Thm}{Theorem}
\newtheorem{Postulate}{Postulate}
\newtheorem{Lemma}{Lemma}
\theoremstyle{definition}
\newtheorem*{Definition}{Definition}
\newtheorem*{Example}{Example}
\newtheorem*{Remark}{Remark}
\newtheorem{problem}{Exercise}
\newtheorem*{Question}{Question}

\let\oldenumerate=\enumerate
\def\enumerate{
	\oldenumerate
	\setlength{\itemsep}{5pt}
}
\let\olditemize=\itemize
\def\itemize{
	\olditemize
	\setlength{\itemsep}{5pt}
}

\allowdisplaybreaks
\begin{document}
\title{Math 1512 Lecture Notes}


\section{6/07}

\begin{enumerate}
	\item Introductions + Group introductions (ask year, possible major, where from, interesting fact, what you want to gain from this course) 
	\item Go over syllabus and first quiz online
	\item Begin lecture (40 min?)
	\item Worksheet (20 min ?)
	\item Go over worksheet (20 min?)
\end{enumerate}

\subsection{Intro to Limits (2.1, 2.2)} 

(ask if people know what limits are)

First, let's talk about why we care about limits - we care about limits because of a real world application: what is called instantaneous velocity and a more general mathematical application: finding the slope of a line tangent to a curve. (These two are actually related which will be seen in the examples). 

Looking at the first application - average vs instantaneous velocity. What is the difference? For example if on a road trip from let's say Albuquerque to Santa Fe you travel the distance (roughly 60 miles) in 1 hour. Average velocity is then 60 mi/ 1 hr = 60 mph. However instantaneous velocity will be different. You won't be going 60 mph in the city for example (the speed on the speedometer) 

\begin{Example}
	A rock is launched vertically upward from the ground with a speed of 49 m/s. From physics we know that the position function for the rock is given by $s(t) = -9.8t^2 + 49 t$. We can then find the average velocity of the rock between the times
	\begin{enumerate}
		\item[a.] $t = 1$ and $t = 3$ - 
			$s(1) = 39.2$m  and $s(3) = 58.8$m so the average velocity is given by $s(3)- s(1)/ 2 = 19.6/2 = 9.8 $ m/s
		\item[b.] $t = 1$ and $t = 2$ - 
			$s(2) = 58.8 m$ so then average velocity is given by $s(2) - s(1)/1 = 19.6$m/s
	\end{enumerate}



	(Draw a graph of this example) 
\end{Example}

		\begin{Example}
		A rock is launched vertically upward from the ground with a speed of 96 ft/s. Formula for the position of the rock is given by $s(t) = -16t^2 + 96t$. We can do the same thing as in the previous example.
		
		\begin{enumerate}
			\item[a.] Between $t = 3$ and $t = 1$ we have $(144 - 80)/2 = 32$ 
			\item[b.] Between $t = 2$ and $t = 1$ we have $(128 - 80)/1 = 48$. 
		\end{enumerate} 
	\end{Example}
	(Draw graph) 
	Now if we look at the graphs we see that finding average velocity is finding the slope of the secant line (a secant line is a line that intersects a graph at two points). But we want to eventually get to finding the instantaneous velocity. So how do we get there? We can estimate this by computing the average over smaller and smaller intervals. As we do see the value that the average approaches is called the limit. 
	
	Looking back at the example with the rock we have the following table - 
	
	
	\begin{tabular}{| c | c |}
		\hline
		Time Interval & Average velocity(ft/s) \\
		\hline
		[1, 2] & 48 \\ \hline
		[1, 1.5] & 56 \\ \hline
		[1, 1.1] & 62.4 \\ \hline
		[1, 1.01] & 63.84 \\\hline
		[1, 1.001] & 63.984 \\\hline
		[1, 1.0001] & 63.9984 \\\hline
	\end{tabular}

	From this table what can we guess is our limit (instantaneous velocity at 1 second)? We can write this as $\lim_{t \to 1} (s(t) - s(1))/(t - 1) = 64$ ft/s. 
	
	Going back to the graph let us see how this relates to the slop of the tangent line by drawing the secant lines and seeing that they get close the the tangent line. 
	
	[2.2]
\begin{Definition}
	Suppose the function $f$ is defined for all $x$ near $a$ except possibly at $a$. If $f(x)$ is arbitrarily close to $L$ (as close to $L$ as we like) for all $x$ sufficiently close (but not equal) t o $a$, we say $\lim_{x \to a} f(x) = L$ (the limit of $f$ as $x$ approaches $a$ equals $L$). 
\end{Definition}

Important to note that $\lim_{x \to a} f(x)$ depends on the values of $f$ near $a$ if it exists, NOT on $f(a)$. In some cases $\lim f(x)$ = $f(a)$ but it may possibly be different or $f(a)$ may not exist. 



Most common way to see limits is through a graph. (Graph a function with all three cases above)

Another way to find limits is through a table. For example the function $f(x) = (\sqrt{x} - 1)/(x - 1)$. We can look at the table of values going close to one and then estimate what $\lim_{x \to 1} f(x)$ is. 

\begin{tabular}{| c | c | c | c | c | c | c | c | c |} \hline
	x & 0.9 & 0.99 & 0.999 & 0.9999 & 1.0001 & 1.001 & 1.01 & 1.1 \\
	 \hline
	f(x) & 0.5131670 & 0.5012563 & 0.5001251 & 0.5000125 & 0.4999875 & 0.4998751 & 0.4987562 & 0.4880885 \\
	\hline
\end{tabular}

So what do you think is a reasonable estimate for what the limit is? 


The limits we've been working with so far are called two-sided limits. You can see this by looking at the table above. We see that as $f$ approaches 1 from smaller values and larger values you get the same value. We can also talk about one-sided limits - limits where you only approach a value from one direction (either "above" or "below"). These are called "right" and "left" -sided limits respectively. 

\begin{Definition}
	
	
	\begin{enumerate}
		\item[1.] Right-sided limit - Suppose $f$ is defined for all $x$ near $a$ with $x > a$. If $f(x)$ is arbitrarily close to $L$ for all $x$ sufficiently close to $a$ with $x > a$, we write $\lim_{x \to a^{+}} f(x) = L$ and say the limit of $f$ as $x$ approaches $a$ from the right (or from above) equals $L$. 
		\item[2.] Similarly for a left-sided limit, suppose $f$ is defined for all $x$ near $a$ with $x < a$. If $f(x)$ is arbitrarily close to $L$ for all $x$ sufficiently close to $a$ with $x < a$ we write $\lim_{x \to a^{-}} f(x) = L$ and say the limit of $f$ as $x$ approaches $a$ from the left (or from below) equals $L$. 
	\end{enumerate}


\end{Definition}

\begin{Example}
	An example function for this would be the function $f(x) = (x^3 - 8)/(4(x - 2))$. We can graph and make a table for this to evaluate the three limits we want to look at. 
	
	(graph like a parabola but DNE at $x = 2$) 
	
	Table: 
	
	\begin{tabular}{| c | c | c | c | c | c | c | c | c |}
		\hline
		x & 1.9 & 1.99 & 1.999 & 1.9999 & 2.0001 & 2.001 & 2.01 & 2.1 \\ \hline
		f(x) & 2.8525 & 2.985025 & 2.99850025 & 2.99985000 & 3.0015000 & 3.00150025 & 3.015025 & 3.1525 \\ \hline
	\end{tabular}
	
	Based on this we see that the the left and right side limits exist and equal the two side limit. 
\end{Example}

This does lead to a theorem about the relationship between a two side limit and the one sided limits - 
\begin{Theorem}
	Assume $f$ is defined for all $x$ near $a$ except possibly at $a$. Then $\lim_{x \to a} f(x) = L$ if and only if $\lim_{x \to a^+ } f = L$ and $\lim_{x \to a^{-}} f = L$. 
\end{Theorem}

However it is not always the case that the two sided limit exists even if the one-sided limits exist. Important to talk about how you know whether a limit exists or not. For now it is easiest to look at either a graph or a table. 

\section{6/09}

\begin{enumerate}
	\item[a.] Quick review of what was covered last class + notes on where to find some more info. Questions? Talk about MML and access problems.
	\item[b.] Lecture on 2.3 + 2.4. (50 min) 
	\item[c.] Worksheet + Examples (on Zoom so maybe walk through them after giving some time to work on them) (40 min) 
	\item[d.] Close out with talking about the quiz and the homework. 
\end{enumerate}

\subsection{Techniques for Computing Limits}

Last time we looked at graphical and numerical methods for calculating limits (using a calculator and tables = numerical). However, these can lead to incorrect results ocassionally so we want to look at algebraic/analytic methods for evaluating limits. 

\subsubsection{Linear Functions}

Linear functions are the easiest to look at so that is what we will start with. Remember that a linear function is usually defined as a function $f(x) = mx + b$ (ask what $m$ and $b$ are?). We can look at a graph of such a function and see simply that $\lim_{x \to a} f(x)  = f(a)$ since $f(x)$ approaches $f(a)$ as $x$ approaches $a$. This leads to the following theorem - 
\begin{Theorem}
	Let $a, b$ and $m$ be real numbers. For linear functions $f(x) = mx + b$ $\lim_{x \to a}f(x) =f(a) = ma + b$. 
\end{Theorem}

We can then try this with a few examples. Consider $\lim_{x \to 3} \frac{1}{2} x - 7. $ (A note about notation - we can either first define a function by saying "Let/Define $f(x) = \frac{1}{2} x - 7$" and then write $\lim_{x\to 3} f(x)$ or we can just not use function notation and write the limit directly like before). 

What we are doing here (direct substitution) is the easiest way to evaluate limits. However it is not always possible. Before we get into that, let us first also talk about some "Laws" that will help us simplify and evaluate limits -
\subsubsection{Limit Laws/Properties/Rules}

\begin{Theorem}
	Assume that $\lim_{x \to a} f(x)$ and $\lim_{x \to a} g(x)$ exist. The following properties hold, where $c$ is a real number and $n > 0$ is an integer. 
	\begin{enumerate}
		\item \textbf{Sum:} $\lim_{x \to a} (f(x) + g(x)) = \lim_{x \to a} f(x) + \lim_{x\to a} g(x)$ 
		\item \textbf{Difference:} $\lim_{x \to a} (f(x) - g(x)) = \lim_{x \to a} f(x) - \lim_{x \to a} g(x)$. 
		\item \textbf{Constant Multiple:} $\lim_{x \to a} (c f(x)) = c \lim_{x \to a} f(x)$ 
		\item \textbf{Product:} $\lim_{x\to a} (f(x) g(x)) = (\lim_{x \to a} f(x) ) (\lim_{x \to a} g(x))$ 
		\item \textbf{Quotient:} $\lim_{x \to a} \left(\frac{f(x)}{g(x)} \right) = \frac{\lim_{x \to a} f(x)}{\lim_{x \to a} g(x)}$ assuming $\lim_{x \to a} g(x) \neq 0$. 
		\item \textbf{Power:} $\lim_{x \to a} (f(x))^n = (\lim_{x \to a} f(x))^n $
		\item \textbf{Root:} $\lim_{x\to a}(f(x))^{1/n} = (\lim_{x \to a} f(x))^{1/n}$ assuming $f(x) > 0$ for $x$ near $a$ if $n$ is even. 
	\end{enumerate}
\end{Theorem}

The proof of this theorem (and any theorem we talk about is in section 2.7). Maybe do a couple examples here.

\subsubsection{Limits of Polynomials and Rational Functions} 
We can know use the limit properties to find limits of polynomials and rational functions. For example if we wanted to find $\lim_{x \to 2} (7 x^3 + 3 x^2 + 4x + 2)$, we then use the sum, constant multiple and power properties from above to get
\begin{align*}
	\lim_{x \to 2} (7 x^3 + 3 x^2 + 4x + 2) &= \lim_{x\to 2} 7x^3 + \lim_{x\to2} 3x^2 + \lim_{x\to2}(4x + 2) \\
										    &= 7 \lim_{x\to 2} x^3 + 3 \lim_{x\to 2} x^2 + \lim_{x\to 2}(4x + 2) \\
										    &= 7 (\lim_{x \to 2} x)^3 + 3 (\lim_{x \to 2} x)^2 + \lim_{x \to 2}(4x + 2). \\
										    &= 7 (2)^3 + 3 (2)^2 + (4(2) + 2) = 56 + 12 + 10 = 78.
\end{align*}
At the end we can use the theorem about limits of linear functions to directly substitute in and evaluate. However we do not need to do all of this work every time we want to evaluate the limit of a polynomial (or rational function). Instead we have this extremely useful theorem 
\begin{Theorem}
	Assume $p$ and $q$ are polynomials and $a$ is a constant. Then 
	\begin{enumerate}
		\item $\lim_{x\to a} p(x) = p(a)$ 
		\item $\lim_{x \to a} \frac{p(x)}{q(x)} = \frac{p(a)}{q(a)}$ assuming $q(a) \neq 0$. 
	\end{enumerate}
\end{Theorem}
	
\subsubsection{One-Sided Limits}
Up until now we have been talking about two sided limits. But what about one-sided limits? Do our limit properties and theorem apply for one-sided limits as well? Yes they do for the most part with some slight modifications. The one property that is modified for one-sided limits is the "root" property. So for one-sided limits we have 
\begin{Theorem}
	\begin{enumerate}
		\item $\lim_{x \to a^+} (f(x))^{1/n} = (\lim_{x \to a^+} f(x))^{1/n}$ assuming $f(x) \geq 0$ for $x$ near $a$ with $x > a$ if $n$ is even.
		\item Similarly $\lim_{x \to a^-} (f(x))^{1/n} = (\lim_{x \to a^-} f(x))^{1/n}$ assuming $f(x) \geq 0$ for $x$ near $a$ with $x < a$ if $n$ is even. 
	\end{enumerate}
\end{Theorem}

An example for one-sided limits 
\begin{Example}
	Let $f(x) = \begin{cases}
		-2x + 4 & \textnormal{if } x \leq 1, \\
		\sqrt{x - 1} & \textnormal{if } x > 1.
	\end{cases}$  

We want to then find the values of $\lim_{x \to 1^-} f(x)$, $\lim_{x \to 1^+} f(x)$ and $\lim_{x \to 1} f(x)$ if they exist. 
\end{Example}


\subsubsection{Other techniques}

For most of what we have done here we have had the ability ot use direct substitution. What about cases where this is not possible? For example the three we looked at on Monday on the worksheet are ones where we cannot use direct substitution. This is where we have to use algebra to help us solve. 

First there are two possible cases where we cannot use direct substitution. 1) is where $f(a)$ exists, but $\lim_{x \to a} f(x) \neq f(a)$ (you have a jump in the function) 2) the limit exists but $f(a)$ does not. 

Let us look at an example for the first worksheet. 
\begin{Example}
	Let us evaluate the limit $\lim_{x \to 2} \frac{x^2 - 4} {x - 2}$. Note that if we try direct substitution we get $0/0$ which does not tell us anything. The method we will use to solve this is to factor and then cancel. (factor and then cancel, then evaluate).  
	
	\begin{align*}
		\lim_{x \to 2} \frac{x^2 -4}{x - 2} = \lim_{x\to 2} \frac{(x -2)(x +2)}{x -2} = \lim_{x \to 2} (x + 2) = 4.
	\end{align*}
\end{Example}

\begin{Example}
	The second method we can use is multiplication by conjugates. You will normally use this when your function has square roots in it. Taking a look at another example from the first work sheet we have  $\lim_{t \to 9} \frac{t - 9}{\sqrt{t} - 3}$. The conjugate of the denominator is $\sqrt{t} +3$. So multiplying gives us .... 
	
	\begin{align*}
		\lim_{t \to 9}  \frac{t - 9}{\sqrt{t} - 3} \cdot \frac{\sqrt{t} + 3}{\sqrt{t} +3} = \lim_{t \to 9} \frac{(t - 9) (\sqrt{t} + 3)}{t - 9} = \lim_{t \to 9} \sqrt{t} + 3 = 6
	\end{align*}
\end{Example}

%	$(\sqrt{t} - (3 + x))(\sqrt{t} + (3 + x))$
	
	


\section{6/10}
\begin{enumerate}
	\item[a.] Begin with review of previous day (Limit properties, direct substitution and then factor-cancel/conjugates) 
	\item[b.] Finish 2.3 with Squeeze Theorem 
	\item[c.] Start 2.4
	\item[d.] Worksheet for practice.
\end{enumerate}

\subsection{Evaluating Limits (cont.)} 
\subsubsection{Squeeze Theorem}

Another important theorem that can help evaluate limits is the Squeeze Theorem. As the name suggests, what you do is that you squeeze the limit that you want to find in between limits of things that you already now. For formally we have 
\begin{Theorem}
	Assume functions $f, g$ and $h$ satisfy $f(x) \leq g(x) \leq h(x)$ for all values of $x$ near $a$, except possibly at $a$. If $\lim_{x \to a} f(x) = \lim_{x\to a} h(x) = L$ then $\lim_{x \to a} g(x) = L$. 
\end{Theorem}

\begin{Example}
	As an example of how to use this theorem let us look at the limit $\lim_{ x\to 0} x^2 \sin(1/x)$. Why can we not just use direct substitution? What about our algebraic methods from before? 
	
	By properties of trig functions we know that $-1 \leq \sin(1/x) \leq 1$ for all $x$ near 0 (Except 0). As for $x \neq 0, x^2 > 0$ we can multiply both sides by $x^2$ and get $-x^2 \leq x^2\sin(1/x) \leq x^2$. Now we have a sequence of inequalities of functions like in our theorem and furthermore we know that $\lim_{x \to 0} -x^2  = \lim{x \to 0} x^2 = 0$ so by the Squeeze theorem we have that $\lim_{x \to 0} x^2 \sin(1/x) = 0.$ This theorem is extremely useful many times when dealing with trig functions (we will use it later on as well). 
\end{Example}

\begin{Example}
	While the previous trig limit was done using the Squeeze theorem not all trig limits can be calculated by the squeeze theorem. Instead you need to use trig identities. (who remembers trig identities?) 
	
	My personal three trig identities to remember are the following three : 
	\begin{align}
		&\sin^2 x + \cos^2 x = 1 (\textnormal{ The Pythagorean Identity}) \\
		&\sin 2x = 2 \sin x \cos x \\
		&\cos 2x = \cos^2 x - \sin^2 x (\textnormal{The Sine and Cosine Double Angle Formulas}).
	\end{align} 

	As an example let us look at evaluating the limit 
	\begin{align*}
		\lim_{x \to 0} \frac{1 - \cos 2x } {\sin x}.
	\end{align*}
	We cannot use direct substitution as we get 0/0. So we need to use our trig identities to solve. So what identity do you think we can use? 
\end{Example}

Using these two we will eventually (section 2.6) show that we can use direct substitution with trig functions. 

\subsection{Infinite Limits}
	We have until know talked about evaluating limits that are of finite value. There are two other limit scenarios we have not covered, one of which we have briefly talked about - infinite limits. An infinite limit is when the function values increase or decrease without bound near a point. An example of which I gave on the first day of class $y = \frac{1}{x}$. (draw a graph and look at the limit near $x = 0$). Similarly we briefly talked about an example from the worksheet yesterday which was similar. This is the scenario that is mentioned in 2.4 in the book. The other limit scenario is limits at infinity. This would be where we consider the limit of a function at really large $x$ values (i.e. $\lim_{x \to \infty}$). This is covered in 2.5. 
	
	Let us give a slightly more formal definition of infinite limits - 
	\begin{Definition}
		Suppose $f$ is defined for all $x$ near $a$. If $f(x)$ grows arbitrarily large for all $x$ sufficiently close (but not equal) to $a$ then we say $\lim_{x \to a} f(x) = \infty$. (Similarly if $f(x)$ is negative and grows large in magnitude then $-\infty$). 
	\end{Definition}

	An important remark - while we write and say that a limit is either positive or negative infinity it is important to note that $\lim_{x \to a} f(x) = \infty$ means that the limit DOES NOT EXIST.  (use $\frac{1}{x^2})$ as an example). 
	
	As always we can talk about one-sided infinite limits. In fact we have already discussed this before with the example of 1/x. 
	
	Looking at these kind of graphs of rational functions, note that whenever we have infinite limits there is always this line in the center - This line is important and called the vertical asymptote. Here is the formal definition
	
	\begin{Definition}
		If $\lim_{x \to a} f(x) = \pm \infty$, $\lim_{x\to a^+} f(x) = \pm \infty$, or $\lim_{x \to a^-} f(x)  = \pm \infty$ the line $x  = a$ is called a vertical asymptote of $f$.
	\end{Definition}

	\subsubsection{How to find V. Asymptotes}
	
	So how to we find vertical asymptotes? If you are given a graph you would look for where the function goes to infinity/has infinite limits (ex. draw picture). If you are given a function like $g(x) = \frac{x - 2}{(x - 1)^2 (x - 3)}$ , and we cannot use a computer/calculator to graph it how do we find the v. asymptotes? You look at where the denominator is 0. 
	
	\subsubsection{What about Infinite Limits}
	So we might be able to find where the vertical asymptotes are but how do we find the infinite limits near a vertical asymptote without graphing? Unfortunately to do so requires some simple arithmetic and looking at signs (positive vs negative). As an example of how we would do this let us go back to the the function $g(x)$ from before. What are the vertical asymptotes? Let us look at the the first one at $x = 1$. Now we want to know $\lim_{x \to 1^-} g(x)$ and $\lim_{x \to 1^+} g(x)$. First the first one-sided limit we need a number really close to 1 but less than 1. So take $0.99$ as an example. If we look at $g(0.99)$ we get 
	\begin{align*}
		g(0.99) = \frac{0.99 - 2}{(0.99 - 1)^2 (0.99 - 3)} = \frac{-1.01}{(0.1)^2 (-2.01)} = \frac{101}{100} \cdot \frac{100}{1} \cdot {100}{201} = \frac{1010000}{201}
	\end{align*}
	Now the important thing for me, is that this is a positive number. Now if we end up calculating what $g(0.999)$ is we see that it is $\frac{1001000000}{2001}$ so we see that it is getting bigger. However, we do not need to calculate the second value because of a fact we know - as $x = 1$ is a vertical asymptote what do we know that $\lim_{x\to 1^-} g(x)$ has to be? (it has to either be positive or negative infinity). 
	
	Similarly if we look at the limit from the right side we get that $g(1.01)$ is also positive. So that one sided limit also goes to positive infinity. Let us look at then the other vertical asymptote. We will get that the limit from the left is negative ,while from the right it is positive. 
	
	In general the method we want to look at is first make sure that ONLY the denominator is going to 0 for the value we are looking at. Then you want to look at the sign of the function near the value we are looking. Then as the denominator goes to 0, based on the sign we can decide whether it goes to positive or negative infinity. 
	\subsection{Limits at Infinity}
	
	As we have talked about infinite limits, let us then talk about hte other we way we can mix limits and infinity. Now limits at infinity means we are looking as $x$ gets really large, not the function itself. These limits are important when looking at the end behavior of a function (useful when trying to see if a system every reaches an equilibrium state). 
	
	
	Let us first look at an example. Consider $f(x) = \frac{x}{\sqrt{x^2 + 1}}$. The graph of this function looks like - . Notice that as $x \to \infty$ we have that $f(x) \to 1$ and similarly $x\to -\infty$ $f(x) \to -1$. These lines $y = 1$ and $y = -1$ are called horizontal asymptotes. 

	Here is a formal definition of a horizontal asymptote - 
	\begin{Definition}
		If $f(x)$ becomes arbitarirly close to a finite number $L$ for all sufficiently large and positive $x$ then we say $\lim_{x \to \infty} f(x) = L$. Similarly if $f(x)$ becomes arbitrarily close to a finite number $M$ for all sufficiently large in magnitude and negative $x$ then we say $\lim_{x \to -\infty} f(x) = M$. 
	\end{Definition}

	A very important limit at infinity to remember is this $\lim_{x \to \pm \infty } \frac{\pm 1}{x^p} = 0$ where $p$ is any real number. 

	Note: It is very important to note that unlike vertical asymptotes, where $f(x)$ CANNOT cross a vertical asymptote, $f(x)$ can cross a horizontal asymptote. 
	
	\begin{Example}
		As an example consider $f(x) = 5 + \frac{\sin x}{\sqrt{x}}$. 
		
		We want to look at $\lim_{x \to \infty} f(x)$. By our limit laws we have 
		\begin{align*}
			\lim_{x\to \infty} f(x) = \lim_{x \to \infty} 5 + \lim_{x\to\infty} \frac{\sin x}{\sqrt{x}}. 
		\end{align*}
		Now we know what the first limit is but what about the second? (ask for guesses). We can actually show that $\lim_{x\to\infty} \frac{\sin x}{\sqrt{x}} = 0$.  
		
		We can use the squeeze theorem. Remember that 
		\begin{align*}
			-1 &\leq \sin x \leq 1 \\
			\frac{-1}{\sqrt{x}} &\leq \frac{\sin x}{\sqrt{x}}\leq \frac{1}{\sqrt{x}} 
		\end{align*}
		We know by the important limit at infinity that I mentioned that the functions on the left and right go to 0 as $x \to \infty$ so the function in the middle does as well. So going back to limit of $f(x)$ we see that it is 5. So there is a horizontal asymptote of $y = 5$. 
	\end{Example} 

\section{6/11}
\begin{enumerate}
	\item[a.] Short review of limits at infinity 
	\item[b.] Finish section on limits at infinity
	\item[c.] Worksheet
	\item[d.] Questions
\end{enumerate}

\subsection{Infinite Limits at Infinity}
Last class we look at limits at infinity and infinite limits. What happens we combine them? We get infinite limits at infinity. An easy example is looking at any polynomial, for example $y = x^3$. These limits are important as they tell us the behavior of polynomials for large $x$. 

This leads us to the following theorem that talks about limits at infinity for powers and polynomials
\begin{Theorem}
	Let $n$ be a positive integer and let $p$ be the polynomial $p(x) = a_n x^n + \cdots + a_2 x^2 + a_1 x + a_0$ where $a_n \neq 0$. 
	\begin{enumerate}
		\item[a.] $\lim_{x\to\pm \infty} x^n = \infty$, $n$ even 
		\item[b.] $\lim_{x\to\infty} x^n = \infty$ and $\lim_{x\to-\infty} x^n = -\infty$, $n$ odd 
		\item[c.] $\lim_{x\to\pm \infty} \frac{1}{x^n} = 0$ 
		\item[d.] $\lim_{x \to \pm \infty} p(x) = \lim_{x \to \pm \infty} a_n x^n = \pm \infty$ depending on $n$. 
	\end{enumerate}
\end{Theorem}

(look at some more polynomial examples)

While end behavior of polynomials is not that interesting or difficult to find, we move on the next level - rational functions. For example consider the functions 
\begin{align*}
	f(x) &= \frac{x^3 - 2x + 1}{2x + 4} \\
	g(x) &= \frac{4 x^4 - 2x^2}{x^4 - x^3 + x^2 -1} \\
	h(x) &= \frac{3x - 2}{x^2 - 4}
\end{align*}

So how do we find these limits? (walk through the dividing method). 

We have talked about vertical and horizontal asymptotes however there is one other type of asymptote that we can observe with rational functions - slant asmyptotes. What is a slant asmyptote? Consider a function like $y = x + \frac{1}{x}$. (draw a graph). Looking at the function we can see what is meant by a slant asymptote. Then how do we find such asymptotes? We need to use long division. Looking at another example - $f(x) = \frac{2x^2 + 6x - 2}{x + 1}$. 


To wrap up and summarize all of the work we've done with rational functions we have the following theorem - 
\begin{Theorem}
	Suppose $f(x) = \frac{p(x)}{q(x)}$ is a rational function where $p(x)$ is a polynomial of degree $m$ and $q$ is a polynomial of degree $n$ then 
	\begin{enumerate}
		\item[1)] If $m < n$ (degree of numerator less than denominator) then $\lim_{x\to\pm\infty} f(x) = 0$ and $y = 0$ is a horizontal asymptote of $f$. 
		\item[2)] If $m = n$ (degree of numerator = to deg of denominator) the $\lim_{x \to \pm\infty} f(x) = a_m/b_n$ and $y = a_m/b_n$ is a horizontal of $f$. 
		\item[3)] If $m > n$ then $\lim_{x \to \pm \infty} f(x) = \infty$ or $-\infty$ and $f$ has no horizontal asymptote. 
		\item[4)] If $m  = n+ 1$ then $\lim_{x \to \pm \infty} f(x) = \infty$ or $-\infty$ and $f$ has no horizontal asymptote but $f$ has a slant asymptote
		\item[5)] Assuming $f$ is in reduced form ($p$ and $q$ share no common factors), vertical asymptotes occur at the zeros of $q$. 
	\end{enumerate}
\end{Theorem}

As a note - what this theorem implies is that a rational function can only have 1 horizontal asymptote if there is one. 

Now let us look at algebraic functions. Consider the example $f(x) = \frac{10 x^3 - 3x^2 + 8}{\sqrt{25x^6 + x^4 + 2}}$. To find limits at infinity we need to use the dividing trick from before with a slight modification. We need to take the square root of the highest power in the denominator and then look at it based on if we are going to positive or negative infinity. 

\section{6/14}
\subsection{Continuity at a point}

The next important topic we are going briefly cover is continuity. (Draw a couple pictures) What is the difference between these functions? (make sure to have ones with jumps/breaks). Graphs with jumps or breaks are not continuous functions.... another way of thinking about it is a continuous function can be drawn without lifting the pencil. 

Now using that informal definition we can get away with figuring out if a function is continuous or not most of the time. However for more complex cases we cannot - like the function $h(x) = \begin{cases}
	x \sin \frac{1}{x} & x \neq 0 \\ 0 & x = 0
\end{cases}$. 

We cannot tell if $h$ is continuous at $0$ based on the graph. So we need a more formal definition. 

\begin{Definition}
	(Continuity at a point) A function is continuous at $a$ if $\lim_{x \to a} f(x) = f(a)$. 
\end{Definition}

This definition implies a few things - 1) Both $\lim_{x\to a}f(x)$ and $f(a)$ must exist and 2) they must be equal. Let us return to our example function from before and take a look at if it is continuous at $0$ using this formal definition. 

Let us look at some more examples to see if they are continuous at a certain point - 
\begin{Example}
	$$f(x) = \frac{3x^2 + 2x+ 1}{x - 1}$$ at $a = 1$. (Not continuous)
\end{Example}
\begin{Example}
	However look at same function at $a = 2$ (it is continuous)
\end{Example}

Similar to how we had limit rules/properties we have some properties for continuity at a point - 
\begin{Theorem}
	If $f$ and $g$ are continuous at $a$, then the following functions are also continuous at $a$ (assume $c$ is a constant and $n > 0$ an integer)
	\begin{enumerate}
		\item[a.] $f + g$
		\item[b.] $f - g$
		\item[c.] $c f$ 
		\item[d.] $f g$ 
		\item[e.] $f/g$ assuming $g(a) \neq 0$
		\item[f.] $(f)^n$ 
	\end{enumerate}
\end{Theorem}

The proofs of most of these properties follows from the limit properties. By the continuity properties we know the following
\begin{Theorem}
	\begin{enumerate}
		\item[a.] A polynomial function is continuous for all $x$. 
		\item[b.] A rational function $p/q$ is continuous for all $x$ for which $q(x) \neq 0$
	\end{enumerate}
\end{Theorem}

There is one function operation we have yet to really touch on and that is composition of functions. The following theorem helps us talk about continuity of the composition of functions - 
\begin{Theorem}
	If $g$ is continuous at $a$ and $f$ is continuous at $g(a)$ then the composite function $f \circ g$ is continuous at $a$. Furthermore, the limit of $f \circ g$ can be evaluated by direct substitution, that is $\lim_{x\to a} f(g(x)) = f(g(a))$. 
\end{Theorem}


From the previous theorem we can get the following results about the limits of composite functions - 
\begin{Theorem}
	\begin{enumerate}
		\item[1.] If $g$ is continuous at $a$ and $f$ is continuous at $g(a)$ then $\lim_{x\to a} f(g(x)) = f(\lim_{x\to a} g(x))$. 
		\item[2.] If $\lim_{x\to a} g(x) = L$ and $f$ is continuous at $L$ then $\lim_{x\to a} f(g(x)) = f(\lim_{x\to a} g(x))$. 
	\end{enumerate}
\end{Theorem}

This theorem is helpful in evaluating limits of composite functions - 
As an example let us look at 
\begin{Example}
	$$f(x) = \left(\frac{x^4 - 2x + 2}{x^6 + 2x^4 + 1}\right)^{10}$$. 
	
	What if we want to find $\lim_{x\to 0} f(x)$. (explain how to use the theorem above)
\end{Example}
\subsection{Continuity on an Interval} 
\begin{Definition}
	A function $f$ is continuous on an interval $I$ if it is continuous at all points of $I$. If $I$ contains its endpoints, continuity on $I$ means continuous from the right $(\lim_{x\to a^+} f(x) = f(a)$) or left ($\lim_{x \to b^-} f(x) =f(b))$ at the endpoints. 
\end{Definition}

An example - 
\begin{Example}
	Determine the intervals of continuity for $$f(x) = \begin{cases}
		x^2 + 1 & x \leq 0 \\ 3x + 5 & x > 0
	\end{cases}$$

	($f$ is left continuous at 0 so the intervals are $(-\infty, 0]$ and $(0, \infty)$). 
\end{Example}
\subsection{Continuity with Roots} 
\begin{Theorem}
	Assume $n$ is a positive integer. If $n$ is an odd integer then $(f(x))^{1/n}$ is continuous at all points at which $f$ is continuous. If $n$ is even then $(f(x))^{1/n}$ is continuous at all points $a$ at which $f$ is continuous and $f(a) > 0$.
\end{Theorem}

\begin{Example}
	The most common example of a function with a root in it would be a semicircle - $f(x) = \sqrt{9 - x^2}$. By the previous theorem since $f$ involves an even root, we know that $f$ is continuous when $9 - x^2 > 0$ and is continuous, so (-3, 3). We want to look at the endpoints $-3$ and $3$. So we then look at the limits $\lim_{x\to -3^+} f(x)$ and $\lim_{x\to 3^-} f(x)$ and see that $f$ is right cont. at -3 and left cont. at 3 so the interval of continuity for $f$ is $[-3, 3]$. 
\end{Example}
\subsection{Continuity of Trig Functions} 
\begin{Theorem}
	The functions $\sin x, \cos x, \tan x, \cot x, \sec x,$ and $\csc x$ are continuous at all points of their domains. 
\end{Theorem}

\subsection{Intermediate Value Theorem} 
A very common problem we run into is trying to find $x$ such that $f(x) = L$ for some $L$. However, before even trying to find solutions how do we know that there even exists a solution?  We use what is called the Intermediate Value Theorem (IVT). 
\begin{Theorem}
	(IVT) Suppose $f$ is continuous on the interval $[a, b]$ and $L$ is a number strictly between $f(a)$ and $f(b)$. Then there exists at least one number $c$ in (a, b) satisfying $f(c) = L$. 
\end{Theorem}
\begin{Example}
For example we can use this to check if the equation $f(x) = x^3 + x + 1 = 0$ has a solution on the interval $[-1, 1]$. 
\end{Example}

\section{6/16}

\subsection{Derivatives} 

Now that we have talked about limits and continuity we can get into the first major topic of Calculus - derivatives. But what are derivatives? This goes back to the first topic we talked about in this class - instantaneous rate of change (or the slope of a tangent line). If we look at what we did in the first section we had secant lines and then we took smaller and smaller intervals to estimate instantenous rate of change. This leads us to the definition of the derivative. (draw a picture)

\begin{Definition}
	The derivative of a function $f$ at a point $a$ is given by either of the following limits, provided that the limits exist and $a$ is in the domain of $f$ - \begin{align}
		f'(a) &= \lim_{x \to a} \frac{f(x) - f(a)}{x - a} \\
		f'(a) &= \lim_{h \to 0} \frac{f(a + h) - f(a)} {h}.
	\end{align}
	If $f'(a)$ exists we say that $f$ is differentiable at $a$. 
\end{Definition} 

\begin{Remark}
	A note about notation. If we are using function notation the derivative of $f$ at a point $a$ is denoted as $f'(a)$. If we are using Leibniz notation the derivative of $f$ at a point $a$ is denoted as $\left.\frac{df}{dx}\right|_{x = a}$. Another possible notation we can use is $y'(a)$ or $\left.\frac{dy}{dx}\right|_{x = a}$. .   
\end{Remark}

We can now use this to calculate the tangent lines for functions at a point. 
\begin{Example}
	For example take $f(x) = \sqrt{2x + 1}$. We can then find (using the definition) the derivative at $x = 2$ and then find the slope of the tangent line at that point. 
	
	So going by definition we calculate $f'(2)$ as 
	\begin{align*}
		f'(2) &= \lim_{x \to 2} \frac{f(x) - f(2)}{x - 2} \\
			  &= \lim_{x \to 2} \frac{\sqrt{2x + 1} - \sqrt{5}}{x - 2} \\
			  &= \lim_{x \to 2} \frac{\sqrt{2x + 1} - \sqrt{5}}{x - 2} \cdot \frac{\sqrt{2x + 1} + \sqrt{5}}{\sqrt{2x + 1} + \sqrt{5}} \\
			  &= \lim_{x \to 2} \frac{(\sqrt{2x + 1})^2 - (\sqrt{5})^2}{(x - 2) (\sqrt{2x + 1} + \sqrt{5})} \\
			  &= \lim_{x \to 2} \frac{2x + 1 - 5} {(x - 2)(\sqrt{2x + 1} + \sqrt{5})} \\
			  &= \lim_{x \to 2} \frac{2x - 4}{(x - 2)(\sqrt{2x + 1} + \sqrt{5})} \\
			  &= \lim_{x \to 2} \frac{2}{\sqrt{2x + 1} + \sqrt{5}} = \frac{2}{2\sqrt{5}} = \frac{1}{\sqrt{5}}.
	\end{align*}

	So now that we know the slope we can find the equation of the tangent line. To do so we will use the point-slope form a linear function and this gives that the tangent is $y - \sqrt{5} = \frac{1}{\sqrt{5}} (x - 2)$. 
\end{Example}

We have talked about the derivative at a point, but we can also just talk about the derivative as a function of $x$. 
\begin{Definition}
	The derivative of $f$ is the function $$f'(x) = \lim_{h \to 0} \frac{f(x + h) - f(x)}{h}$$ provided the limit exists and $x$ is in the domain of $f$. If $f'(x)$ exists we say $f$ is differentiable at $x$. If $f$ is differentiable at every point of an open interval $I$, we say that $f$ is differentiable on $I$.  (An open interval is an interval that does not contain its endpoints). 
\end{Definition}

As an example using the definition we have
\begin{Example}
	Let us find the derivative of $f(x) = -x^2 + 6x$ by definition. 
\end{Example}

Similar to how we talked about notation for the derivative at a point we have multiple different ways to notate the derivative of $f$ - $f'(x), \frac{df}{dx}, \frac{d}{dx}(f(x)), D_x (f(x)), \frac{dy}{dx}, $ and $y'$. You can use any notation that you prefer as long as you are consistent. 

I mentioned how continuity is related to derivatives. But how? Continuity is a result of differentiabilty. 
\begin{Theorem}
	If $f$ is differentiable at $a$ then $f$ is continuous at $a$. 
\end{Theorem} 

Another way to think about this is using the "contrapositive" - If $f$ is not continuous at $a$ then $f$ is not differentiable at $a$.

\begin{Remark}
	It is important to note that the opposite is NOT true - You can be continuous at a point but not differentiable. Consider $f(x) = |x|$. This function is continuous at $0$, however it is not differentiable. The reason being that the limit does not exist (from the right we have 1 and left we have -1). 
\end{Remark} 



\subsubsection{Graphs of Derivatives}
An interesting thing to do is look at the graphs of a derivative of a function and see how it relates to the original function. We will come back to this later when we can take the derivatives of more functions using derivative rules, but for now let us use the fact that the derivative represents the slope of the tangent line to sketch the graphs of the derivative. 


\section{6/17}
Now that we have defined how to take a derivative, we will talk about rules to help make calculating derivatives easier. 
\subsection{Derivative Rules} 

The first two rules we will talk about are the rules for taking derivatives of constant functions and derivatives of a simple monomial - $x^n$. 

If we look at the graph of a constant function we see that we have a slope of zero. This leads to the very simple constant rule
\begin{Theorem}
	If $c$ is a real number, then $\frac{d}{dx} (c) = 0$. 
\end{Theorem}

Next let us look at the power rule. The power rule is as follows 
\begin{Theorem}
	$\frac{d}{dx} x^n = n x^{n  - 1}$, where $n$ is a positive integer
\end{Theorem}

How do we get the rule? There is proof in the textbook, but I will show a different proof. The proof I will show uses the binomial theorem, which tells us how to expand $(x + y)^n$. For those who don't remember what this is the expansion goes as follows 
\begin{align*}
	(x + y)^n = x^n + \begin{pmatrix}
		n \\ n - 1
	\end{pmatrix} x^{n - 1} y + \begin{pmatrix}
	n \\ n - 2
\end{pmatrix} x^{n - 2} y^2 + \cdots + \begin{pmatrix}
n \\ 2 
\end{pmatrix} x^2 y^{n - 2} + \begin{pmatrix}
n \\ 1
\end{pmatrix} xy^{n - 1} + y^{n}
\end{align*}

So let us work through from the definition of $f'(x)$ from last class. 

\begin{align*}
	f'(x) &= \lim_{h \to 0} \frac{f(x + h) - f(x)}{h} \\
		  &= \lim_{h \to 0} \frac{(x + h)^n - x^n}{h} \\
		  &= \lim_{h \to 0}  \frac{x^n + \begin{pmatrix}
		  		n \\ n - 1
		  	\end{pmatrix} x^{n - 1} h + \begin{pmatrix}
		  		n \\ n - 2
		  	\end{pmatrix} x^{n - 2} h^2 + \cdots + \begin{pmatrix}
		  		n \\ 2 
		  	\end{pmatrix} x^2 h^{n - 2} + \begin{pmatrix}
		  		n \\ 1
		  	\end{pmatrix} xh^{n - 1} + h^{n} - x^n}{h} \\
	  	  &= \lim_{h \to 0} \frac{\begin{pmatrix}
	  	  		n \\ n - 1
	  	  	\end{pmatrix} x^{n - 1} h + \begin{pmatrix}
	  	  		n \\ n - 2
	  	  	\end{pmatrix} x^{n - 2} h^2 + \cdots + \begin{pmatrix}
	  	  		n \\ 2 
	  	  	\end{pmatrix} x^2 h^{n - 2} + \begin{pmatrix}
	  	  		n \\ 1
	  	  	\end{pmatrix} xh^{n - 1} + h^{n}}{h} \\
  	  	&= \lim_{h \to 0} \begin{pmatrix}
  	  		n \\ n - 1
  	  	\end{pmatrix} x^{n - 1}  + \begin{pmatrix}
  	  		n \\ n - 2
  	  	\end{pmatrix} x^{n - 2} h + \cdots + \begin{pmatrix}
  	  		n \\ 2 
  	  	\end{pmatrix} x^2 h^{n - 3} + \begin{pmatrix}
  	  		n \\ 1
  	  	\end{pmatrix} xh^{n - 2} + h^{n - 1} \\
    	&= \begin{pmatrix}
    		n \\ n - 1
    	\end{pmatrix} x^{n - 1} = \frac{n!}{(n - 1)! 1!} x^{n - 1} = n x^{n - 1}. 
\end{align*}

While we have only shown that this is true for positive integers, in fact the power rule applies for ALL real numbers - 
\begin{Theorem}
	(general power rule) $\frac{d}{dx} x^n = n x^{n - 1}$, $n$ is any real number. 
\end{Theorem}

The next rule that we will look at is the constant multiple rule. 
\begin{Theorem}
	If $f$ is differentiable at $x$ and $c$ is a constant, then $\frac{d}{dx} (c f(x)) = c f'(x)$. 
\end{Theorem}

We can also work this one through from the definition - 
\begin{align*}
	\frac{d}{dx} (cf(x)) &= \lim_{h \to 0} \frac{(c f(x + h)) - (c f(x))}{h} \\
						 &= \lim_{h \to 0} \frac{c (f(x + h) - f(x))}{h} \\
						 &= c \lim_{h \to 0} \frac{f(x + h) - f(x)}{h} \\
						 &= c f'(x).
\end{align*}

The next rule is the sum rule. Simply put this is as follows - 
\begin{Theorem}
	If $f$ and $g$ are differentiable at $x$ then $\frac{d}{dx}(f(x) + g(x)) = f'(x) + g'(x)$. 
\end{Theorem}

Again let us work through this by definition - 
\begin{align*}
	\frac{d}{dx} (f(x) + g(x)) &= \lim_{h \to \infty} \frac{(f(x + h) + g(x+h)) - (f(x) + g(x))}{h} \\
						       &= \lim_{h \to \infty} \frac{f(x + h) - f(x)}{h} + \frac{g(x + h) - g(x)}{h} \\
						       &= \lim_{h \to \infty} \frac{f(x + h) - f(x)}{h} + \lim_{h \to \infty} \frac{g(x + h) - g(x)}{h} \\
						       &= f'(x) + g'(x). 
\end{align*}

We can generalize this sum rule to work with more than just two summands. 

There are two more rules we will look at - the product and quotient rule. 

These two rules are 
\begin{Theorem}
	\begin{align*}
		\frac{d}{dx}(f(x) g(x)) &= f'(x) g(x) + f(x) g'(x) \\
		\frac{d}{dx}\left(\frac{f(x)}{g(x)}\right) &= \frac{f'(x) g(x) - f(x) g'(x)}{(g(x))^2}.
	\end{align*}
\end{Theorem}

Both of these we can also show from the definition. However they are more complicated than any of the rules we have done before. We will go through the just the product rule for right now - 
\begin{align*}
	\frac{d}{dx}(f(x) g(x)) &= \lim_{h \to 0} \frac{f(x + h)g(x + h) - f(x) g(x)} {h} \\
						    &= \lim_{h \to 0} \frac{f(x + h) g(x + h) - f(x) g(x + h) + f(x)g(x + h) - f(x) g(x)}{h} \\
					    	&= \lim_{h \to 0} \frac{(f(x + h) - f(x)) g(x + h) + f(x) (g(x + h) - g(x))}{h} \\
					    	&= \lim_{h \to 0} \frac{f(x + h) - f(x)}{h} g(x + h) + f(x) \lim_{h \to 0} \frac{g(x + h) - g(x)}{h} \\ 
					    	&= f'(x) g(x) + f(x) g'(x).  
\end{align*}


Now that we have gone over all of these rules, let's summarize all of the rules we have 
\begin{align}
	\textnormal{(Constant)} \frac{d}{dx} (c) &= 0 \\ 
	\textnormal{(Power)} \frac{d}{dx} (x^n) &= n x^{n - 1} \textnormal{ n is any real number} \\
	\textnormal{(Constant Multiple)} \frac{d}{dx}(c f) &= c f'(x) \\ 
	\textnormal{(Sum)} \frac{d}{dx}(f  + g) &= f'(x) + g'(x) \\
	\textnormal{(Product)} \frac{d}{dx} (f g) &= f' g + f g' \\
	\textnormal{(Quotient)} \frac{d}{dx} \left(\frac{f}{g}\right) &= \frac{f' g - f g'}{g^2} 
\end{align}


Now let us get into some examples. Most of the time you will be using more than one rule at the same time. So we will look over a few examples, before moving on to the worksheet that will have many more. 

\begin{Example}
	Polynomials. One of the most common functions. Let us look at a not so simple problem. Let us find the derivative of $y = (2x^3 - 4)(x^2 +3x + 2)$. There are two ways to solve this. One is to FOIL out (expand) and then take the derivative. Or we can use the product rule. Let us do both. 
\end{Example}

\begin{Example}
	Let us look at an example that pull all of the rules together. Let us find the derivative of $$y = \frac{4x(2x^3 - 3x^{-1})}{x^2 + 1}$$. There are two ways to approach a rational function - either we use the quotient rule, or we rewrite the rational function as a product and use the product rule. 
	
	Let us use try to do both. (Ans. $$\frac{8x(2x^4 + 4x^2 + 3)}{(x^2 + 1)^2}$$)
\end{Example}

\section{6/18}

A few last remaining misc details about derivatives - 

\subsection{Higher Order Derivatives} We can take higher-order derivatives. So the second, third etc. (show how to notate it). In particular the second derivative will become important as we keep going 
\subsection{Derivatives as Rates of Change} 

\section{6/21}

Today is the first exam. 
\begin{enumerate}
	\item[a.] Begin with a quick review/ask if there are any questions. 
	\item[b.] Give exam (60 min)
	\item[c.] Start on some new material.
\end{enumerate}

\subsection{Review before Exam}

A few common mistakes that happened on the quizzes that I noted as I was grading them. 

\begin{enumerate}
	\item[1.] Not reading the directions. If the directions say use the definition use the definition. 
	\item[2.] Not taking the derivative correctly. The one that tripped the most people up was the term in the derivative rules quiz $\frac{10}{x^2 \sqrt{x}}$. Many people used the product rule on the denominator, but that is incorrect. You have to either rewrite this as $x$ to a power or use the quotient rule. The other mistake was in canceling during the algebra after using the quotient rule. Simplifying is where many people make mistakes so remember to take the time to make sure you have done the algebra properly. 
\end{enumerate}

\subsection{Derivatives of Trig Functions}
The power rule we talked about last week is our main tool for dealing with derivatives of polynomials, rational and algebraic functions. However we have yet to talk about two other common types of functions - Trig and Exponential functions. 

Let us first talk about the derivatives for trig functions. 
\subsubsection{Two Special Limits}
There are two special limits that we will need to know to find the derivatives of trig functions - 
\begin{Theorem}
	\begin{align*}
		\lim_{x \to 0} \frac{\sin x}{x} &= 1 \\
		\lim_{x \to 0} \frac{\cos x - 1}{x} &= 0.
	\end{align*}
\end{Theorem}

The proof of these two limits are in the textbook. The first one is a classic example of using the Squeeze Theorem. Here we take a sector of the unit circle degree $x$. Then we draw a triangle inside the circle and a triangle outside. Then calculating the areas of the triangles we get that the area of the sector follows the following chain - $\frac{1}{2} \cos x \sin x < \frac{x}{2} < \frac{1}{2} \tan x$. Rearranging this we can squeeze the quantity that we want to find. 

Now we can calculate the derivatives of sine and cosine by definition. 

Starting with $f(x) = \sin x$ we have
\begin{align*}
	f'(x) &= \lim_{h \to 0} \frac{\sin (x + h) - \sin x} {h} \\
		  &= \lim_{h \to 0} \frac{\sin x \cos h + \cos x \sin h - \sin x} {h} \\
		  &= \lim_{h \to 0} \frac{\sin x (\cos h - 1) + \cos x \sin h}{h} \\
		  &= \sin x \lim_{h \to 0} \frac{\cos h - 1}{h} + \cos x \lim_{h \to 0} \frac{\sin h}{h} \\
		  &= \cos x
\end{align*}

Similarly for $g(x) = \cos x$ we have 
\begin{align*}
	g'(x) &= \lim_{h \to 0} \frac{\cos (x + h) - \cos x} {h} \\
	      &= \lim_{h \to 0} \frac{\cos x \cos h - \sin x \sin h - \cos x}{h} \\
		  &= \cos x \lim_{h \to 0} \frac{\cos h - 1}{h} - \sin x \lim_{h \to 0} \frac{\sin h}{h} \\
		  &= -\sin x.
\end{align*}

This gives us that 
\begin{Theorem}
	\begin{align*}
		\frac{d}{dx} \sin x &= \cos x \\
		\frac{d}{dx} \cos x &= - \sin x.
	\end{align*}
\end{Theorem}

Using this and the rules we covered last week we can now find derivatives involving trig functions. 
\begin{Example}
	Find the derivative of $y = x^2 \cos x$. Here we would use the product rule. 
	The derivative of $y = \frac{1 - \sin x}{1 + \sin x}$. Here we would use the quotient rule. 
\end{Example}

Again with finding derivatives of other trig functions like $\tan x$ we would just write these functions in terms of $\sin$ and $\cos$ and then use the appropriate derivative rules. Once we do this we can find the derivatives to all simple trig functions
\begin{Theorem}
\begin{align*}
	\frac{d}{dx} \sin x &= \cos x \\
	\frac{d}{dx} \cos x &= - \sin x \\
	\frac{d}{dx} \tan x &= \sec^2 x \\
	\frac{d}{dx} \sec x &= \sec x \tan x \\ 
	\frac{d}{dx} \csc x &= - \csc x \cot x \\
	\frac{d}{dx} \cot x &= - \csc^2 x.
\end{align*}
\end{Theorem}

Like I mentioned last class we can look at higher order derivatives. Trig functions are an example where we can always take a derivative of a trig function and never get 0. 

\subsection{Derivatives of Exponential Functions}

The other type of functions we have not take the derivative of are exponential functions - such as $e^x$. This is a special exponential function since it has $e$ as the base. A more general exponential function is $b^x$ where $b$ is any real number. 

(Ask if people remember the rules for dealing with exponential functions)

We can also talk about the limits at infinity for exponetial functions and those depend on whether our base is greater than 1 or inbetween 0 and 1. 

We can try and take the derivative of a general exponetial function using the definition - 
\begin{align*}
	\frac{d}{dx} b^x &= \lim_{h \to 0} \frac{b^{x + h} - b^x}{h} \\
					 &= \lim_{h \to 0} \frac{b^x \cdot b^h - b^x}{h} \\
					 &= b^x \lim_{h \to 0} \frac{b^h - 1}{h}.
\end{align*}

It is interesting to note that the limit that remains is just the definition of $\left.\frac{d}{dx} b^x \right|_{x = 0}$. We will not cover what this limit is in this course, but what we will do is define this - 
\begin{Definition}
 Define $e$ as the number such that $\lim_{h \to 0} \frac{e^h - 1}{h} = 1$. 
\end{Definition}

If we do this we can see that by the work with did using the definition of the derivative - 
\begin{Theorem}
$$\frac{d}{dx} e^x = e^x$$
\end{Theorem}

For the purposes of this course, this is the only exponential function we will be working with. 

\subsection{Derivatives as a Rate of Change Revisited}

Let us come back to looking at derivatives as a rate of change now that we can take the derivatives of more functions. There are a few common types of word problems that come up when talking about derivatives. One of the most common is dealing with position functions and velocity. 

We originally started the class with one such example, where we were throwing a ball and had a function that measured the hieght of the ball after it was thrown. 
\begin{Example}
	We will throw a ball off a building of height 20 ft at a speed of 32 ft/s. The motion of the ball is then given by the function $h(t) = 20 + 32t - 16t^2$. This function $h$ is called a position function. Another common letter used for a position function is $s$. 
	
	Now if we want to find the velocity of a ball at a point we calculated the limit as the secant line became a tangent line - in other words, the velocity function $v(t)$ is just the derivative of $h$, $v(t) = h'(t)$. 
	
	We can go even further, if we want to find the instantaneous change in velocity, i.e. the acceleration of the ball, we take the derivative again - giving the relation $a(t) = v'(t)$. 
	
	In general - the velocity is the derivative of the position function and the acceleration is the derivative of the velocity. This is the key relationship to take away from this situation. 
\end{Example}

There are many other examples that we can see of derivatives in action in other fields, from physics to economics. (There just was a problem on your first exam). 

\section{6/23}

\begin{enumerate}
	\item[a.] Start class with touching on rates of change (the physics example)
	\item[b.] Chain Rule
	\item[c.] Worksheet
\end{enumerate}

\subsection{The Chain Rule}

The Chain Rule is one of the most powerful/important derivative rules we have, while at the same time also being one of the hardest to understand at a first glance. Which is why a lot of this is going to be either done through examples and through you guys practicing a lot through worksheets. 


Let us start with the statement of the chain rule. There are two main ways that the chain rule is stated in textbooks - 
\begin{Theorem} 
 Suppose that $y = f(u)$ is differentiable at $u = g(x)$ and $u = g(x)$ is differentiable at $x$. The composite function $y = f(g(x))$ is differentiable at $x$ and its derivative can be expressed in two equivalent ways: 
 \begin{align*}
 	\frac{dy}{dx} &= \frac{dy}{dy} \frac{du}{dx} \\ 
	\frac{d}{dx} f(g(x)) &= f'(g(x)) \cdot g'(x). 
\end{align*}
\end{Theorem}

This looks complicated, but can we get this rule from the definition? The answer is yes. It is interesting to look at as it requires some manipulation of limits in a different way than what we have done before - 
\begin{align*}
	\frac{d}{dx} f(g(x)) &= \lim_{h \to 0} \frac{f(g(x + h)) - f(g(x))}{h} \\
						 &= \lim_{h \to 0} \frac{f(g(x + h)) - f(g(x))}{h} \cdot \frac{g(x + h) - g(x)}{g(x + h) - g(x)} \\ 
						 &= \lim_{h \to 0} \frac{f(g(x + h)) - f(g(x))}{g(x + h) - g(x)} \cdot \lim_{h \to 0} \frac{g(x + h) - g(x)}{h} 
\end{align*}

The second limit is just the definition of $g'(x)$, so we need to look at the first. Let $u = g(x + h)$ and $v = g(x)$. Notice that as $h \to 0$ then $u \to v$. Substituting this into our first limit we get
\begin{align*}
	\lim_{h \to 0} \frac{f(g(x + h)) - f(g(x))}{g(x + h) - g(x)} &= \lim_{u \to v} \frac{f(u) - f(v)}{u - v} \\
			&= f'(v) = f'(g(x)).
\end{align*}

We get the second to third line by the definition of the derivative at a point. 

Putting these two together gives us the result that $\frac{d}{dx} f(g(x)) = f'(g(x)) g'(x)$. 


At the start it always helps to write everything out to make sure you are using the chain rule properly. That way you can keep track of your $f$'s , and $g$'s. 
(go over a few examples)

Remember, as with all derivative rules the chain rule has to be used properly along with other rules. We can now touch on a few examples of problems mentioned in previous classes. We can also talk about how to deal with a quotient without using the quotient rule. (maybe do an example using both) 

\subsubsection{A Special Case}

A special case of the chain rule is when we are using it on a power of a function ex. $(g(x))^p$. In this case we have - 
\begin{Theorem}
	If $g$ is differentiable for all $x$ in its domain and $p$ a real number then $$\frac{d}{dx} (g(x))^{p} = p (g(x))^{p - 1} g'(x)$$. 
\end{Theorem}

This is one of the most common scenarios when using the chain rule so keeping this always in the back of your mind when working with taking derivatives is helpful. 

(do some more examples)

\subsubsection{More complicated Compositions} 

It is possible that we will run into a composition of three or more functions. This just means that we need to be even more careful as to how we apply the chain rule. In these cases we will usually be applying the chain rule more than once.

As an example we can look at $e^{\sin 2x}$ or $\sin (\sqrt{x^2 + 2})$. We can think of these functions as $f(g(h(x)))$ a composition of three functions. So then $\frac{d}{dx}f(g(h(x)))$ given by the Chain Rule is 
\begin{align*}
	\frac{d}{dx} f(g(h(x))) &= f'(g(h(x))) [g(h(x))]' \\
	&= f'(g(h(x))) \cdot g'(h(x)) h'(x) \\
	&= f'(g(h(x))) g'(h(x)) h'(x).
\end{align*}

We can use this on the example functions given to find their derivatives. 

\section{6/25} 

We are now going to be moving away from the theory of how to take derivatives and move towards applying derivatives to solve different types of problems. At this point you should be very comfortable using the derivative rules as well as comfortable using implicit differentiation as the first thing we are going to touch on relies on you being able to implicitly differentiate properly. 


\subsection{Related Rates}

So related rates. We are going back to the idea of derivatives as rates of change. In particular we are going to be looking at how variables change with respect to time. The defining feature of a related rates problem is that will be looking at how two or more variables that are related change with time. 

For example a setup for a problem would be something like - we have two planes approaching an airport with known speeds, one plane from the west and another from the north. How fast is the distance between the two planes changing. This problem has three related variables - the distance between the two planes and the position of the two planes. They are all related and the goal is to find the rate of change of one of those variables at a moment of time. 

So let us work through an example and then I'll try to give a general way of approaching a related rates problem. 
\begin{Example}
	Consider an oil rig leak in the ocean where the oil spreads in a circular patch around the rig. If the radius of the oil patch increases at a rate of $30 m/hr$, how fast is the are of patch increasing when the patch has a radius of $100$ meters. 
	
	So first we need to consider what variables are changing. In this case there are two - the radius of the oil spill and the area of the oil spill. We need to find a formula that relates these two variables. What is a formula that relates the area of a circle and its radius? Well the formula for the area of a cirle - $A = \pi r^2$. As our variables are functions of time we can also express this in our formula - $A(t) = \pi (r(t))^2$.  
	
	What is our goal? Our goal is to find the rate of change of the area of the circle, which is $A'(t)$. We are given that the radius increases at a rate of 30 m/hr which means that $r'(t) = 30$ m/hr.We know that the radius that we want to examine is 100 m, so $r(t) = 100$m. Now going back to our formula how do we introduce derivatives? Implicit differentiation! So let us implicitly differentiate and we get $A'(t) = 2 \pi r(t) r'(t)$. Now we can plug in the values that we know to find $A'(t)$ - $A'(t) = 2 \pi (100)(30) = 6000\pi$. Now are we done? No! Never forget your units. What are our units here? They are $m^2$/hr. So the area of the oil spill increases at a rate of 6000$\pi$ $m^2$/hr when the radius is 100 m. 
\end{Example}

Now that we have seen how to work through a problem here is a procedure that you can use to help you solve a related-rates problem - 
\begin{enumerate}
	\item[1.] After reading the problem, sketch a picture (usually helpful) and make a note of the given information. 
	\item[2.] Write one or more equations that relate the quantities you are going to be working with 
	\item[3.] Use implicit differentiation with respect to time $t$ to introduce the rates of change. 
	\item[4.] Solve for the quantity that we want to find and then substitute the given values that you noted down in step 1. 
	\item[5.] Double check you have not forgotten your units and pay attention to signs. 
\end{enumerate}

Using this procedure let us go back to our example of two planes. 
\begin{Example}
	We two small planes approaching an airport, one flying due west at 120 mi/hr and the other flying due north at 150 mi/hr. Assuming a constant elevation, how fast is the distance between the planes changing when the westbound plane is 180 mi from the airport and the northbound plane is 225 miles from the airport. 
	
	Ok let us do the first step- draw a picture and make a note of the given information. So we have a triangle and let us label the side with the plane going north $y$ and the one withe the westbound plane as $x$ and then the distance between the two planes (hypotenuse) as $z$. These are all functions depending on $t$. We know that the speed of the westbound plane is $\frac{dx}{dt} = -120$ mi/hr and the speed of the northbound plane is $\frac{dy}{dt} = - 150$ mi/hr. Why are these rates negative? Because they are coming towards the airport not going away from it. Our reference point (where we are standing if you want to think of it that way) is the airport so the planes are coming towards us and not away so the - signs. (Can also think of it as the distance between the planes and the airport is shrinking). 
	
	Now what is our equation to relate them? Well what equation goes with a right triangle? The pythogorean theorem. $z^2 = x^2 + y^2$. So now let us do step three - implicitly differentiate. Implicitly differentiating gives us 
	\begin{align*}
		2z \frac{dz}{dt} &= 2x \frac{dx}{dt} + 2y \frac{dy}{dt} \\
		\frac{dz}{dt} &= \frac{x \frac{dx}{dt} + y \frac{dy}{dt}}{z}.
	\end{align*}
	Now that we did part of step 4 we can then substitute in what we know. Our $x = 180$, $y = 225$, and we can find $z$ using the Pyathgorean theorem to be about 288 mi. Plugging these in gives us that $\frac{dz}{dt} = - 192$ mi/hr. 
\end{Example}

A lot of related rates is going to be just working through problems like this over and over again. The Pythagorean Thm is one of the common formulas you will need to know to solve related rates problems. Other problems will include dealing with volume and areas of standard shapes. 

Let us look at a slightly more involved example that requires some more knowledge of triangles - 
\begin{Example}
	Let us say that coffee is draining out of a conical filter at a rate of 2.25 cubed inches per min. If the cone is 5 in tall and has a radius of 2 in, how fast is the coffee level dropping when the coffee is 3 in deep. 
	
	So following the procedure let us sketch the problem. We then have a sketch of the cone and looking at this we can see the three relevant variables - the volume $V$, the radius $r$ and the height $h$. We want to find $\frac{dh}{dt}$  at the instant when $h = 3$ in and $\frac{dV}{dt} = - 2.25$ cubic inches per min. We will also draw this cross section which is a triangle that will come in handy in a bit. Now we want to write our relevant formula and that would be the volume of a cone which is $V = \frac{1}{3} \pi r^2 h$. Now in this equation we have to deal with $r$ the radius. In our problem statement we are only given that the radius of the cone is 2in but we have no info about the radius of the level of coffee in the cone. This is where we use the triangle we drew. By similar triangles we know that ration of long side with the short side is $\frac{r}{h} = \frac{2}{5}$. Using this we can find an expression for $r$ in terms of $h$ - $r = \frac{2}{5}h$. We then substitute this into our volume equation and get $V = \frac{4}{75}\pi h^3$. 
	
	Now we can implicitly differentiate and get $\frac{dV}{dt} = \frac{4}{25}\pi h^2 \frac{dh}{dt}$. Solving for the value we want to find which is rate of change of height we get $\frac{dh}{dt} = \frac{25 dV/dt}{4\pi h^2}$. Substituting what we know that $\frac{dh}{dt}$ is roughly -0.497 in/min. 
\end{Example}


Another common type of related rates problem deals with rate of change over time of an angle. Here is an example of such a problem - 
\begin{Example}
	An observer stands 200 m from the launch site of a hot-air balloon at an elevation equal to the elevation of the launch site. The balloon rises vertically at a constant rate of 4 m/s. How fast is the angle of elevation of the balloon increasing 30 seconds after the launch. 
	So again let us draw a picture. We have a right triangle where the base of the triangle is 200 m long, the height is $y$ and the angle is given by $\theta$. We know that $\frac{dy}{dt} = 4$ m/s and we want to find $\frac{d\theta}{dt}$. 
	
	We need an equation to relate these two variables - so we turn to trig functions. In this case the equation that relates the two is $\tan \theta = \frac{y}{200}$. Then implicitly differentiating gives us $\sec^2 \theta \frac{d\theta}{dt} = \frac{1}{200}\frac{dy}{dt}$. Now solve for what we want - $\frac{d\theta}{dt} = \frac{dy/dt \cdot \cos^2 \theta}{200}$. So we need to find $\theta.$ After 30s the height of the balloon is $120$m. Using pythagorean theorem we know that hypothenuse has length approx 233.24 m and then $\cos \theta$ is approx 0.86. Using this we get that $\frac{d\theta}{dt}$ is approx 0.015 rad/s. 
\end{Example}

These are examples of some of the most common types of related rates problems. 

\section{6/28}
\subsection{Applications of Derivatives: Maxima and Minima}

One of the most important applications of derivatives is how we can use them to analyze functions. There is a lot of information we can get out of a derivative that will help us learn more about an arbitrary function and eventually we will put all of this information together to help graph a function. For now let us talk about maxima and minima. There are two kinds - absolute and local. 
\begin{Definition}
	Let $f$ be defined on a set $D$ containing $c$. If $f(c) \geq f(x)$ for every $x$ in $D$ then $f(c)$ is an absolute maximum value of $f$ on $D$. Similarly if $f(c) \leq f(x)$ for every $x$ in $D$, then $f(c)$ is an absolute minimum value of $f$ on $D$. An absolute extreme value is either an absolute max or an absolute min. These are sometimes called global max and min values.
\end{Definition}

Let us do a quick review of what open and closed intervals are. An open interval is an interval that does not contain either of its end points, while a closed interval is one that does. 
\begin{Example}
	Consider the function $y = 4 - x^2$. We can look at finding the absolute max and min on various intervals - $(-\infty, \infty), [0, 2], (0, 2], (0, 2)$. 
\end{Example}

If we want to know whether absolute max and min values exist, we need two things - 
\begin{Theorem}
	(Extreme Value Theorem or EVT) A function that is continuous on a closed interval $[a, b]$ has an absolute max value and an absolute min value on that interval. 
\end{Theorem}

Now let us look at local extrema - 
\begin{Definition}
	Suppose $c$ is an interior point of some interval $I$ on which $f$ is defined. If $f(c) \geq f(x)$ for all $x$ in $I$ then $f(c)$ is a local maximum. Similarly if $f(c) \leq f(x)$ for all $x$ in $I$ then $f(c)$ is a local minimum. Local extrema are also called relative extrema. 1
\end{Definition}

(draw a couple examples here, including one where absolute and local max/min are the same and one where absolute and local max/min are different)

While we have talked about these points we have not linked how they are related to derivatives - we will now do that. If we look at the examples above we notice that the local max and min occur at points where the derivative of the function is 0. This leads to the Local Extreme Value Theorem - 
\begin{Theorem}
	If $f$ has a local maximum or minimum value at $c$ and $f'(c)$ exists, then $f'(c) = 0$. 
\end{Theorem}

In general local extrema can occur at points $c$ where  $f'(c)$ does not exist, but most examples in this course will be where $f'(c)$ does exist. 

It is not however always true that if $f'(c) = 0$ then we have a local max or min. So we need to define another term to talk about these special points - 
\begin{Definition}
	An interior point $c$ of the domain of $f$ at which $f'(c) = 0$ or $f'(c)$ fails to exist is called a critical point of $f$. 
\end{Definition}

So let us look at an example of finding critical points and then determining whether we have a local max/min - 
\begin{Example}
	Consider $y = \frac{x}{x^2 + 1}$. 
\end{Example}

What about locating absolute maxima and minima? Normally on these types of problems we are given an interval to work on. The general procedure is in 3 steps - 
\begin{enumerate}
	\item[1.] Find the critical points of $f$ 
	\item[2.] Evaluate $f$ at the critical points and at the endpoints of the interval 
	\item[3.] Choose the largest and smallest values for the absolute max and min. 
\end{enumerate}

So let us look at an example - 
\begin{Example}
	Consider $f(x) = x^4 - 2x^3$ on the interval $[-2, 2]$. 
\end{Example}

A common word problem application of this is finding the maximum height of a trajectory. 
\begin{Example}
	A stone is launched vertically upward from a bridge 80 ft above the ground at a speed of 64 ft/s. Its height above the ground $t$ seconds after the launch is given by $f(t) = -16t^2 + 64t + 90$ for $0 \leq t \leq 5$. When does the stone reach its maximum height. (go through the 3 step process above)
\end{Example}


\subsection{Mean Value Theorem (MVT)} 

So the use of finding maxima and minima is to help us in graphing functions as well as solving optimization problems that we will eventually get to. To get there though, we need to talk about one of the most important theorems in Calculus - the Mean Value Theorem. 

\subsubsection{Rolle's Theorem} 

The MVT relies on a theorem known as Rolle's Theorem. Rolle's Theorem states the following 
\begin{Theorem}
	Consider a function $f$ that is continuous on a closed interval $[a, b]$ and differentiable on the open interval $(a, b)$. If $f(a) = f(b)$ then there is at least one point $c \in (a, b)$ such that $f'(c) = 0$. 
\end{Theorem}

The proof of this theorem is an application of the EVT that was mentioned before. 

An example of a problem dealing with Rolle's Theorem that you might see is -
\begin{Example}
	Find an interval $I$ on which Rolle's Theorem applies to $f(x) = x^3 - 7x^2 + 10x$. Then find all points $c$ in $I$ at which $f'(c) = 0$. 
	
	So to do this problem we just need an interval $[a, b]$ where $f(a) = f(b)$. The easiest way to find this is to find two points where $f(x) = 0$ and then take the interval between them. We can factor $f(x) = x(x - 2)(x - 5)$. We have a few options here but let us just take the interval $[0, 5]$. Now the second part of the problem is to find all points at which $f'$ is 0 and this just amounts to finding the critical points of $f$. 
\end{Example}

\subsubsection{Mean Value Theorem}

Now let us talk about the main result in this section - the MVT. The statement of the MVT is 
\begin{Theorem}
	If $f$ is continuous on the closed interval $[a, b]$ and differentiable on $(a, b)$, then there is at least one point $c$ in $(a, b)$ such that $$\frac{f(b) - f(a)}{b - a} = f'(c).$$
\end{Theorem}

So how do we use this theorem? Well here is one example - 
\begin{Example}
	Determine whether the function $f(x) = 2x^3 - 3x + 1$ satisfies the conditions of the MVT on the interval $[-2, 2]$. If so find the point(s) guaranteed to exist by the theorem. 
	
	So when it says "satisfies the conditions" we need to check if $f$ is continuous and differentiable on the interval $[-2, 2]$. It is because $f$ is a polynomial. So then on the interval the average rate of change is $\frac{f(2) - f(-2)}{2 - (-2)}  = 5$. So our goal is to find a point $x$ where $f'(x) = 5$. Solving for this gives two points $x = \pm 2/\sqrt{3}$. 
\end{Example}

\subsubsection{Consequences of the MVT} 

There are two important consequences of the MVT - 
\begin{Theorem}
	If $f$ is differentiable and $f'(x) = 0$ at all points on an open interval $I$, then $f$ is a constant function on $I$. 
\end{Theorem}

\begin{Theorem}
	If two functions have the property that $f'(x) = g'(x)$ for all $x$ on an open interval $I$, then $f(x) - g(x) = C$ on $I$ where $C$ is a constant, i.e. $f$ and $g$ differ by a constant. 
\end{Theorem}

\section{6/30}
\subsection{Graphing with Derivatives}
We are now going to get into talking about how we can use derivatives to help us graph functions. 
\subsubsection{Increasing vs Decreasing Functions} 

An important feature about functions we can talk about is whether they are increasing or decreasing. 
\begin{Definition}
By definition we have that a function $f$ is increasing on an interval $I$ if $f(x_2) > f(x_1)$ whenever $x_2 > x_1$. Similarly $f$ is decreasing on an interval $I$ if $f(x_2) < f(x_1)$ whenever $x_2 > x_1$. 
\end{Definition}

As a note - a function that is always increasing or decreasing is called monotonic. We can say a function is "monotone increasing" or "monotone decreasing". 

So how do we know when a function is increasing or decreasin? Well looking a graph of an increasing/decreasing function let us look at the slope of the tangent line. Note that when a function is increasing the tangent line slope is positive and when a function is decreasing the slope of the tangent line is decreasing. 

\begin{Theorem}
Test for Intervals of Increase and Decrease. Suppose $f$ is a continuous function on an interval $I$. If $f'(x) > 0$ for all points on $I$ then $f$ is increasing on $I$ and similarly if $f'(x) < 0$ for all points on $I$ then $f$ is decreasing on $I$. 
\end{Theorem}

Last class there was some talk about how we use the Mean Value Theorem. Well here is one place where we end up using it - to prove this statement. The MVT is a very powerful theorem in proving many important results in Calculus which is why it is covered in every Calc class. 

So let's talk about how we use the MVT here to prove this statement. Let $a$ and $b$ be any two distinct points in the interval $I$ with $b > a$. By the $MVT$ we know that there is some $c$, $a < c < b$ where $$\frac{f(b) - f(a)}{b - a} = f'(c)$$. We can rearrange this to get $$f(b) - f(a) = f'(c)(b - a)$$. Now we know that $b - a > 0$ as $b > a$. If $f'(c) > 0$ then $f(b) - f(a) > 0$ so for all $a, b$ in $I$ (since we picked our $a$ and $b$ at random) with $b > a$ we have that $f(b) > f(a)$, i.e. $f$ is increasing on $I$. Similarly if $f'(c) < 0$, $f(b) - f(a) < 0$ so $f$ is decreasing on $I$. This is just one example of how the MVT is used to help us prove important facts that we will continuously use in the future. 

\subsubsection{Intervals of Increase/Decrease and Local Max/Min}

We can use the fact we just looked at to help us identify points that are local minimums or local maximums. Looking at a picture of what these look at notice that around a local minimum we have that (from left to right) the slope of the tangent line goes from negative, to 0, to positive. While for a local minimum we have the slop of the tangent line goes from positive, 0, to negative. This leads to one of the first major tools we have in graphing a function - the First Derivative Test.
\begin{Theorem}
(First Derivative Test) Assume that $f$ is continuous on an interval that contains a critical point $c$ and $f$ is differentiable on an interval containing $c$ except perhaps at $c$ itself. 
	\begin{enumerate}
		\item[1.] If $f'$ changes sign from positive to negative as $x$ increases through $c$ then $f$ has a local maximum at $c$. 
		\item[2.] If $f'$ changes sign from negative to positive as $x$ increases through $c$ then $f$ has a local minimum at $c$. 
		\item[3.] If $f'$ does not change sign at $c$ then $f$ has no local extreme value at $c$. 
	\end{enumerate}
\end{Theorem}

So let us do an example of putting together finding intervals of increase/decrease with the first derivative test. 

\begin{Example}
	Consider the function $f(x) = 3x^4 - 4x^3 - 6x^2 + 12 x + 1$. 
	
	To do anything we need to first find the derivative. This is just $f'(x) = 12x^3 - 12x^2 - 12x + 12 = 12(x^3 - x^2 - x + 1) = 12(x - 1)(x^2 - 1) = 12(x - 1)^2 (x + 1)$. 
	As a note - from now on as we are always going to be looking a critical points of functions it will be very helpful for us to try and factor the derivative as much as we can immediately when we calculate it. Looking at the derivative we have above it is not hard to see that we will have two critical $x$-values: $x = 1$ and $x = -1$.
	To find intervals of increase and decrease we just need to plug in some numbers and see what the sign of our derivative is. I just using simple/small integers. For example we can use -2, 0 and 2. Plugging this into our derivative we get 
	\begin{align*}
		f'(-2) &= 12(-2 -1)^2 (-2+1) < 0 \\
		f'(0) &= 12(0 - 1)^2 (0 + 1) > 0 \\
		f'(2) &= 12(2 - 1)^2 (2 + 1) > 0
	\end{align*}
	
	So from this we see that we have an interval of decrease from $(-\infty, -1)$ and then an interval of increase from $(-1, 1)\cup (1, \infty) = (-1, \infty)$. Note that we never include any of the end points in our intervals of increase and decrease as either they are infinity or critical points which are points where $f'(x) = 0$ (neither increasing or decreasing) or $f'(x)$ does not exist.

	Now let us use the first derivative test to find where we have local extrema. The test says that when the derivative changes sign we have either a local max or min and when it does not change we have neither. Note that around -1, we have that the derivative changes sign from negative to positive - so by the first derivative test $f(-1)$ is a local minimum. Around 1 the derivative does not change sign, so that means that $f(1)$ is not a local max or local min.
\end{Example} 	

\subsubsection{Absolute Max/Min on Any Interval}
Last class we talked about Absolute Max/Min on a closed interval. The question remains whether we can extend talking about absolute max to any arbitrary interval. Well there is a result that deals with this - 
\begin{Theorem}
	Suppose $f$ is continuous on an interval $I$ that contains exactly one local extremum at $c$. 
	\begin{itemize}
		\item If a local minimum occurs at $c$, then $f(c)$ is the absolute minimum of $f$ on $I$. 
		\item If a local maximum occurs at $c$, then $f(c)$ is the absolute maximum of $f$ on $I$. 
	\end{itemize}
\end{Theorem}

We can see an example of this theorem at work when we look at absolute extreme values of a quadratic function on any open interval. 

\subsubsection{Concavity and Inflection Points}

So far we have talked about increasing and decreasing intervals using the first derivative. I might have mentioned in passing that the second derivative would be a useful tool when graphing a function. Well this is where we are going to talk about the second derivative and what it can tell us about a function. 

Let us talk first about concavity. A function can be concave up or concave down on an interval. Concave up is shaped like a U, while concave down is shaped like an upside down U. The point at which a function changes concavity (from up to down or vice versa) is called an inflection point. 

So how do we determine when a function is concave up or down? How do we determine where the inflection points are? To do so we use the second derivative. 
\begin{Theorem}
	Suppose $f''$ exists on an open interval $I$. 
	\begin{itemize}
		\item If $f'' > 0$ on $I$, then $f$ is concave up on $I$. 
		\item If $f'' < 0$ on $I$, then $f$ is concave down on $I$. 
		\item If $c$ is a point of $I$ at which $f''$ changes sign at $c$ then $f$ has an inflection point at $c$. 
	\end{itemize}
\end{Theorem}

Again similar to how we looked at critical points and saw that not all critical points are a local max or min, not all points where $f''(x) = 0$ are inflection points (ex. $f(x) = x^4$). 

\begin{Example}
Finding intervals of concavity is similar to how we found intervals of increase and decrease earlier. Consider $f(x) = 3x^4 - 4x^3 - 6x^2 + 12 x + 1$ again. We found that $f'(x) = 12(x + 1)(x - 1)^2$ previously. So now let us find the second derivative. By the product rule we have $f''(x) = 12 (x - 1)^2 + 24(x + 1)(x - 1) = 12(x - 1)[(x - 1) + 2(x + 1)] = 12(x - 1)(3x + 1)$. So we see that $f''(x) = 0$ at $x = 1$ and $x = -\frac{1}{3}$. We then check to see if the function is positive or negative around these points. So using nice integer values like $-1$, $0$ and $2$ we see that 
	\begin{align*}
		f''(-1) &= 12(- 1 - 1)(3(-1) + 1) > 0 \\
		f''(0) &= 12 (0 - 1)(0 + 1) < 0 \\
		f''(2) &= 12 (2 - 1) (3(2) + 1) > 0
	\end{align*}
	
	We can see from this that since $f''(x) > 0$ on the intervals $(-\infty, -\frac{1}{3})$ and $(1, \infty)$, $f$ is concave up on these intervals and that $f$ is concave down on the interval $(-\frac{1}{3}, 1)$ as $f''(x) < 0$ on this interval. 
\end{Example}

If you recall we found that $f(-1)$ is a local minimum of the function in the previous example. We just showed that $f''(-1) > 0$ so $f$ is concave up at $-1$. This leads to the second way one can look for local extreme values - the Second Derivative Test. 
\begin{Theorem} 
	Suppose $f''$ is continous on an open interval containing $c$ with $f'(c) = 0$. 
	\begin{itemize}
		\item If $f''(c) > 0$ then $f$ has a local minimum at $c$. 
		\item If $f''(c) < 0$ then $f$ has a local maximum at $c$. 
		\item If $f''(c) = 0$ then we cannot tell if it is a max or a min. 
	\end{itemize}
\end{Theorem}

\subsubsection{Review of Derivative Properties}

So let us put together everything we have found out so far. We have looked at two properties of functions - increasing/decreasing intervals and concavity. We use the first derivative to help up figure out when a function is increasing vs. decreasing. We then looked at the second derivative to tell us more about concavity. We also noted that a point where concavity changes is called an inflection point. 

Along the way using the nature of concavity or what it means for a function to switch from increasing to decreasing or vice versa we developed two tests to help us determine local max/min - the first derivative test and the second derivative test.

We can now put all of this together to help us graph functions using derivatives. 


\section{7/1}

\subsection{Graphing}
So last class we talked about using the first derivative to find intervals of increase and decrease and using the second derivative to find intervals of concavity. We have previously talked about finding vertical, horizontal and slant asymptotes. We will now put all of these together to help us graph a function. 

Let us start with a list of things that if we note down will help us sketch a graph of the function without needing to use a graphing calculator. 

\subsubsection{Procedure for Graphing}
\begin{itemize}
	\item[1.] First note the domain/interval of interest of the function. It is also useful at this stage to think about if your function has some sort of symmetry (even or odd function). 
	\item[2.] Find the vertical and horizontal asymptotes of a function if they exist. (end behaviour)
	\item[3.] Then find the first and second derivatives of the function. 
	\item[4.] Find the critical points and possible inflection points by solving $f'(x) = 0$ and $f''(x) = 0$. 
	\item[5.] Identify intervals of increase/decrease and concavity. 
	\item[6.] Using the previous step (test of your choice) find the local extreme values and inflection points if they exist. 
	\item[7.] Find $x$ and $y$ intercepts. 
	\item[8.] Then finally sketch the function. 
\end{itemize}

In reality all the previous steps can be done in any order. As long as you get all of that information it will not be difficult to sketch the graph of a function. 

There is no better way to see this than going through a few examples of how to do all of this. 
\subsubsection{Examples}
Let us start simple and make our way to harder functions. 
\begin{Example}
 Let us try to sketch $f(x) = x^3 -6x^2 + 9x$. We can factor this to find our $x$-intercepts, $f(x) = x(x^2 - 6x + 9) = x(x - 3)^2$ So this gives us our two $x$-intercepts $(0, 0)$ and $(0, 3)$. This also gives us our $y$-intercept. As $f$ is a polynomial we know that the end behaviour is that it goes to negative infinity as $x$ goes to negative infinity and $f$ goes to positive infinity as $x$ goes to positive infinity. 
 
 Now let us take the first derivative and look at critical points - $f'(x) = 3x^2 - 12x + 9 = 3 (x^2 - 4x + 3) = 3(x - 1)(x - 3)$. So here we have two critical points when $x = 1$ and $x = 3$. Looking for intervals of increase and decrease we have 
 \begin{align*}
 	f'(0) &= 3(- 1)(-3) > 0 \\
	f'(2) &= 3(2 - 1)(2 -3) = 3(1)(-1) < 0 \\
	f'(4) &= 3(4 - 1)(4 - 3) > 0.
\end{align*}

So the intervals of increase are $(-\infty, 1)$ and $(3, \infty)$ and the interval of decrease is $(1, 3)$. We can use this to find our local max/min  - By the first derivative test we have that there is a local max when $x = 1$ and a local min when $x = 3$. Finding the points gives us $(1, 4)$ and $(3, 0)$. 

So next moving on to concavity we take the second derivative - $f''(x) = 6x - 12 = 6(x - 2)$ and see that we have one possible point of inflection when $x = 2$. Notice that as this function is linear the sign will change, so $(2, 2)$ is an inflection point. Then looking to see where we have concave up vs down Look at $f''(0) < 0$ and $f''(3) > 0$ so $f$ is concave down on $(-\infty, 2)$ and concave up on $(2, \infty)$. 

Now we can put all of this together and sketch a graph of this function. 

\begin{tikzpicture}
	\draw[->] (-0.5, 0) -- (3.2, 0) node[right] {$x$};
	\draw[->] (0, -4) -- (0, 6) node[above] {$y$};
	\draw[scale=0.5, domain=-0.5:4.5, smooth, variable=\x, blue] plot ({\x}, {\x*\x*\x - 6*\x*\x + 9*\x});
\end{tikzpicture}
\end{Example}

Let us try and example where we can talk about symmetry - 
\begin{Example}
	Sketch $f(x) = x + \sin 2x$ on the interval $[-\pi, \pi]$. So let us talk about how symmetry helps us here. Note that $f(-x) = (-x) + \sin 2(-x) = -x + \sin (-2x) = -x - \sin 2x = - (x + \sin 2x) = - f(x)$. So $f$ is what is called an odd function. So using what we find out about $x$ on the positive axis we can sketch $f$ on the negative axis. So let us for simplicities sake just consider $f$ on the interval $[0, \pi]$. 
	
	So looking at this we have only one $x$ intercept and that's $(0, 0)$. 
	
	Then let us look at critical points - $f'(x) = 1 + 2 \cos 2x = 0$. So this is $0$ when $\cos 2x = - \frac{1}{2}$, which on the interval $[0, \pi]$ is only when $2x = \frac{2}{3}\pi$ and $2x = \frac{4}{3}\pi$, so when $x = \frac{\pi}{3}, \frac{2\pi}{3}$. Then looking at intervals of increase and decrease - 
	\begin{align*}
		f'(\pi/ 6) &= 1 + 2 \cos \frac{\pi}{3} > 0  \\
		f'(\pi/2) &= 1 + 2 \cos \pi < 0 \\
		f'(3\pi/4) &= 1 + 2 \cos \frac{3}{2}\pi > 0 
	\end{align*}
	
	So from this we can see that the intervals of increase are $[0, \frac{\pi}{3})$ and $(\frac{2\pi}{3}, \pi]$ and decreasing on $(\frac{\pi}{3}, \frac{2\pi}{3})$. 
	
	Moving on to the second derivative and concavity, $f''(x) = -4\sin 2x$. Then $f''(x) = 0$ only at$\frac{\pi}{2}$ in the interval $[0, \pi]$ (we do not look at the end points on of the interval for now). Note then that 
	\begin{align*}
		f''(\frac{\pi}{3})  &= -4 \sin \frac{2\pi}{3} < 0 \\	
		f''(\frac{2\pi}{3}) &= -4 \sin \frac{4\pi}{3} > 0
	\end{align*}
	
	So we see that on the interval $[0, \frac{\pi}{2})$ $f$ is concave down and then from $(\frac{\pi}{2}, \pi]$ $f$ is concave up. So $(\frac{\pi}{2}, \frac{\pi}{2} - 1)$ is an inflection point. By the second derivative test we know that $(\frac{\pi}{3}, \frac{\pi}{3} + \sqrt{3}{2})$ is a local max and $(\frac{2\pi}{3}, \frac{2\pi}{3} - \sqrt{3}{2})$ is a local minimum. 
	
	Using this we can attempt to sketch what our function looks like on the interval $[0, \pi]$. Then by the symmetry we have of $f$ where $f$ is an odd function we can rotate about the line $y = x$ to sketch $f$ on the interval $[- \pi, 0]$. Doing this gives us that there is another local minimum (this time) at $(-\frac{\pi}{3}, - \frac{\pi}{3} - \sqrt{3}{2})$ and a local maximum at $(-\frac{2\pi}{3}, -\frac{2\pi}{3} + \sqrt{3}{2})$. Intervals of increase at $[-\pi, -\frac{2\pi}{3})$ and $(-\frac{\pi}{3}, 0]$ and decrease on $(-\frac{2\pi}{3}, -\frac{\pi}{3})$. For concavity we have concave up on the interval $(\frac{-\pi}{2}, 0)$ and concave down on the interval $[-\pi, -\frac{\pi}{2})$. 
	
\begin{tikzpicture}
	\draw[->] (-3.2, 0) -- (3.2, 0) node[right] {$x$};
	\draw[->] (0, -3.2) -- (0, 3.2) node[above] {$y$};
	\draw[color=red, domain=-3.2:3.2, smooth] plot({\x}, {\x + sin(2*\x r)});
\end{tikzpicture}
\end{Example}

Let us look at a more complicated example - a rational function. 
\begin{Example}
	Consider the function $f(x) = \frac{x^3}{4- x^2}$. As we have a rational function the first thing we can do is find vertical and horizontal asymptotes. Solving the denominator for 0 gives us $x = \pm 2$ as our vertical asymptotes. Notice that we have an $x$ and $y$ intercept at $(0,0)$. We know that as the power of the numerator is exactly one larger than the denominator we have a slant asymptote so then using polynomial division we get that $f(x) = -x + \frac{4x}{4 - x^2}$. So we have a slant asymptote of $y = -x$. 

	Now let us look at the first derivative. 
	\begin{align*}
		f'(x) &= \frac{3x^2 (4 - x^2) - x^3(-2x)}{(4 - x^2)^2} \\ 
		      &= \frac{12x^2 - 3x^4 + 2x^4}{(4 - x^2)^2} \\
			  &= \frac{12x^2 - x^4}{(4 - x^2)^2} \\
			  &= \frac{x^2 (12 - x^2)}{(4 - x^2)^2}
	\end{align*}
	
	From this we can see that we have 5 critical points. We have critical points when $x = 0, \pm 2 \sqrt{3} = \sqrt{12}$ and $\pm 2$. Then looking at intervals of increase and decrease we want to find points on each of the intervals - 
	\begin{align*}
		f'(-4) &= \frac{16 (12 - 16)}{(4 - 25)^2} < 0 \\
		f'(-3) &= \frac{9 (12 - 9)}{(4 - 9)^2} > 0 \\
		f'(-1) &= \frac{1 (12 - 1)}{(4 - 1)^2} > 0 \\
		f'(1) &= \frac{1 (12 - 1)}{(4 - 1)^2} > 0 \\
		f'(3) &= \frac{9 (12 - 9)}{(4 - 9)^2} > 0 \\
		f'(4) &= \frac{16(12 - 16)}{(4 - 25)^2} < 0
	\end{align*}
	
	From this we see that we have intervals of increase on $(-2\sqrt{3}, -2), (-2, 0), (0, 2)$ and $(2, 2\sqrt{3})$ and intervals of increase $(-\infty, -2\sqrt{3})$ and $(2\sqrt{3}, \infty)$. So from this we see that we have a local max when $x = 2\sqrt{3}$ the point being $(2\sqrt{3}, -3\sqrt{3})$ and a local min when $x = -2\sqrt{3}$, the point being $(-2\sqrt{3}, 3\sqrt{3})$ 
	
	For concavity we need to take the second derivative. This is where dealing with certain rational functions gets really tedious and challenging. It is easy to make a mistake at this point when calculating the second derivative. I will do this by rewriting our $f'$ as $f'(x) = (12x^2 - x^4)(4 - x^2)^{-2}$. Then taking the derivative we get -
	\begin{align*}
		f''(x) &= (24x - 4x^3)(4 - x^2)^{-2} + (12x^2 - x^4)(-2)(4 - x^2)^{-3}(-2x) \\
			   &= \frac{24x - 4x^3}{(4 -x^2)^{2}} + \frac{4x(12x^2 - x^4)}{(4 - x^2)^3} \\
			   &= \frac{(24x - 4x^3)(4 - x^2) + 4x(12x^2 - x^4)}{(4 - x^2)^3} \\
			   &= \frac{96x - 16x^3 - 24x^3 + 4x^5 + 48x^3 - 4x^5}{(4 - x^2)^3} \\
			   &= \frac{96x - 8x^3}{(4 - x^2)^3} = \frac{8x (12 + x^2)}{(4 - x^2)^3}. 
	\end{align*}
	
	From this note that $x^2 + 12 > 0$ for all $x$ so that $f''(x) = 0$ only at $x = 0$. So this is a possible inflection point. HOWEVER, it is possible for concavity of a function to change on different sides of a vertical asymptote. So we need to look at where we have vertical asymptotes as well, which are at $x = \pm 2$. So looking for intervals of concavity we have - 
	\begin{align*}
		f''(-3) &= \frac{8(-3)(12 + 9)}{(4 - 9)^{3}} > 0. \\
		f''(-1) &= \frac{8(-1)(12 + 1)}{(4 - 1)^3 } < 0 \\
		f''(1) &= \frac{8(1) (12 + 1)}{(4 - 1)^3 } > 0 \\ 
		f''(3) &= \frac{8(3) (12 + 9)}{(4 - 9)^3} < 0
	\end{align*}
	
	So from this we see that $f$ is concave up on the intervals $(-\infty, 2)$ and $(0, 2)$ and concave down on the intervals $(-2, 0)$ and $(2, \infty)$. 
	
	We can now sketch what this function would look like. 
	
	\begin{tikzpicture}
		\begin{axis} [
		axis lines = center,
		xlabel = $x$,
		ylabel = {$y$},
		xmax = {6},
		xmin = {-6},
		ymax = {10},
		ymin = {-10},
		restrict y to domain = -10:10,
		]
		
		\addplot [
			domain=-6:6,
			samples=100,
			color=black,
		]
		{(x^3)/(4 - x^2)};
		
		%slant asymptote
		\addplot[dashed] {-x};
		%vertical asymptote at x = -2
		\draw[dashed] ({axis cs:-2, 0}|-{rel axis cs:0,0}) -- ({axis cs:-2, 0}|-{rel axis cs:0, 1});
		%vertical asymptote at x = 2
		\draw[dashed] ({axis cs:2, 0}|-{rel axis cs:0,0}) -- ({axis cs:2,0}|-{rel axis cs:0, 1});
		\end{axis}
	\end{tikzpicture}
\end{Example}

\section{7/2}

\subsection{Optimization}

So we are now going to talk about a type of application of derivative problem called optimization problems. The main goal of such problems is that we have a function that we want to optimize with respect to a set of constraints. 

As an example of how this works think about this question - if you are given two numbers that add to 20, what is the largest the product of those two numbers could be. This is a very simple example of an optimization problem. Our constraint is that the two numbers add to 20. We would write this as $x + y = 20$. The function we want to optimize is the product of the two, so $P = xy$. Now this is a function with two variables $x$ and $y$. We want to rewrite $P$ as a function of one variable, say $x$. The way we do this is through the constraint. Rewriting the constraint we get $y = 20 - x$. So then $P = x(20 - x) = 20x - x^2$. Now to maximize this (which is what the question asks to find) we then take the derivative and find the critical point as that will tell us where the local max/min of our function is. This case the derivative is simple - $P' = 20 - 2x$. Solving for the critical point we get $x = 10$. Now since we know that $P$ is a quadratic we have that $x = 10$ gives us a maximum, but normally we will have to use a derivative test to double check that we are getting a max or a min. So as $x = 10$ is the max, using our constraint we have that $y = 10$. So the largest product of two numbers that add to 20 is 100. 

\subsubsection{More Examples}
	Let us go through some examples. 
	\begin{Example}
		Suppose a rancher has 400 ft of fence for constructing a rectangular corral. One side is formed by a barn and requires no fence. Three exterior fences and two interior fences partition the corral into three rectangular regions. What dimensions of the corral maximize the enclosed area? What is the area of that corral? 
		
		So let us first create a sketch where $x$ is the width and $y$ is the length of the corral. The amount of fence required is $4x + y$ and as the rancher has 400 ft the constraint is $4x + y = 400$. The function we want to maximize is the area function of a rectangle which is $A = xy$. Again rewriting the constraint to solve for one variable, $y = 400 - 4x$. So then $A = xy = x(400 - 4x) = 400x - 4x^2$. Now we want to maximize this function. But note that as $x$ is a length, it is larger than 0 and has to be less than 100 due to the constraint. This gives us the domain we want to maximize $A$ on which is $0 \leq x \leq 100$. Take the derivative $A' = 400 - 8x$ and then solve for $x$ which gives $x = 50$. We know as this is again a quadratic that this is going to be a local max - so we have a maximum area when $x = 50$ and $y = 400 - 4(50) = 200$ which gives us a maximum area of $50*200 = 10000$. 
	\end{Example}
	
	\begin{Example}
		Suppose an airline policy states that all baggage must be box-shaped with a sum of length, width and height not exceeding 64 in. What are the dimensions and volume of a square-based box with the greatest volume under these condition? 
		
		So let us do a rough sketch. Since we are using a square based box our width and length are the same. So we have the sum of the length, width and height being $2w + h = 64$. This is our constraint. We want to maximize the volume function $V = w^2 h$. Using our constraint let's express $h$ in terms of $w$ - $h = 64 - 2w$. Then substituting gives $V = w^2 (64 - 2w) = 64w^2 - 2 w^3$. Our interval of interest for $w$ is $0 \leq w \leq 32$. As we want to maximize the volume we take the derivative of $V$ and get $V' = 128 w - 6w^2 = 2w(64 - 3w)$. Setting equal to 0 gives two critical points 0 and $\frac{64}{3}$. To check that we have a max let us use the second derivative test. $V'' = 128 - 12w$. Note that when we plug in $0$ $V''>0$, but when we plug in $\frac{64}{3}$, $V'' < 0$. So we have a local maximum at $\frac{64}{3}$. Hence, the dimensions and volume of a square-based box with the greatest volume is $\frac{64}{3} \times \frac{64}{3} \times \left(\frac{64}{3}\right)$ and the volume is $\left(\frac{64}{3}\right)^3$ cubic inches.  
	\end{Example}
	
\subsubsection{Procedure for Optimization Problems}
	\begin{enumerate}
		\item[1.] Read the problem, identify the variables and organize using a picture if possible. 
		\item[2.] Identify the objective function (function to be optimized) and write it using the variables you choose. 
		\item[3.] Identify the constraint function and express it using the variables you chose. 
		\item[4.] Use the constraint(s) to eliminate all but one independent variable of the objective function. 
		\item[5.] Find the interval of interest of the objective function. 
		\item[6.] Find the absolute max/min of the objective function on the interval of interest.
	\end{enumerate}
	
So now using this let us do some more examples. 
\subsubsection{Even More Examples}
	\begin{Example}
		A water storage tank in a small community is build in the shape of a right circular cylinder with a capacity of $32000 \textnormal{ ft}^3$. The interior wall and floor of the fank must be cleaned and treated annually. Labor costs for cleaning the wall are twice as high per square foot as the cost to clean the floor. Find the radius and height of the tank that minimize the cleaning cost. 
		
		So let us give some of the variables names while sketching a picture. Let $C$ be the cost of cleaning, $r$ the radius and $h$ the height. Then the objective function is the cost function which is given by $C = 2(2\pi rh) + \pi r^2$. 
		The constraint we are given is the capacity of the cylinder, which is the volume. So we have $V = \pi r^2 h = 32000$. It is probably easiest to solve for $h$ here so we have that $h = \frac{32000}{\pi r^2}$. We can plug this into our objective function. 
		Doing this gives us the new objective function $C(r) = \frac{128000}{r} + \pi r^2$. We have no constraint on the radius so our interval of interest is $0 < r < \infty$. So then taking the derivative we get 
		\begin{align*}
			C'(r) &= -\frac{128000}{r^2} + 2\pi r \\
				  &= \frac{2\pi r^3 - 128000}{r^2} = 0.
		\end{align*}
		Solving for $r$ gives us $r^3 = \frac{128000}{2 \pi} = \frac{64000}{\pi}$ or $r = \frac{40}{\sqrt[3]{\pi}}$. 
		We need to confirm that this is a minimum so let us take the second derivative - $C''(r) = \frac{256000}{r^3} + 2\pi$. Note that if we plug in $r$ that this will be positive so we do have a minimum. 
		
		So our radius that minimizes the cost is $r = \frac{40}{\sqrt[3]{\pi}}$ and the corresponding height is $h = \frac{32000}{\pi} r^{-2} = \frac{32000}{\pi} \cdot \frac{\sqrt[3]{\pi^2}}{1600} = \frac{20}{\sqrt[3]{\pi}}$. So this would be the dimensions of the tank that minimize the cost of cleaning.
	\end{Example}
	
	\begin{Example}
		An 8-foot tall fence runs parallel to the side of a house 3 feet away. What is the length of the shortest ladder that clears the fence and reaches the house? 
		
		So let us draw a picture. We have the 8 foot fence then the ladder going as the hypotenuse. The wall of the house creates a similar triangle. Let $L$ be the length of the ladder, $x$ be the distance from the ladder to the fence and $b$ be the height the ladder makes with the house.  The length of the ladder is given by Pythagorean theorem as $L^2. = (x + 3)^2 + b^2$. It is important to note that we can work $L^2$ as the max/min of $L^2$ is the same as the max/min of $L$. By similar triangles we have that $\frac{8}{x} = \frac{b}{3 + x}$. So expressing $b$ in terms of $x$ gives $b = \frac{8 (x + 3)}{x}$. Plugging this into our function for $L^2$ we have 
		\begin{align*}
			L^2 &= (x + 3)^2 + b^2 \\
				&= (x + 3)^2 + \left(\frac{8 (x + 3)}{x}\right)^2 \\
				&= (x + 3)^2 + \frac{64 (x + 3)^2}{x^2} \\
				&= (x + 3)^2 \left(1 + \frac{64}{x^2}\right).
		\end{align*}
		
		Since $L^2$ is nonnegative $L$ and $L^2$ have the same local extrema. So we can just take the derivative of $L^2$ - 
		\begin{align*}
			(L^2)' &= 2(x + 3)\left(1 + \frac{64}{x^2}\right) + (x + 3)^2 \left(-\frac{128}{x^3}\right) \\
				   &=2(x + 3) \left[ 1 + \frac{64}{x^2} - (x + 3)\frac{64}{x^3}\right] \\
				   &= \frac{2(x + 3)(x^3 - 192)}{x^3}.
		\end{align*}
		
		As $x > 0$ we know that $x+ 3 \neq 0$. Therefore we only have a 0 when $x^3 - 192 = 0$ which gives us $x = \sqrt[3]{192} = \sqrt[3]{64*3} = 4 \sqrt[3]{3}$. We can use the first derivative test to see that this is a local minimum. Plugging into our function for $L^2$ (using a calculator) gives a value of about 224.77 which means that $L$ is about 15ft. 
	\end{Example}	
	
	Another common example using the Pythagorean Theorem is finding the closest point on a function to a certain point. 
	\begin{Example}
		What point on the line $y = 3x + 4$ is closest to the origin? In this case our constraint is already given to us - $y = 3x + 4$. The function we want to optimize is the distance function $d = \sqrt{x^2 + y^2}$ (this is the distance of a point from the origin). Similar to the previous example we can just minimize the function under the square root, i.e. minimize $d^2 = x^2 + y^2$. We use our constraint to express $y$ in terms of $x$ and we get $d^2 = x^2 + (3x + 4)^2$. Then taking the derivative we have $(d^2)' = 2x + 2(3x + 4)(3) = 2x + 18x + 24 = 0$. Solving gives $x = -\frac{6}{5}$. By the first derivative test we know that this is a minimum for the function. So the point on the line $y = 3x + 4$ that is closest to the origin is $\left(-\frac{6}{5}, \frac{2}{5}\right)$. 
	\end{Example}
	
	Let's do another example - 
	\begin{Example}
		You want to see a certain number $n$ of items in order to maximize your profit. Market research tells you that if you set the price at \$1.50, you will be able to sell $5000$ items and for every 10 cents you lower the price below \$1.50 you will be able to sell another 1000 items. Suppose that your fixed costs total \$2000, and the per item cost of production is \$0.50. Find the price to set per item and number of items sold in order to maximize profit. Also dtermine the maximum profit you can get.Math 
		
		So the first thing we want to do is convert this into a function maximization problem. Since we want to maximize profit by setting the price per item, call our function $P(x)$ where $P$ is profit and $x$ is price per item. Profit is given by revenue minus costs and the revenue is the number of items sold times the price per item. This gives us that the function for profit is $$P(x) = nx - 2000 - 0.5n$$. The number of items sold is itself a function of $x$ - $$n = 5000 +\frac{1000 (1.5 - x)}{0.1}$$ Now we substitute for $n$ in the profit function and get 
		\begin{align*}
			P(x) &= nx - 2000 -0.5n \\
				 &= \left(5000 + \frac{1000(1.5 - x)}{0.10}\right)x - 2000 - 0.5\left(5000 + \frac{1000(1.5 - x)}{0.1}\right) \\
				 &= (5000 + 10000(1.5 - x))x - 2000 - 0.5(5000 + 10000(1.5 - x)) \\
				 &= 5000x + 15000x - 10000x^2 - 2000 - 2500 - 7500 + 5000x \\
				 &= -10000x^2 + 25000x - 12000.
		\end{align*}
		
		We want to know the maximum value of this function when $x$ is between 0 and 1.5. We take the derivative $P'(x) = -20000x + 25000$ and solving for $x$ when $P'(x) = 0$ gives us $x = 1.25$. Since this is going to be the local maxiumum we know that this is going to be the global maximum as well ($P$ is a quadratic with a negative leading term). So the maximum profit is \$3625 which is attained when we set the price to \$1.25, sell 7500 items. 
	\end{Example}
	
\section{7/7}

\subsection{Linear Approximation/Linearization}

We are now going to move on to another application of the derivative that is useful - linear approximation. Consider if we have a smooth curve and take a point $P$. If we draw the tangent line to this curve and zoom in, the curve and the tangent line will line up almost perfectly. In other words what we can do is approximate the curve using the tangent line for points near the point we take the tangent line at. 

We have been finding slope of tangent lines and equations of tangent lines quite often during this course so you should be familiar with the process.
\begin{Definition}
If we have a function $f$ that is differentiable on an interval containing a point $a$ then the equation of the tangent line at the point $(a, f(a))$ is given by $ y - f(a) = f'(a) (x - a)$ or $y = f(a) + f'(a) (x - a)$. We will call this tangent line the \textit{linear approximation} to $f$ at $a$ and denote it as $L(x)$. 
\end{Definition}

\begin{Example}
	To see how this works let us look at an example. Suppose you are driving along a highway at a constant speed. If your average speed is in mi/hr is given by $s(x) = \frac{3600}{60 + x}$ if you travel one mile in $x$ seconds more or less than 60 seconds. So for example if you travel one mile in 62 seconds then $x = 2$ and. your average speed is $s(2) = 58.06$ mi/hr. Similarly if you travel a mile in 57 seconds then $x = -3$ and $s(-3) = 63.16$ mi/hr. 
	
	Now this calculation requires a calculator is not exactly something you want to be doing while driving. So we will use a linear approximation at the point 0 (a mile a minute). So you compute the derivative and get $s'(x) = -3600(60 + x)^{-2}$ and then $s'(0) = -1$. $s(0) = 60$ so then using our formula from before we see that $s(x) \approx L(x) = s(0) + s'(0)(x - 0) = 60 - x$. This is a much easier calculation to do while driving. By our approximation if you travel one mile in 62 seconds ($x = 2)$ then $L(2) = 58$mi/hr gives your approximate speed. Similarly if you do a mile in 57 seconds then $s(-3) = 63$ mi/hr gives you approximate speed. Notice how our approximate values are close to the actual values given in our function
\end{Example}

Let us look at another example. 
\begin{Example}
	Find the linear approximation to $f(x) = \sqrt{x}$ at $x = 1$ and use it to approximate $\sqrt{1.1}$
	
	Again we use a linear approximation. But we need to pick around which point. To do so let us take a look at what we are approximating. Since we are approximating $\sqrt{1.1}$ let us pick $x = 1$. So then using our formula for lienar approximation we have that $L(x) = f(1) + f'(1) (x - 1) = 1 + \frac{1}{2} (x - 1) = \frac{1}{2} x + \frac{1}{2}$. 
	
	Now using this we can approximate $\sqrt{1.1} \approx L(1.1) = \frac{1}{2} (1.1 + 1) = 1.05$. The exact value is about $1.0488$ so our approximation is pretty good. 


	It is important to note if we are approximating some other value we will most likely have to change the point at which we are using a linear approximation at. For example if we wanted to do find $\sqrt{0.1}$ then it is not a great idea to use the approximation we just used - $L(0.1) = 0.55$ but $\sqrt{0.1}\approx 0.3162$. Instead we want to pick a point closes. to 0.1. We cannot use $x= 0$ as $f'(0)$ is undefined, so instead take $x = 0.09$. Then using a linear approximation we have that $\sqrt{0. 1} \approx L(0.1) = f(0.09) + f'(0.09)(0.1 - 0.09) = 0.3 + \frac{10}{6} \cdot \frac{1}{100} = \frac{19}{60}$, which is about 0.3167. A much better approximation. 
\end{Example}

Let us take a look at another example - 
\begin{Example}
Let us find the linearization of the function $f(x) = \sqrt{x + 3}$ at $a = 1$ and use it to approximate the nubmers $\sqrt{3.98}$ and $\sqrt{4.05}$. Again the process for finding the linear approximation is the same. First find the derivative - $f'(x) = \frac{1}{2} (x + 3)^{-1/2}$ and then find $f'(1) = \frac{1}{4}$ and $f(1) = 2$. Then our linear approximation is given by $$L(x) = f(1) - f'(1) (x - 1) = 2 - \frac{1}{4}(x - 1) = \frac{x}{4} + \frac{7}{4}.$$

	Now we want to use this to estimate our square roots. Note that since the function we have a linear approximation of is $\sqrt{x + 3}$ we need to find $x$ such that $\sqrt{x + 3} = \sqrt{3.98}$. In this case that is $x = 0.98$. So this is what we plug into our approximation - $L(0. 98) = \frac{0.98}{4} + \frac{7}{4} = 1.995$. Similarly we have that $\sqrt{4.05} \approx L(1.05) = \frac{7}{4} + \frac{1.05}{4} = 2.0125.$ So it is important to note that your $x$ is not always the value you are trying to estimate but depends on the function which you got your linear approximation from. 
\end{Example}

\subsubsection{Linear Approximation + Concavity}

So far we have just talked about linear approximation, but we can add concavity into the picture to get even more information about our approximation. 

Looking at these pictures we can see that when we have our function is concave up at the point we are approximating at that our estimate will be an underestimate. If our function is concave down then our estimate will be an overestimate. Furthermore based on what the value of $f''$ is at that point we can tell if our error will be large or small. The larger $|f''(a)|$ is the larger the error. 

Let us take a look at an example - 
\begin{Example}
	Consider the function $f(x) = \sqrt[3]{x}$. Let us find the linear approximations for the function at $x = 1$ and $x = 27$. First finding $f'(x)$ we get $f'(x) = \frac{1}{3 x^{2/3}}$ So then $f(1) = 1, f(27) = 3$, $f'(1) = \frac{1}{3}$ and $f'(27) = \frac{1}{27}$. Using this we can now calculate the linear approximation as 
	\begin{align*}
		L_1 (x) &= 1 + \frac{1}{3} (x - 1) = \frac{1}{3} x + \frac{2}{3} \\
		L_{2} (x) &= 3 + \frac{1}{27}(x - 27) = \frac{1}{27} x + 2
	\end{align*}
	
	We can then estimate $\sqrt[3]{2}$ and $\sqrt[3]{26}$ as $$\sqrt[3]{2} \approx L_1 (2) = \frac{1}{3}\cdot 2 + \frac{2}{3} = \frac{4}{3}$$ and $\sqrt[3]{26}$ as $$\sqrt[3]{26} \approx L_2 (26) = \frac{1}{27} \cdot 26 + 2 \approx 2.963.$$
	
	If we were to take the second derivative to look at concavity we have $f''(x) = -\frac{2}{9} x^{-5/3}$. Since for both the points we took approximations at the second derivative is negative, we have that our estimates are overestimates. Furthermore since $|f''(1)|\approx 0.22$ and $|f''(27)|\approx 0.00091$ we know that our estimate of $\sqrt[3]{26}$ is more accurate than our estimate of $\sqrt[3]{2}$. 
	
	Note that talking about error of estimates requires a calculator many times so will not be asked on a test. However you might be asked to say whether your estimate is an overestimate or underestimate. 
\end{Example}

\subsubsection{Linear Approximation + Change}

We can also use this idea of linear approximation to estimate the change in a function over a period. Our formula for a linear approximation is $$f(x) \approx f(a) + f'(a)(x - a).$$ If we were to rearrange this we would get $$f(x) - f(a) \approx f'(a) (x - a).$$ We can denotet the left hand side as $\Delta y$ (change in $y$) and $(x - a)$ as $\Delta x$ (change in $x$). Now we have an apprixmation formula for approximating change $\Delta y \approx f'(a) \Delta x$. Let us see how this works. 
\begin{Example}
	Let us approximate the change in $y = f(x) = x^9 - 2x + 1$ when $x$ changes from $1.00$ to $1.05$. So for the $a$ we use $a = 1$. Then calculating $f'(x) = 9x^8 - 2$ we see that $f'(1) = 7$. So then by our approximating change formula we have that the change in $y$ when $x$ changes from $1.00$ to $1.05$ is $\Delta y \approx f'(1) \Delta x = 7*0.05 = 0.35$. 
\end{Example}

Another example given as a word problem would be  - 
\begin{Example}
	Approximate the change in surface area of a spherical hot-air baloon when the radius decreases from $4$m to $3.9$m. 
	
	In this case we need a formula for surface area of a sphere. That formula is $S(r) = 4 \pi r^2$. Taking the derivative we have $S'(r) = 8 \pi r$. We want to approximate near $r = 4$ so we find $S'(4) = 32 \pi$. Then using our approximation formula we have $\Delta y \approx S'(4) \Delta x = 32 \pi (-0.1) = 3.2\pi \approx -10.05$. So the change in surface area is approximately -10.05 $\textnormal{m}^2$. Or you can say the surface area will decrease by approximately 10.05 $\textnormal{m}^2$. 
\end{Example}

\subsection{Differentials}

The previous work we did with using linear approximation to approximate a change in $y$ using linear approximation. What if we want to talk about a change in our linear approximation over a range of $x$ values? To do this we have a very specific type of variables and that is differentials. We give these variables a very specific notation as well - $dy$ is the change in the linear approximation and $dx$ is the change in $x$. The formula for differentials is in fact the same as what we were working with previously - $dy = f'(x) dx$. 

\begin{Definition}
To give a formal definition we have - Let $f$ be differentiable on an interval containing $x$. A small change in $x$ is denoted by the differential $dx$. The corresponding change in $f$ is approximated by the differential $dy = f'(x) dx$. That is $\Delta y = f(x + dx) - f(x) \approx dy = f'(x) dx$. 
\end{Definition}

The use of differentials is again almost the same as with approximating change in $y$ we did before. 
\begin{Example}
	Consider $f(x) = 3 \cos^2 x$. Let us write a formula for estimating small change $dx$. We first find the derivative  - $f'(x) = -6 \cos x \sin x = -3 \sin 2x$. So then our formula for estimating small change in $f(x)$ given small change in $dx$ is $dy = -3 \sin 2x \; dx$. For example if $x$ increases from $\frac{\pi}{4}$ to $\frac{\pi}{4} + 0.1$ then $dy = -3 \sin \frac{\pi}{2} * 0.1 = -0.3$. So there is an approximate change in the function of -0.3. 
\end{Example}

Much like before, we can also have word problems - 
\begin{Example}
	Use differentials to estimate teh amount of paint needed to apply a coat of paint 10 cm thick to a hemispherical dome with diameter 100 m. Recall that the volume of a sphere is given by $V = \frac{4}{3} \pi r^3$. 
	
	Now let us first figure out what our $dr$ is(Why $dr$?). If we are applying a coat of pain 10cm thick to the dome we are adding 10 cm to the radius of the dome. As the diameter of the dome is 100m the radius is 50 m. As we are adding 10cm our $dr = 0.1$ (Why?).
	Next let us take the derivative, we have $V'(r) = 4 \pi r^2$. So then with a radius of 50 we have $V'(50) = 4 \pi (50)^2 = 4 \pi (2500) = 10000\pi$. 
	Finally using our formula for differentials we have $dV = V'(r) dr = (10000\pi)(0.1) = 1000\pi \textnormal{ m}^3$. So the estimated amount of paint needed is $1000\pi \textnormal{ m}^3$. 
\end{Example}
\subsection{L'H\^{o}pital's Rule}

Let us go back to talking about limits. We have previously discussed a few different ways to evaluate limits that are "indeterminate forms". An indeterminate form is a limit that cannot be solved by direct substitution. The two methods we talked about were factoring and canceling, and multiplying the conjugate. 

However there are some limits that we cannot do either to get a result. For example we have the following limit $$\lim_{x \to 0} \frac{\sin x}{x}.$$ This was one of our two special trig limits from when we were talking about trig derivatives, where I said that this limit is equal to 1. If we plug in $x = 0$ into this we get $0/0$. This is one type of indeterminate form.

The other time of indeterminate form is illustrated by the limit $\lim_{x \to \infty} \frac{a x}{x + 1}$. In this case we have $\infty/\infty$ if we "plug" in $\infty$ for $x$. 

To deal with both of these cases we can use a rule called L' H\^{o}pital's Rule. 

\subsubsection{Indeterminate Form 0/0}

\begin{Theorem}
	Suppose $f$ and $g$ are differentiable on an open interval $I$ containing $a$ with $g'(x) \neq 0$ on $I$ when $x \neq a$. If $\lim_{x \to a }f(x) = \lim_{x \to a} g(x) = 0$, then 
	\begin{align*}
		\lim_{x\to a} \frac{f(x)}{g(x)} = \lim_{ x\to a} \frac{f'(x)}{g'(x)}
	\end{align*}
	provided the limit on the right exists. We can replace the limit with $x\to \pm \infty, x\to a^+$ or $x \to a^-$. 
\end{Theorem}

Let us return to our trig integral example at the start to see how this works. 
\begin{Example}
	Consider $\lim_{x \to 0} \frac{\sin x}{x}$. We see that this is of the form $0/0$. As both $\sin x$ and $x$ are differentiable everywhere and the derivative of $x$ is not 0 everywhere we can use l'H\^{o}pital's Rule - 
	\begin{align*}
		\lim_{x\to 0} \frac{\sin x}{x} &= \lim_{x\to 0} \frac{\cos x}{1} \\
					&= \cos 0 = 1
	\end{align*}
\end{Example}

Let us look at another eaxmple 
\begin{Example}
	Evaluate $\lim_{x \to 2} \frac{x^3 - 3x^2 + 4}{x^4 - 4x^3 + 7x^2 - 12x + 12}$. First to check that this is an indeterminate form we plug in 2 for $x$ and see that we get $0/0$. As we are dealing with polynomials we have no issue with differentiability so all we need to do is use l'H\^{o}pital's Rule. 
	\begin{align*}
		\lim_{x\to 2} \frac{x^3 - 3x^2 + 4}{x^4 - 4x^3 + 7x^2 - 12 x+ 12} &= \lim_{x \to 2} \frac{3x^2 - 6x}{4x^3 - 12x^2 + 14x - 12} \\
		&= \lim_{x \to 2} \frac{6x - 6}{12x^2 - 24x + 14} \\
		&= \frac{6(2) - 6}{12(4) - 24(2) + 14} = \frac{6}{14} = \frac{3}{7}.
	\end{align*}
	
	Note that we had to use l'H\^{o}pital's Rule twice because after we used it once we still had an indeterminate form of $0/0$. There might be other problems where you might have to use the rule more than once so do not despair if we you end up with an indeterminate form after using the rule once.
\end{Example}

\subsubsection{Indeterminate Form $\infty/\infty$}

Now let us consider the other indeterminate form $\infty/\infty$. In this case our application of L'H\^{o}pital's Rule is basically the same as before. 
\begin{Theorem}
	Suppose $f$ and $g$ are differentiable on an open interval $I$ containing $a$ with $g'(x) \neq 0$ on $I$ when $ x \neq a$. If $\lim_{x\to a} f(x) = \pm \infty$ and $\lim_{x\to a} g(x) = \pm \infty$ then 
	\begin{align*}
		\lim_{x \to a} \frac{f(x)}{g(x)} = \lim_{x \to a} \frac{f'(x)}{g'(x)}
	\end{align*}
	provided the limit on the right exists.
\end{Theorem}

Let us look at an example - 
\begin{Example}
	Evaluate $\lim_{x \to \infty} \frac{4x^3 - 6x^2 + 1}{2x^3 - 10 x + 3}$. Note that we have $\infty/\infty$ here so we can use l'H\^{o}pital's Rule. 
	
	\begin{align*}
		\lim_{x \to \infty} \frac{4x^3 - 6x^2 + 1}{2x^3 - 10x + 3} &= \lim_{x \to \infty}\frac{12x^2 - 12x}{6x^2 - 10} \\
			&= \lim_{x \to \infty} \frac{24x - 12}{12x} \\
			&= \lim_{x \to \infty} \frac{24}{12} = 2.
	\end{align*}
	
	Notice that we had to keep using l'Hopital's rule on this problem because we kept getting indeterminate forms.
\end{Example}

\subsubsection{Related Forms $0 \cdot \infty$ and $\infty - \infty$}

There are a few other forms that you can run into where we cannot directly apply L' Hopital's Rule but if we do some algebra we can. 
\begin{Example}
 Evaluate $\lim_{x \to \infty} x^2 \sin\left(\frac{1}{4x^2}\right)$. Notice here that we have an indeterminate limit of the form $\infty \cdot 0$. So we cannot immediately use l'Hopital's Rule. So normally what we do is rewrite this in a form where can use l'Hopital's Rule. To do this recall that $x^2 = \frac{1}{1/x^2}$. So then we get 
\begin{align*}
	\lim_{x \to \infty} x^2 \sin(\frac{1}{4x^2}) &= \lim_{x \to \infty} \frac{\sin (1/4x^2)}{1/x^2} \\
	&= \lim_{x \to \infty} \frac{\cos(1/4x^2) \cdot (1/4)(-2x^{-3})}{-2 x^{-3}} \\ 
	&= \lim_{x \to \infty} \frac{1}{4} \cos\left(\frac{1}{4x^2}\right) \\
	&= \frac{1}{4}.
\end{align*}

Notice that we rewrite the limit into the form 0/0 and used l'Hopital's Rule. 
\end{Example}

The other related form is $\infty - \infty$. 
\begin{Example}
	Evaluate $\lim_{x \to \infty} (x - \sqrt{x^2 - 3x})$. Note that this limit is of the form $\infty - \infty$. Again we want to rewrite this in a way we can use l'Hopital's Rule. 
	
	\begin{align*}
		\lim_{x \to \infty} x - \sqrt{x^2 - 3x} &= \lim_{x \to \infty}(x - \sqrt{x^2(1 - 3/x)}) \\
		&= \lim_{x \to \infty}(x - x\sqrt{1 - 3/x}) \\
		&= \lim_{x \to \infty}[x (1 - \sqrt{1 - 3/x})] \\
		&= \lim_{x \to \infty} \frac{1 - \sqrt{1 - 3/x}}{1/x} \\
		&= \lim_{x \to \infty} \frac{-\frac{1}{2} (1 - 3/x)^{-1/2}\cdot (3/x^2)}{-1/x^2} \\
		&= \lim_{x \to \infty} \frac{3}{2} (1 - 3/x)^{-1/2} = \frac{3}{2}
	\end{align*}
\end{Example}

\subsubsection{Things to be Careful About}
So we have gone over the most common cases where you will need to use L' Hopital's Rule. However there are a few key things to always remember when you are trying to use the rule/ when you should not use the rule. 
\begin{enumerate}
		\item L'Hopital's Rule says $\lim_{x \to a} \frac{f(x)}{g(x)} = \lim_{x \to a} \frac{f'(x)}{g'(x)}$. This means you should not be using the quotient rule. You just need to find $f'$ and $g'$ and evaluate the limit of $f'/g'$. 
		\item You need your limit to be in the indeterminate form of $0/0$ or $\infty/\infty$ before applying l'Hopital's rule. 
		\item If you are going to need to use l'Hopital's Rule more than once, make sure to simplify after you have used it 
		\item You can get trapped in an unending cycle of using l'Hopital's Rule and if you do than you must use a different method of evaluating the limit. 
		\item Be sure that the limit produced by l'Hopital's Rule exists. If a limit does not exist after using the rule that does not mean the original limit does not exist. 
\end{enumerate}
\section{7/9}
\subsection{Antiderivatives} 
We are now going to move on from taking derivatives to try and do the opposite - finding antiderivatives. Lust us start with the definition -
\begin{Definition}
	A function $F$ is an antiderivative of $f$ on an interval $I$ provided $F'(x) = f(x)$ for all $x$ in $I$. 
\end{Definition}

It is important to note that an function can have an an infinite number of antiderivatives by just adding a constant. For example an antiderivative of $x^2$ is $\frac{1}{3} x^3$ but so is $\frac{1}{3} x^3 - 200000000$. Due to this we normally talk about the family of antiderivatives and at a $+C$ whenever we find the antiderivative. So using the example we were talking with before we say that the family of antiderivatives of $x^2$ is $\frac{1}{3} x^3 + C$. 

\subsection{Indefinite Integrals} 

We have a special notation for taking derivatives using either the prime notation ($f'(x)$) or the $\frac{d}{dx}$ notation. We have a special notation for finding the antiderivative of a function $f(x)$ and that is the indefinite integral - $\int f(x) \; dx$.  Using this notation the example from above is $\int x^2 \; dx = \frac{1}{3} x^3 + C$.  

Now let us look at finding rules for integrals like what we found for taking derivatives. The most important one is the power rule - 
\begin{Theorem}
	$$\int x^p \; dx = \frac{1}{p + 1} x^{p  +  1} + C $$
	where $p \neq -1$
\end{Theorem}

We will be using this rule quite often much like how often we use the power rule for taking derivatives. HOWEVER it is important to note that the power rule for indefinite integrals above does not work when $p = - 1$. This is an integral we will not be dealing with in this course. 

There are two other rules that we will be using - the constant multiple rule and sum rule
\begin{Theorem}
	\begin{align*}
		\int c f(x) \; dx &= c \int f(x) \; dx \\
		\int f(x) + g(x) \; dx &= \int f(x) \; dx + \int g(x) \; dx
	\end{align*}
\end{Theorem}

\subsection{Simple Trig Integrals}

Just like we had rules based on the power rule, we can find indefinite integrals of trig functions that we know the derivatives to 
\begin{Theorem}
	\begin{align*}
		\int \sin x \; dx &= - \cos x + C \\
		\int \cos x \; dx &= \sin x + C \\
		\int \sec^2 x \; dx &= \tan x + C \\
		\int \sec x \tan x \; dx &= \sec x + C \\
		\int \csc x \cot x \; dx &= - \csc x + C \\
		\int \csc^2 x \; dx &= -\cot x + C .
	\end{align*}
\end{Theorem}

These are the six main integral we will work with for now. 

\subsection{Introduction to Differential Equations} 

A differential equation is an equation involving derivatives of an unknown function. For now we will be only working with simple initial value problems that look like - 
\begin{Example} 
	Suppose $f'(x) = 2x + 4$ and $f(0) = 34$ find $f(x)$. 
	
	To find $f$ we just integrate our derivative - $\int 2x + 4 \; dx = x^2 + 4x + C$. Then we use our initial value to find $C$ - $0^2 + 4(0) + C = 34$ so $C = 34$. This gives us that the the function $f$ that satisfies our differential equation is $f(x) = x^2 + 4x + 34$. 
\end{Example} 

\section{7/14} 
\subsection{Motion Problems (cont.)} 

Recall that when we were talking about derivatives we had some examples where we were looking at motion problems. We were given a position function and had to find/work with either velocity or acceleration. When we did this we ended up with the following relationship - velocity is the derivative of the position function and acceleration is the derivative of the velocity function. 

Now that we are talking about antiderivatives we can also work in the other direction. Velocity is the antiderivative of acceleration and the position function is the antiderivative of the velocity. Using this we can now answer a different kind of problem then before - 
\begin{Example} 
	Runner A begins at the point $s(0) = 0$ and runs on a straight and level road with velocity $v(t) = 2t$. Runner $B$ begins with a head start at the point $S(0) = 8$ and runs with a velocity $V(t) = 2$. Find the positions of the runners for $t \geq 0$ and determine who is ahead at $t = 6$. 
	
	The question is asking us to find $s(t)$ and $S(t)$ and then compare $s(6)$ with $S(6)$. So let us first find $s(t)$. Since the position function is the antiderivative of velocity we have the following initial value problem - $s'(t) = 2t, s(0) = 0$. Integrating we have $s(t) = t^2 + C$ and using $s(0) = 0$ we know that $s(t) = t^2$. 
	
	Similarly we have $S'(t) = 2, S(0) = 8$. Integrating we have $S(t) = 2t + C$ and using $S(0) = 8$ we get $S(t) = 2t + 8$. 
	
	So then $s(6) = 36$ and $S(6)  = 12 + 8 = 20$. So this means that runner A is ahead of runner B after 6 time units. 
\end{Example}

Another common thing to look at are initial value problems dealing with acceleration due to gravity - 
\begin{Example}
	Suppose a stone is thrown vertically upward at $t = 0$ with a velocity of $40$ m/s from the edge of a cliff that is 100 m above a river. 
	\begin{enumerate}
		\item[a.] Find the velocity of the object for $t \geq 0$. 
		
		This is just asking us to solve an initial value problem. We know that acceleration due to gravity is $a(t) = -9.8$. This is negative as gravity is a downward force. So we want to solve $v'(t) = -9.8, v(0) = 40$. Integrating and solving for $C$ gives us $v(t) = -9.8t + 40$. 
		
		\item[b.] Find the position of the object for $t \geq 0$. 
		
		Again this is another initial value problem where we have $s'(t) = v(t) = -9.8t + 40$ $s(0) = 100$. Integrating and solving gives $s(t)= -4.9 t^2 + 40t + 100$. 
		\item[c.] Find the maximum height of the object above the river. 
		
		We this problem just wants to find the max of $s$. We know that $s'(t) = -9.8t + 40$ so just solving for when $t$ is 0 should give us the critical point. Solving gives us $t = \frac{40}{9.8} = 40 \cdot \frac{10}{98} = \frac{200}{49} \approx 4.1$. So the height of the ball is a max at roughly 4.1 seconds after launch and the height is $s(\frac{200}{49}) \approx 182$. 
		
		\item[d.] With what speed does the object strike the river? 
		
		To do this we need to first find the time the object hits the ground. By using the quadratic formula we have that $$ t = \frac{-40 \pm \sqrt{1600 + 4(100)(9.8)}}{-9.8}$$ which after simplifying gives $t \approx -2.0$ and $t \approx 10.2$. As we can only use positive values of $t$ we know that we need to look at the speed when $t = 10.2$ and that is $|v(10.2)| \approx 60$. So the speed that the object strikes the river with is 60 m/s. 
	\end{enumerate}
\end{Example}

\subsection{Approximating Areas Under Curves}

When we started talking about derivatives we discussed that a derivative represented the slope of the tangent line and how it represented instantaneous velocity. Similarly we can talk about a geometric/real world representation of the integral - area under a curve/displacement. 

\subsubsection{Area Under a Velocity Curve} 
If we consider a simple velocity function $v = 60$ with a simple scenario of a car traveling at a constant velocity of 60 mi/hr along a straight highway for two hours and then consider the displacement of the car. This is just the distance traveled - which is 120 mi. If we look at a graph of the situation we see that the displacement can be formed by taking the area under the curve, i.e. the rectangle formed. 

If we want to consider a slightly more complicated example let us consider the velocity function $v(t) = t^2$ where $0 \leq t \leq 8$. We can then try to approximate the displacement in various ways - 
\begin{enumerate}
	\item[a.]We can divide the interval $[0, 8]$ into two subintervals $[0, 4]$ and $[4, 8]$ We can then see that if we do this we can approximate the area under the curve using two rectangles (using the midpoints) and we get that the displacement is approximately $4*4 + 36*4 = 160$. However note that there is this large gap that we haven't considered. 
	\item[b.] To get a better estimate we increase the number of subintervals/rectangles we can use. If we use 4 rectangles our estimate becomes 168. 
	\item[c.] If we go to 8 rectangles our estimate becomes 170. As we can see the estimate keeps getting better and better as we keep adding rectangles. 
\end{enumerate}

Much like what we did with slope and using limits, this is another place where we will incorporate limits. 

\subsubsection{Approximating Areas By Riemann Sums} 

We can also talk about approximating areas under curves that are not related to velocity. To do this we will construct something called a Riemann sum. First let us take a continuous function $f(x)$ on an interval $[a, b]$. We will then divide the interval into $n$ subintervals each with a length $\Delta x = \frac{b - a}{n}$. This is what we call a regular partition. 
\begin{Definition}
	Suppose $[a, b]$ is a closed interval containing $n$ subintervals $[x_0, x_1], \ldots [x_{n - 1}, x_n]$ of equal length $\Delta x = (b - a)/n$ with $a = x_0$ and $b = x_n$. The endpoints $x_0, x_1, \ldots x_n$ create a regular partition of $[a, b]$ and the $k$th grid point $x_k = a + k\Delta x$ for $k = 0, 1, \ldots, n$. 
\end{Definition}

Now using this partition if we create a rectangle on each subinterval and take the value of $f$ on a point $x_{k}^{*} $ on each subinterval we get that the area of the rectangle is $f(x_{k}^{*}) \Delta x$. If we then add up the areas of all of these rectangles we get what is called a Riemann sum. 

\begin{Definition}
	Suppose $f$ is defined on a closed interval $[a, b]$ which is divided into $n$ subintervals of equal length $\Delta x$. If $x_{k}^{*}$ is any point in the $k$the subinterval $[x_{k -1}, x_k]$ for $k = 1, 2, \ldots, n$. Then $$f(x_{1}^{*}) \Delta x + f(x_{2}^{*}) \Delta x + \cdots + f(x_{n}^{*})\Delta x$$ is called a Riemann sum for $f$ on $[a, b]$. This sum is called a \textbf{left Riemann sum} if $x_{k}^*$ is the left endpoint of $[x_{k - 1}, x_k]$. The sum is called a \textbf{right Riemann sum} if $x_{k}^*$ is the right endpoint of $[x_{k - 1}, x_k]$. The sum is called a \textbf{midpoint Riemann sum} if $x_{k}^*$ is the midpoint of $[x_{k - 1}, x_k]$. 
\end{Definition}

\begin{Example}
	If we want to look at an example of calculating these sums, let us consider the function $f(x) = x^2$ on the interval $[0, 8]$ with $n = 8$. If we divide this into $8$ subintervals, each interval will be length of 1. We can then calculate the three types of Riemann sums we have. 
	\begin{enumerate}
		\item[a.] For a left Riemann sum, we take the left endpoint of each interval and get $$(0) + 1 + 4 + 9 + 16 + 25 + 36 + 49 = 140$$ 
		\item[b.] For a right Riemann sum, we take the right end point of each interval and get $$ 1 + 4 + 9 + 16 + 25 + 36 + 49 + 64 = 204$$. 
		\item[c.] The midpoint sum is the most tedious the calculate as it will use midpoints but doing so we get $$\frac{1}{4} + \frac{9}{4} + \frac{25}{4} + \frac{49}{4} + \frac{81}{4} + \frac{121}{4} + \frac{169}{4} + \frac{225}{4} = \frac{680}{4} = 170$$
	\end{enumerate}

	The actual answer is the area under the curve is approximately 170.67. As we can see left Riemann sums are underestimates, right Riemann sums are overestimates and midpoint sum is an estimate in the middle. 
\end{Example}

\subsubsection{Sigma (Summation) Notation} 

Now if we want to get really accurate estimates we need to keep adding rectangles. Taking the sum of a really large number of rectangles would become really long, so we want to find a way to simplify writing out a large sum. That is where we bring in summation notation. We use the Greek letter $\Sigma$ to represent a sum. For example to express $1 + 2 + 3 + \cdots + 10$ we write this is $\sum_{k = 1}^{10} k $. The symbol $k$ is what we call an index. At the bottom of the sigma we put the lower limit and at the top we put the upper limit. 

So looking at some other examples we have 
\begin{Example}
	\begin{enumerate}
		\item[a.] $\sum_{k = 1}^{99} k = 1 + 2 + \cdots + 99 = 4950$ 
		\item[b.] $\sum_{ k = 1}^{4} k^2 = 1^2 + 2^2 + 3^2 + 4^2 = 30$ 
	\end{enumerate}
\end{Example}

As a note the index in a sum is called a dummy variable. You can choose any letter you want for it. The most common choices are $i, j, k, n, m$. 

Let us now talk about some important properties of sums - 
\begin{Theorem}
	\begin{enumerate}
		\item Constant Multiple Rule $$\sum_{k = 1}^{n} c a_k = c \sum_{k = 1}^{n} a_k$$ 
		\item Addition Rule $$\sum_{k = 1}^{n} (a_k + b_k) = \sum_{k = 1}^{n} a_k + \sum_{k = 1}^{n} b_k$$
	\end{enumerate}
\end{Theorem}

There are also a few important sum formulas that we will use, this are formulas for the sums of powers of integers -
\begin{Theorem}
	\begin{align*}
		\sum_{k = 1}^{n} c &= cn \\
		\sum_{k = 1}^{n} k &= \frac{n(n + 1)}{2} \\
		\sum_{k = 1}^{n} k^2 &= \frac{n(n + 1)(2n + 1)}{6} \\
		\sum_{k = 1}^{n} k^3 &= \frac{n^2 (n + 1)^2}{4}
	\end{align*}
\end{Theorem}

\subsubsection{Riemann Sums Using Sigma Notation} 

Going back to how we defined a Riemann sum before we can rewrite it using Sigma notation - 
\begin{Definition}
	(Riemann Sum v2) 	Suppose $f$ is defined on a closed interval $[a, b]$ which is divided into $n$ subintervals of equal length $\Delta x$. If $x_{k}^{*}$ is any point in the $k$the subinterval $[x_{k -1}, x_k]$ for $k = 1, 2, \ldots, n$. Then $\sum_{k = 1}^{n} f(x_{k}^{*}) \Delta x$ is called the Riemann sum of $f$ on $[a, b]$. For an easy way to calculate the $x_{k}^{*}$ for the sums we will be dealing with we have 
	\begin{enumerate}
		\item (left Riemann sum) $x_{k}^* = a + (k - 1) \Delta x$ 
		\item (right Riemann sum) $x_{k}^* = a + k\Delta x$ 
		\item (midpoint Riemann sum) $x_{k}^{*} = a + (k - \frac{1}{2})\Delta x$
	\end{enumerate}
\end{Definition}

To see this in action let us go back again to our example of $f(x) = x^2$ on the interval $[0, 8]$ but this time let us instead take 50 sub intervals. Our $\Delta x = \frac{8}{50} = 0.16$. Then if we want to take the right Riemann sum we know that our $x_{k}^{*} = 0 + k 0.16$. So then our Riemann sum becomes 
$$\sum_{k = 1}^{50} f(0.16k) \Delta x = \sum_{k = 1}^{50} (0.16 k)^2 (0.16) = \sum_{k = 1}^{50} (0.16)^3 k^2$$. 

Using properties of sums and one of the formulas we wrote down before we have that 
\begin{align*}
	\sum_{k = 1}^{50} f(0.16k) \Delta x&= \sum_{k = 1}^{50} (0.16)^3 k^2 \\
									   &= 0.004096 \left(\frac{(50)(51)(101)}{6}\right) \\
									   &\approx 175.821
\end{align*}
(using a calculator). 

\subsection{Definite Integrals}

We can now begin to move toward talking about a definite integral (as opposed to indefinite integrals which we have dealt with before)
\subsubsection{Net Area}

Previously all of the function we approximate the area under the curve of were nonnegative. But what happens if we have a function where the curve goes below the $x$-axis? In this case we say that the area is negative. 

For example if we consider the Riemann sum for the function $f(x) = 1 - x^2$ on the interval $[1, 3]$ with $n = 4$, we see that when we calculate the Riemann sum we get a value of $-6.625$. This is because values for $f(x)$ on this interval are negative. 

If we were to extend our interval to $[0, 3]$ we would then see that the area from $[0, 1.5]$ would be positive and add while we subtract the area below the curve. So for example if we are using $n = 6$ on the interval $[0, 3]$ we have two rectangles that are positive and the rest that are negative resulting in a net Riemann sum of -3.875. 

\section{7/15} 
\subsection{The Definite Integral} 

Like I have hinted at before, we want to be able to implement limits into our Riemann sum to get towards a definition of our integral. We will do this by changing our partition from a regular one (one that has subintervals of even length) to a more general partition. 

\begin{Definition}
	(General Riemann Sum) Suppose $[x_0, x_1], [x_1, x_2], \ldots, [x_{n - 1}, x_n]$ are subintervals of $[a, b]$ with $a = x_0 < x_1 < \cdots < x_{n - 1} < x_n = b$. Let $\Delta x_k$ be the length of the subinterval $[x_{ k -1}, x_k]$ and let $x_{k}^*$ be any point in that subinterval. Then $\sum_{k = 1}^{n} f(x_{k}^{*}) \Delta x_k$ is called a general Riemann sum for $f$ on $[a, b]$. 
\end{Definition}

In practice you will not really work with general Riemann sums, but regular Riemann sums like in many of the previous examples. 

Now note that if we force the length of the subintervals to get smaller and smaller we will need more and more subintervals to fully partition our interval $[a, b]$. That is to says as $\Delta \to 0$ we have that $n \to \infty$. This is the limit that we want to define our integral. 
\begin{Definition}
	A function $f$ defined on $[a, b]$ is integrable on $[a, b]$ if $\lim_{\Delta \to 0} \sum_{k = 1}^{n} f(x_{k}^*) \Delta x_k$ exists and is unique over all partitions of $[a, b]$ and all choices of $x_{k}^{*}$ on a partition. This limit is the definite integral of $f$ from $a$ to $b$ which we write $$\int_{a}^{b} f(x) \; dx = \lim_{\Delta \to 0} \sum_{k = 1}^{n} f(x_{k}^*) \Delta x_k.$$
\end{Definition}

It is important to note that for a definite integral $\int_{a}^{b} f(x) \; dx$ $a$ is the lower bound while $b$ is the upper bound. The $f(x)$ is called the integrand and the $dx$ is the differential. It is important to always remember the $dx$. Another thing to note is that variable $x$ is a dummy variable much like the $k$ in the summation notation. We can replace the $x$ with any other variable we choose for the integral

\subsubsection{Evaluating Definite Integrals (Geometry)} 

Let us start talking about how we evaluate these definite integrals. There are certain definite integrals where we do not need to look at Riemann sums for and can just evaluate based on geometry. 

\begin{Example}
	Consider the integral $\int_{-1}^{1} x \; dx$. This happens to be an integral that I have mentioned before. Let us draw a sketch of what the area we are trying to find is. Note that this area is composed of two triangles and that one is below the $x$ axis. This means we are going to be adding the area above the $x$-axis and subtracting the area below. Giving us a net area of $0$. 
\end{Example}

Let us consider another example 
\begin{Example}
	Consider the integral of the function $\int_{3}^{4} 2x  + 3 \; dx$. We can draw this as well. Note that this area is going to be trapezoid. If you do not remember the area formula for a trapezoid then you can look that up. In this case our area is going to be $\frac{1}{2}(9 + 11) = 10$. 
\end{Example}

The last example of a definite integral that you can calculate using geometry would be the following 
\begin{Example}
	Consider the integral $\int_{0}^{1} \sqrt{1 - (x - 1)^2} \; dx$. This function is the upper half of a circle of radius 1 centered at (1, 0). So in particular the area that this is calculating would be a quarter of the area of said circle. We know the formula for the area of a circle so this would mean that the value of this integral is $\frac{\pi}{4}$. 
\end{Example}

These are three types of functions you will the most commonly when asked a question about evaluating a definite integral using geometry. We will eventually talk about how to evaluate the first two integrals analytically in a bit, however the third integral will only get covered in calc 2. 

What about calculating definite integrals that we cannot do geometrically? Well we will get to that after looking at a few properties of definite integrals. 

\subsubsection{Properties of Definite Integrals}

The first two properties we will look at are what happens when we reverse the order of integration and when we have identical limits of integration - 
\begin{Definition}
	\begin{align}
		\int_{a}^{b} f(x) \; dx &= - \int_{b}^{a} f(x) \; dx \\
		\int_{a}^{a} f(x) \; dx &= 0.
	\end{align}
\end{Definition}


We talked about the indefinite integral of a sum earlier and similarly we have a property about the definite integral of a sum 
\begin{Theorem}
	$$\int_{a}^{b} (f(x) + g(x)) \;dx = \int_{a}^{b} f(x) \; dx + \int_{a}^{b} g(x) \; dx$$
\end{Theorem}

There is a similar property about a constant multiple
\begin{Theorem}
	$$\int_{a}^{b} c f(x) \; dx = c \int_{a}^{b} f(x) \; dx$$
\end{Theorem}

Both of these properties are not hard to see directly from the definition of a definite integral and using the sum properties we discussed last class. 

\begin{align*}
	\int_{a}^{b} f(x) + g(x) \;dx &= \lim_{\Delta \to 0} \sum_{k = 1}^{n} f(x_{k}^{*}) \Delta x_k \\ &= \lim_{\Delta \to 0} \left(\sum_{k = 1}^{n} f(x_{k}^*) \Delta x_k + \sum_{k = 1}^{n} g(x_{k}^*) \Delta x_k \right) \\
	&= \int_{a}^{b} f(x) \; dx + \int_{a}^{b} g(x) \; dx.
\end{align*}

Similarly using the constant multiple property of sums we get that constant multiple property of definite integrals. 

If we consider the integral of $f(x)$ over an interval $[a, b]$ we have $\int_{a}^{b} f(x) \; dx$. However, what if we want to split our interval into two? Take $p$ such that $a < p < b$ then we have the following property about integrals $\int_{a}^{b} f \; dx = \int_{a}^{p} f(x) \; dx + \int_{p}^{b} f(x) \; dx$. This property even works when we have $p$ being outside of the interval. Instead we have that $\int_{a}^{b} f = \int_{a}^{p} f - \int_{b}^{p} f$. We can rewrite the second integral using our first property to get as a sum $\int_{a}^{p} f + \int_{p}^{b} f$. 

What about taking the integrals of absolute values? Well this integral is just going to be the sum of the areas of the region bounded by the graph of $f$ and the $x$-axis. 

\subsubsection{Bounds on Definite Integrals}

The next major topic we are going to talk about is an important theorem in calculus. Before we get into it let us talk about putting some bounds on integrals that will help us. 

\begin{Theorem}
	If $f(x) \geq 0$ on $[a, b]$ then $\int_{a}^{b} f(x) \; dx \geq 0$. 
\end{Theorem}

We can also compare definite integrals. 
\begin{Theorem}
	If $f(x) \geq g(x)$ on $[a, b]$ then $\int_{a}^{b} f(x) \; dx \geq \int_{a}^{b} g(x) \; dx$. 
\end{Theorem}

We can also stick lower and upper bounds on the value of an integral.
\begin{Theorem}
	As $f$ is continuous on $[a, b]$, $f$ attains an absolute maximum $M$ and $f$ attains an absolute minimum $m$ on $[a, b]$. So then $$m(b - a) \leq \int_{a}^{b} f(x) \; dx \leq M(b - a)$$. 
\end{Theorem}

These are three additional properties of definite integrals. 
\subsubsection{Using Limits to Evaluate Definite Integrals} 

We can use our knowledge of Riemann sums and limits to evaluate certain definite integrals. 

\begin{Example}
	Let us find $\int_{0}^{2} (x^3 + 1) \; dx$ by evaluating a right Riemann sum and letting $n \to \infty$. 
	
	Let us take our interval $[0, 2]$ and divide it into $n$ subintervals of length $\Delta x = \frac{2}{n}$. Since we are calculating a right Riemann sum, we know that the point we are going to be using $x_{k}^* = \frac{2k}{n}$. 
	
	So then our right Riemann sum is 
	\begin{align*}
		\sum_{k = 1}^{n} f(x_{k}^{*}) \Delta x &= \sum_{k = 1}^{n} \left(\left(\frac{2k}{n}\right)^3 + 1 \right) \frac{2}{n} \\
		&= \frac{2}{n} \sum_{k = 1}^{n} \left(\frac{8 k^3}{n^3} + 1 \right) \\
		&= \frac{2}{n} \left(\frac{8}{n^3} \sum_{k = 1}^{n} k^3 + \sum_{k = 1}^{n} 1 \right) \\
		&= \frac{2}{n} \left(\frac{8}{n^3} \left(\frac{n^2 (n + 1)^2}{4}\right) + n\right) \\
		&= \frac{4(n^2 + 2n + 1)}{n^2} + 2.
	\end{align*}

	Then to evaluate $\int_{0}^{2} (x^3 + 1) \; dx$ we let $n \to \infty$. 
	So
	\begin{align*}
		\int_{0}^{2} (x^3 + 1) \; dx &= \lim_{n \to \infty} \sum_{k = 1}^{n} f(x_{k}^*) \Delta x \\
		&= \lim_{n \to \infty} \left(\frac{4(n^2 + 2n + 1)}{n^2} + 2 \right) \\
		&= 4 \lim_{n \to \infty} \left(\frac{n^2 + 2n + 1}{n^2} \right) + \lim_{n \to \infty} 2 \\
		&= 4(1) + 2 = 6.
	\end{align*}

Therefore, $\int_{0}^{2} (x^3 + 1)\; dx = 6$. 
\end{Example}


As we can see calculating an integral directly from a Riemann sum is quite tedious. We want to find an easier way to evaluate a definite integral and that is our next topic - The Fundamental Theorem of Calculus. 

\subsection{The Fundamental Theorem of Calculus} 

\subsubsection{Area Functions}

Let us talk about area functions. We have already seen that integrals represent area under curves so it should come as no surprise that area functions can be expressed as an integral. 

So let us define an area function - 
\begin{Definition}
	Let $f$ be a continuous function for $t \geq a$. The area function for $f$ with left endpoint $a$ is $$A(x) = \int_{a}^{x} f(t) \; dt$$ where $x \geq a$. The area function gives the net area of the region bounded by the graph of $f$ and the $t$-axis on the interval $[a, x]$. 
\end{Definition}

Let us work through an example. 
\begin{Example}
	Consider the trapezoid bounded by the line $f(t) = 2t + 3$ and the $t$-axis between $t = 2$ and $t = x$. The area function $A(x) = \int_{2}^{x} f(t) \; dt$ gives the area of the trapezoid for $x \geq 2$. (Draw a picture)
	
	\begin{enumerate}
		\item[a.] We can find $A(2)$ - this is just going to be 0. 
		\item[b.] If we want to find $A(5)$ we just have that we have a trapezoid with height $3$ and sum of the two parallel sides being $f(2) + f(5)$. So then $A(5) = \int_{2}^{5} 2t + 3 \; dt = \frac{1}{2}(3)(7 + 13) = 30$. 
		\item[c.] If we want to find a general formula for what $A(x)$ is we again go back to our formula for the area of a trapezoid. So 
		\begin{align*}
			A(x) &= \frac{1}{2} (x - 2) (f(2) + f(x)) \\
				 &= \frac{1}{2} (x - 2) (7 + 2x + 3) \\
				 &= (x - 2) (x + 5) \\
				 &= x^2 + 3x - 10.
		\end{align*}
	
		So we get that a formula for the area of the trapezoid where $x \geq 2$ is $A(x) = x^2 + 3x - 10$. 
		\item[d.] This is an important thing to note for what we will cover next class, but if we take the derivative of $A(x)$ note what we get - $A'(x) = 2x + 3$. This happens to be exactly what the equation of the line that we are finding the area under is. 
	\end{enumerate}
\end{Example}


\section{7/16}
\subsection{Fundamental Theorem of Calculus} 

The fundamental theorem of calculus is usually given in two ways. We will now look at the first way it is given using area functions. If we have an area function $A(x)$ we can consider the difference between $A(x + h)$ and $A(x)$. (draw a picture). We can see the area that is different between the two is approximately $h f(x)$. So this gives us the relation that $$\frac{A(x + h) - A(x)}{h} \approx f(x).$$

Now we do something that should look familiar - we take the limit as $h \to 0$. This gives us $$\lim_{h \to 0} \frac{A (x + h) - A(x)}{h} = \lim_{h \to 0} f(x).$$

Notice that the left hand side should look very similar - this is the derivative of $A(x)$ with respect to $x$, in other words $A'(x)$. From this we have $$A'(x) = \frac{d}{dx} \int_{a}^{x} f(t) \; dt = f(x).$$ This just really says the derivative of the integral of $f$ is $f$, which should seem like a very obvious fact. This is the first part of the Fundamental Theorem of Calculus. 
\begin{Theorem}
	(FTC Part 1) If $f$ is continuous on $[a, b]$ then the area function \begin{align*}
			A(x) = \int_{a}^{x} f(t) \; dt, \textnormal{ for } a \leq x \leq b,
	\end{align*}
	is continuous on $[a, b]$ and differentiable on $(a, b)$. Then the area function satisfies $A'(x) = f(x)$ or in other words \begin{align*}
		A'(x) = \frac{d}{dx} \int_{a}^{x} f(t) \; dt = f(x),
	\end{align*}
	which means that the area function of $f$ is an antiderivative of $f$ on $[a, b]$. 
\end{Theorem} 

Now this is a cool theorem and there will always be problems that you will see that use this aspect of the fundamental theorem, but this does not really help us evaluate a definite integral in general. 

The second part of the FTC which we are getting to will do that. Remember that when we were talking about antiderivatives we noted that any two antiderivatives of a function differ by a constant. So if $A(x)$ is an antiderivative of $f$ on $[a, b]$ and $F(x)$ is another antiderivative of $f$ then $F(x) = A(x) + C$. As $A(a) = 0$ (we say this yesterday when talking about area functions) then if we look at $F(b) - F(a)$ we get $(A(b) + C) - (A(a) + C) = A(b)$. By definition of area function we know that $A(b) = \int_{a}^{b} f(t) \; dt$. Putting all of this together and replacing $t$'s with $x$'s we have $\int_{a}^{b} f(x) \; dx = F(b) - F(a)$. 

\begin{Theorem}
	(FTC Part 2) If $f$ is continuous on $[a, b]$ and $F$ is any antiderivative of $f$ on $[a, b]$, then \begin{align*}
		\int_{a}^{b} f(x) \; dx = F(b) - F(a).
	\end{align*}
	We use a shorthand notation to denote $F(b) - F(a)$ which can either be notated as $\left.F(x)\right|_{a}^{b}$ or $\left[F(x)\right]_{a}^{b}$. 
\end{Theorem}

The Fundamental Theorem is an important enough theorem that it is worth going through the proof of. So let us talk about how we get there. 

Let $f$ be continuous on $[a, b]$ and let $A$ be the area function for $f$ with left endpoint $a$. We will first show that $A$ is differentiable on $(a, b)$ and $A'(x) = f(x)$.

We assume that $a < x < b$ and use the definition of the derivative we have 
\begin{align*}
	A'(x) = \lim_{h \to 0} \frac{A(x + h) - A(x)}{h}.
\end{align*}

Let us look at the numerator. We note that by definition $A(x + h) - A(x) = \int_{a}^{x + h} f(t) \; dt - \int_{a}^{x} f(t) \; dt = \int_{x}^{x + h} f(t) \; dt$ (this is by a property of integrals we discussed). Notice that this integral is the net area under the curve $y = f(t)$ on the interval $[x, x + h]$. So we know that as $f$ is continuous on this interval, $f$ attains an absolute max $M$ and absolute min $m$ , so $m \leq f(t) \leq M$ on $[x, x + h]$. 

We then know from one of the properties we discussed yesterday that $$mh \leq \int_{x}^{x + h} f(t) \; dt \leq M h.$$ 

Substituting for the integral we get $$mh \leq A(x + h) - A(x) \leq M h.$$ 
We can then divide by $h$ and take the limit as $h \to 0$ to get $$\lim_{h \to 0} m \leq \lim_{h \to 0} \frac{A(x + h) - A(x)}{h} \leq \lim_{h \to 0} M.$$

Now the $\lim_{h \to 0} m$ refers to what happens to our absolute minimum as $h \to 0$. As $h \to 0$ our interval $[x, x + h]$ gets smaller and smaller till the only point left is $f(x)$. So $\lim_{h \to 0} m = f(x)$. Similarly $\lim_{h \to 0} M = f(x)$. Note that the limit in the middle is by definition $A'(x)$. So by the Squeeze Theorem we have that $A'$ exists (so $A$ is differentiable for $a < x < b$) and $A'(x) = f(x)$. This proves the first part of the FTC. 

The second part of the FTC follows from the same work we did before. 


Now that we have show the FTC let us actually put it to use. We use part 2 of the FTC to evaluate definite integrals. So let us do this - 

\begin{Example}
	\begin{enumerate}
		\item[a.] \begin{align*}
			\int_{0}^{10} (60 x - 6x^2) \; dx &= \left[30 x^2 - 2 x^3\right]_{0}^{10} \\
						&= (30(100) - 2(1000)) - (30(0) - 2(0)) = 3000-2000 = 1000
		\end{align*}
	
	\item[b.] \begin{align*}
		\int_{0}^{2\pi} 3 \sin x \; dx &= \left. -3 \cos x \right|_{0}^{2\pi} \\
		&= (-3 \cos 2\pi) - (-3 \cos 0)  = -3 -(-3) = 0
	\end{align*}
	
	\item[c.] \begin{align*}
		\int_{1/16}^{1/4} \frac{\sqrt{t} - 2t}{t} \; dt &= \int_{1/16}^{1/4} (t^{-1/2} - 2)\; dt \\
		&= \left[2 t^{1/2} - 2t\right]_{1/16}^{1/4} \\
		&= (2 (1/2) - 2(1/4)) - (2(1/4) - 2(1/16)) \\
		&= 1 - (1/2) - (1/2) + 1/8 = 1/8
	\end{align*}
	\end{enumerate}
\end{Example}

These were examples of where we are using part 2 of the FTC, but you can also run into problems where you have to use part 1 of the FTC. These are problems that like to show up on tests/exams. 

\begin{Example}
	\begin{enumerate}
		\item[a.] Simplify $\frac{d}{dx} \int_{x}{1} \sin^2 t \; dt$. 
		
		To do this problem remember what was required of part 1 of the FTC. We had an area function $A(x) = \int_{a}^{x} f(t) \; dt$ and that $A'(x) = f(x)$. So we want to make sure our integral is written in the same way as our area function (mainly the $x$ has to be the upper bound). 
		
		\begin{align*}
			\frac{d}{dx}\int_{x}^{1} \sin^2 t \; dt &= \frac{d}{dx} \left(- \int_{1}^{x} \sin^2 t \; dt \right)\\
			&= - \sin^2 x 
		\end{align*}
	
		\item[b.] A slightly more involved example - simplify $\frac{d}{dx} \int_{0}^{x^2} \cos t^2 \; dt$. The reason this is more involved is because this does not have just an $x$ as the upper bound but an $x^2$. If we were to write this as an area function we would have $A(x) = \int_{0}^{x} \cos t^2 \; dt$, so we want to find $\frac{d}{dx} A(x^2)$. This is a problem where we have to use the chain rule. 
		\begin{align*}
			\frac{d}{dx} \int_{0}^{x^2} \cos t^2 \; dt &= \frac{d}{dx} A(x^2) \\
			&= A'(x^2) (2x) \\
			&= 2x \cos x^4.
		\end{align*}
		Note $A'(x) = \frac{d}{dx} \int_{0}^{x} \cos t^2 \; dt = \cos x^2$. 
	\end{enumerate}

	In general you can refer to the formula $$\frac{d}{dx} \int_{0}^{g(x)} f(t) \; dt = f(g(x)) g'(x).$$ 
\end{Example}


\subsubsection{Integrating Even and Odd Functions} 

This was briefly mentioned yesterday but there are some tricks we can use when integrating even and odd functions over a symmetric interval. The tricks are as follows - 
\begin{Theorem}
	\begin{align*}
		\int_{-a}^{a} f(x) \; dx &= 2 \int_{0}^{a} f(x) \; dx \; f\textnormal{ is even} \\
		\int_{-a}^{a} f(x) \; dx &= 0 \; f\textnormal{ is odd.}
	\end{align*}
\end{Theorem}

Remember that an even function is a function where $f(x) = f(-x)$ and an odd function is $f(x) = - f(-x)$. 

We take a look at evaluating integrals using this property 
\begin{Example}
	\begin{enumerate}
		\item[a.] 
		\begin{align*}
			\int_{-2}^{2} (x^4 - 3x^3) \; dx &= \int_{-2}^{2} x^4 \; dx - 3 \int_{-2}^{2} x^3 \;dx \\
			&= 2 \int_{0}^{2} x^4 - 3(0) \\
			&= 2 \left[\frac{1}{5}x^5\right]_{0}^{2} \\
			&= 2 \left(\frac{1}{5}2^5 - 0\right) = \frac{64}{5}.
		\end{align*}
	
		\item[b.] 
		\begin{align*}
			\int_{-\pi/2}^{\pi/2} (\cos x - 4 \sin^3 x) \;dx &= 2 \int_{0}^{\pi/2} \cos x \; dx - 0  \\
			&= 2 \left.\sin x \right|_{0}^{\pi/2} = 2(\sin \frac{\pi}{2} - 0) = 2(1) = 2. 
		\end{align*}
	\end{enumerate}
\end{Example}


\subsubsection{Average Value of a Function} 

Much like how we can find the average of a set of numbers, we can also find the average value of a function on an interval. This is how we would do it. 

Consider a function $f$ continuous on $[a, b]$. Using a regular partition $x_0 = a < x_1 < x_2 < \cdots < x_n = b$ with $\Delta x = \frac{b - a}{n}$, we can select a point $x_{k}^*$ from each subinterval and find $f(x_{k}^*)$. The average value of the function on this interval using these $n$ points is $$\frac{f(x_{1}^*) + f(x_{2}^*) + \cdots + f(x_{n}^*)}{n}.$$

Remember that $\Delta x = \frac{b - a}{n}$ so we can express $n$ as $\frac{b - a}{\Delta x}$. Plugging this into our expression from above we get $$\frac{f(x_{1}^*) + f(x_{2}^*) + \cdots + f(x_{n}^*)}{(b - a)/\Delta x} = \frac{1}{b - a} \sum_{k = 1}^{n} f(x_{k}^*) \Delta x.$$

The sum on the right should look familiar - this is a Riemann sum. If we take the limit as $n \to \infty $ we get a definite integral that gives us the average value $\overline{f}$ on $[a, b]$ - 
\begin{align*}
	\overline{f} &= \frac{1}{b - a} \lim_{n \to \infty} \sum_{k = 1}^{n} f(x_{k}^*) \Delta x \\ 
	&= \frac{1}{b - a} \int_{a}^{b} f(x) \; dx. 
\end{align*}

This is how we define the average value of a function on an interval. Let us see some examples 
\begin{Example}
	A hiking trail has an elevation given by $f(x) = 60x^3 - 650 x^2 + 1200 x + 4500$, where $f$ is measured in feet above sea level and $x$ represents horizontal distance along the trail in miles with $0 \leq x \leq 5$. What is the average elevation of the trail. 
	
	The trail ranges between elevations of about 2000 and 5000 ft. Let the end points of the trail be our interval, $[0, 5]$. The average elevation of the trail in feet is given by 
	\begin{align*}
		\overline{f} = \frac{1}{b - a} \int_{a}^{b} f(x) \; dx &= \frac{1}{5 - 0} \int_{0}^{5} (60 x^3 - 650 x^2 + 1200 x + 4500) \;dx \\
		&= \frac{1}{5} \left[ 15 x^{4} - \frac{650}{3} x^3 + 600 x^2 + 4500 x\right]_{0}^{5} \\
		&= \frac{1}{5} \left( 15 (5^4) - \frac{650}{3} (5)^3 + 600 (5^2) + 4500 (5) \right) \\
		&= 15 (5^3) - \frac{650}{3} 5^2 + 600(5) + 4500 \\
		&= 1875 - \frac{16250}{3} + 3000 + 4500 \\
		&= \frac{5625 - 16250 + 9000 + 13500}{3} = \frac{11875}{3} = 3958\frac{1}{3}.
	\end{align*}

	So the average elevation of the trail is slightly less than 3960 ft. 
\end{Example}

\subsubsection{Mean Value Theorem for Integrals} 

The idea of the average value of a function over an interval is part of a very important theoretical result. Much like we had a Mean Value Theorem for derivatives, which stated that if $f$ was continuous on $[a, b]$ and differentiable on $(a, b)$ there is a $c$, $a < c < b$ such that $f'(c) = \frac{f(b) - f(a)}{b - a}$, we have something similar using the average value of a function over a interval. 

\begin{Theorem}
	(MVT for Integrals) Let $f$ be continuous on the interval $[a, b]$. There exists a point $c$ in $(a, b)$ such that $$f(c) = \overline{f} = \frac{1}{b - a} \int_{a}^{b} f(t) \; dt.$$ 
\end{Theorem}

In fact the proof of this theorem uses the MVT for functions that I just mentioned. Let us see this in action - Define an area function $F(x) = \int_{a}^{x} f(t)\;dt$. By the FTC (part 1) we know that $F$ is continuous on $[a, b]$ and differentiable on $(a, b)$. Applying the MVT for derivatives we know that there is a point $c$ in $(a, b)$ such that 
\begin{align*}
	F'(c) = \frac{F(b) - F(a)}{ b - a}. 
\end{align*}

By the FTC part 1 we know that $F'(c) = f(c)$ and by FTC part 2 we know that $F(b) - F(a) = \int_{a}^{b} f(t) \;dt$. Putting this together we get the result we wanted which is 
\begin{align*}
	f(c) = \frac{1}{b - a} \int_{a}^{b} f(t) \; dt,
\end{align*}
where $c$ is a point in $(a, b)$. 


Much like for the MVT for derivatives you might run into a problem asking you to use the the MVT for derivatives - 
\begin{Example}
	Find the point(s) on the interval $(0, 1)$ at which $f(x) = 2x(1 - x)$ equals its average value on $[0, 1]$. 
	
	The first thing we need to do is find the average value using the formula - 
	\begin{align*}
		\overline{f} = \frac{1}{1 - 0} \int_{0}^{1} 2x(1 - x) \; dx &= \int_{0}^{1} 2x - 2x^2 \;dx \\
		&= \left[x^2 - \frac{2}{3} x^3 \right]_{0}^{1}  \\
		&= \left(1 - \frac{2}{3} \right) = \frac{1}{3}.
	\end{align*}

	So we want to find $x$ such that $2x (1 - x) = \frac{1}{3}$. We get a quadratic $2x^2 - 2x + \frac{1}{3} = 0$. Using the quadratic theorem we get two points $$\frac{1 - \sqrt{1/3}}{2} \approx 0.211$$ and $$\frac{1 + \sqrt{1/3}}{2} \approx 0.789.$$ So the $x$ values for the two points where $f(x)$ equals its average value are $x \approx 0.211$ and $x \approx 0.789$. 
\end{Example}

\section{7/19}
\subsection{Substitution Rule} 

As of right now we have only covered how to integrate very few types of integrals - polynomials and trig integrals involving simple trig functions. However what if we want to integrate slightly more complicated functions like when a composition is involved. Like for example 
\begin{Example}
	\begin{align*}
		\int \cos 2x \; dx. 
	\end{align*}

By the chain rule we know that $\frac{d}{dx} \sin 2x = 2 \cos 2x$ So by deductive reasoning we can guess that $\frac{1}{2} \sin 2x$ is an antiderivative of $\cos 2x$. By taking the derivative of this we can see that this is true. 
\end{Example}

\subsubsection{Indefinite Integrals}
In general by the chain rule we have \begin{align*}
	\frac{d}{dx} (F(g(x))) = F'(g(x)) \cdot g'(x).
\end{align*}

If we were to integrate both sides we would have that $\int F'(g(x))g'(x) \; dx = F(g(x))$. The substitution rule is called as such because we substitute $u = g(x)$ and $du = g'(x) dx$ to change our integral into a simpler integral in terms of $u$. 

\begin{Theorem}
	Let $u = g(x)$ where $g$ is differentiable on an interval and let $f$ be continuous on the corresponding range of $g$. On that interval, $$\int f(g(x)) g'(x) \; dx = \int f(u) \; du.$$
\end{Theorem}

We can look at a few examples 
\begin{Example}
	\begin{align*}
		\int 2 (2x + 1)^3 \;dx &= \int (2x + 1)^3 (2 dx) \\
							   &= \int u^3 \; du \\
							   &= \frac{u^4}{4} + C \\
							   &= \frac{(2x + 1)^4}{4} + C.
	\end{align*}

	Here we used $u = 2x + 1$ and $du = 2 \; dx$. 
\end{Example}

Another examples - 
\begin{Example}
	\begin{align*}
		\int 2x \cos x^2 \;dx &= \int \cos x^2 (2x \; dx) \\
							  &= \int \cos u \; du \\
							  &= \sin u + C \\
							  &= \sin x^2 + C
	\end{align*}
	Here we used $u = x^2$ and $du = 2x \; dx$. 
\end{Example}

These are two examples of "Perfect substitutions" where we had all the pieces for our substitution already in place. Many times we will not have a perfect substitution and instead have to introduce a constant. 
\begin{Example}
	\begin{align*}
		\int x^4 (x^5 + 6)^{9} \;dx &= \int (x^5 + 6)^9 (x^4 \; dx) \\
								    &= \int \frac{1}{5} u^9 \; du \\
								    &= \frac{1}{50} u^10 + C \\
								    &= \frac{1}{50}(x^5 + 6)^{10} + C. 
	\end{align*}
	Here we have $u = x^5 + 6$, but $du = 5x^4 \; dx$. In our original integral we only have an $x^4$ but not the $5$ so what we do is solve for what we have. Since we have $x^4 \; dx$ we divide by $5$ to get $x^4 \; dx = \frac{du}{5}$. So this is what we substitute in. 
\end{Example}

Another example is 
\begin{Example}
	\begin{align*}
		\int \cos^3 x \sin x \; dx &= \int u^3 (- du) \\
								   &= \frac{u^4}{4} + C \\
								   &= \frac{\cos^4 x}{4} + C.
	\end{align*}
	Here $u = \cos x$ and $du = - \sin x \; dx$ but since we only have a $\sin x \; dx$ we multiply by negative 1 and get $- du = \sin x dx$. 
\end{Example}

\subsubsection{Procedure for using U-Sub} 
Here is a general substitution you can follow - 
\begin{enumerate}
	\item[1.] Given an indefinite integral involving a composite function $f(g(x))$ identify an inner function $u = g(x)$ such that a constant multiple of $g'(x)$ appears in the integrand. 
	\item[2.] Substitute $u = g(x)$ and $du = g'(x) \;dx$ in the integral 
	\item[3.] Evaluate the new indefinite integral with respect to $u$
	\item[4.] Write the result in terms of $x$ using $u = g(x)$. 
\end{enumerate}

\subsubsection{Variations on Substitution}

Sometimes we do not have that certain constants line up when using substitution. This might be a case where we need to do something slightly different. 
\begin{Example}
	Consider $\int \frac{x}{\sqrt{x + 1}} \; dx$. By our procedure we want to use $u = x + 1$. However $du = dx$ and that leaves the $x$ in the numerator. However what we can do is solve for $x$ in our substitution and get $x = u - 1$. Plugging this in will allow us to evaluate this integral - 
	\begin{align*}
		\int \frac{x}{\sqrt{x + 1}} \; dx &= \int \frac{u - 1}{\sqrt{u}} \; du \\
										  &= \int u^{1/2} - u^{-1/2} \; du \\
										  &= \frac{2}{3} u^{3/2} - 2 u^{1/2} + C \\
										  &= \frac{2}{3} (x + 1)^{3/2} - 2(x + 1)^{1/2} + C. 
	\end{align*}
\end{Example}

There is another way we can do this problem as well 
\begin{Example}
	If instead we used the substitution $u = \sqrt{x + 1}$ then $u^2 = x + 1$ and $x = u^2 - 1$ and $dx = 2u \; du$. So then using these substitutions
	\begin{align*}
		\int \frac{x}{\sqrt{x + 1}} \; dx &= \int \frac{u^2 - 1}{u} 2u \; du \\
									      &= 2 \int (u^2 - 1)\; du \\
									      &= 2 \left(\frac{u^3}{3} - u\right) + C \\
									      &= \frac{2}{3} (x + 1)^{3/2} - 2(x + 1)^{1/2} + C. 
	\end{align*}
\end{Example}

\subsubsection{General Integration Formulas}

For certain integrals using u-sub we can find general formulas for certain trig integrals 
\begin{align*}
	\int \cos ax \; dx &= \frac{1}{a} \sin ax +C \\
	\int \sin ax \; dx &= -\frac{1}{a} \cos ax + C \\
	\int \sec^2 ax \; dx &= \frac{1}{a} \tan ax + C \\
	\int \csc^2 ax \; dx &= -\frac{1}{a} \cot ax + C \\
	\int \sec ax \tan ax \; dx &= \frac{1}{a} \sec ax + C \\
	\int \csc ax \cot ax \; dx &= -\frac{1}{a} \csc ax + C. 
\end{align*}

\subsubsection{Definite Integrals}
We can also use the substitution rule for definite integrals. There are two ways you can do this. One you can use u-sub to first find the antiderivative of the function and use the fundamental theorem. Or you can change the variables and change the limits of integration and complete the integration with respect to $u$. 

\begin{Theorem}
	Let $u = g(x)$ where $g'(x)$ is continuous on $[a, b]$ and let $f$ be continuous on the range of $g$. Then $$\int_{a}^{b} f(g(x))g'(x) \;dx = \int_{g(a)}^{g(b)} f(u) \; du.$$ 
\end{Theorem}

Let us see some examples - 
\begin{Example}
	Consider $\int_{0}^{2} \frac{dx}{(x + 3)^3}$. Let us do this both ways. First evaluating the indefinite integral we have \begin{align*}
		\int \frac{dx}{(x + 3)^3} &= \int \frac{du}{u^3} \\
								  &= -\frac{1}{2} u^{-2} + C \\
								  &= -\frac{1}{2} (x + 3)^{-2} + C. 
	\end{align*}

	Then we use part 2 of the FTC to evaluate $F(2) - F(0)$. 
\end{Example}

\begin{Example}
	$\int \sin^2 x \;dx$ and $\int \cos^2 x \; dx$. 
\end{Example}
\end{document}