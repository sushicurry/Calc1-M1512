\documentclass[11pt]{article}
\usepackage{amssymb,amsmath,fancyhdr,graphicx, amsthm}
\headheight -10pt
%%\voffset 0.3 cm
\textheight 26cm% 25.5cm
\textwidth 18cm
\topmargin -2cm
\parindent 0pt
\oddsidemargin -1.2cm \columnsep 18pt
%\columnseprule 0.3pt
\renewcommand{\headrulewidth}{0pt}
\pagestyle{fancy} \lhead{}\chead{}\rhead{}
\lfoot{}\cfoot{}\rfoot{}
\newcommand{\ds}{\displaystyle}
\usepackage[metapost]{mfpic}
\usepackage{multicol}
\begin{document}



%\textbf{NAME:}\hrulefill
%\vspace{0.12in}

\centerline{\textbf{\Large{Derivatives and Continuity Quiz}}}

\vspace{0.2in}
 

\begin{enumerate}

\item We have seen that the {\em derivative of $f(x)$ at $x=a$} can be represented two ways: \[f'(a)=\lim_{h\rightarrow 0}\frac{f(a+h)-f(a)}{h}\] and \[f'(a)=\lim_{x\rightarrow a}\frac{f(x)-f(a)}{x-a}\]

\begin{enumerate}
	\item Compute the derivative of $f(x) = \sqrt{2x+1}$ at $x = 0$ by using either of the definitions.
	\item Compute the derivative of $f(x) = \frac{7}{3x+1}$ at $x = -1$ by using either of the definitions.
\end{enumerate}

\newpage
\item Consider the function $f(x)$ defined piecewise as :
\begin{align*}
	f(x)=\begin{cases} x^3-cx, & x < -2; \\ 
						x^2 - bx + c, & -2\leq x < 3; \\
					    7 + bx & x \geq 3. \end{cases}
\end{align*}
Find the values of $b$ and $c$ such that $f(x)$ is continuous. Remember that a function $f(x)$ is continuous at $x = a$ if $\lim\limits_{x \to a^{-}} f(x) = \lim\limits_{x \to a^{+}} f(x)$
 




\end{enumerate}



\end{document}