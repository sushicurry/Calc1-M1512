\documentclass[11pt,reqno]{article}
%\input{17F-137-HomeworkTemplate}
\title{Homework \#2}
%\maketitle
%\due{Tuesday 9/12}

\headheight -10pt

%%\voffset 0.3 cm

\textheight 26cm% 25.5cm

\textwidth 18cm

\topmargin -2cm

\parindent 0pt

\oddsidemargin -1.2cm \columnsep 18pt

\usepackage{amsmath}
\usepackage{amsthm}
\usepackage{amssymb}
\usepackage{amsfonts}
%\usepackage{graphicx}
\usepackage{latexsym}
\usepackage{times}
\usepackage{fancyhdr}
\usepackage{url}
\usepackage{multicol}
\usepackage{cite}
\usepackage{hyperref}
\usepackage{mathrsfs}
\usepackage[usenames]{color}
\usepackage{enumitem}
%\usepackage{verbatim}

\usepackage{subcaption}
\usepackage{setspace}


\usepackage{tikz}
\usetikzlibrary{decorations.pathreplacing}
\usetikzlibrary{positioning}
\usetikzlibrary{calc}
\usetikzlibrary{arrows,shapes,backgrounds,3d}
\usepackage{fix-cm}
\usetikzlibrary{decorations.pathreplacing,shapes,snakes}

%%
\newcommand{\coursenumber}{Math 137}
\newcommand{\coursename}{Real and Functional Analysis}


\renewcommand{\Re}[1]{\operatorname{Re} #1 }
\renewcommand{\Im}[1]{\operatorname{Im} #1}
\newcommand{\diam}{\operatorname{diam}}

%% New Commands
\newcommand{\D}{\mathbb{D}}
\newcommand{\Z}{\mathbb{Z}}
\newcommand{\R}{\mathbb{R}}
\newcommand{\Q}{\mathbb{Q}}
\newcommand{\C}{\mathbb{C}}
\newcommand{\F}{\mathbb{F}}
\newcommand{\G}{\mathcal{G}}
\newcommand{\M}{\mathcal{M}}
\newcommand{\K}{\mathbb{K}}
\newcommand{\U}{\mathcal{U}}
\newcommand{\UU}{\mathscr{U}}
\newcommand{\s}{\mathscr{S}}
\newcommand{\A}{\mathcal{A}}
\renewcommand{\L}{\mathfrak{L}}
\newcommand{\B}{\mathcal{B}}
\renewcommand{\P}{\mathcal{P}}
\newcommand{\N}{\mathbb{N}}

\DeclareMathOperator{\Aut}{Aut}



\renewcommand{\ker}{\operatorname{kernel}}
\newcommand{\ran}{\operatorname{range}}

\newcommand{\diag}{\operatorname{diag}}
\newcommand{\norm}[1]{\| #1 \|}
\newcommand{\inner}[1]{\langle #1 \rangle}
\newcommand{\E}{\mathcal{E}}
\newcommand{\V}{\mathcal{V}}
\newcommand{\W}{\mathcal{W}}
\newcommand{\WW}{\mathscr{W}}
\newcommand{\T}{\mathbb{T}}
\newcommand{\vecspan}{\operatorname{span}}
\newcommand{\interior}{\operatorname{int}}
\newcommand{\tr}{\operatorname{tr}}
\newcommand{\rank}{\operatorname{rank}}
\newcommand{\nullity}{\operatorname{nullity}}

\newcommand{\colspace}{\operatorname{colspace}}
\newcommand{\rowspace}{\operatorname{rowspace}}
\newcommand{\nullspace}{\operatorname{nullspace}}

\linespread{1}
\setlength{\parskip}{0.5ex plus 0.5ex minus 0.2ex}

%%  Matrices
\newcommand{\minimatrix}[4]{\begin{bmatrix} #1 & #2 \\ #3 & #4 \end{bmatrix}  }
\newcommand{\megamatrix}[9]{\begin{bmatrix} #1 & #2 & #3 \\ #4 & #5 & #6 \\ #7 & #8 & #9\end{bmatrix}  }

\renewcommand{\vec}[1]{{\bf #1}}
\newcommand{\ovec}{\operatorname{vec}}
\renewcommand{\labelenumi}{(\roman{enumi})}
\renewcommand{\hat}{\widehat}

\newcommand{\tworowvector}[2]{[#1\,\,#2]}
\newcommand{\threerowvector}[3]{[#1\,\,#2\,\,#3]}
\newcommand{\fourrowvector}[4]{[#1\,\,#2\,\,#3\,\,#4]}
\newcommand{\fiverowvector}[5]{[#1\,\,#2\,\,#3\,\,#4\,\,#5]}
\newcommand{\sixrowvector}[6]{[#1\,\,#2\,\,#3\,\,#4\,\,#5\,\,#6]}

\newcommand{\twovector}[2]{\begin{bmatrix} #1\\#2 \end{bmatrix} }
\newcommand{\threevector}[3]{\begin{bmatrix} #1\\#2\\#3 \end{bmatrix} }
\newcommand{\fourvector}[4]{\begin{bmatrix} #1\\#2\\#3\\#4 \end{bmatrix} }
\newcommand{\fivevector}[5]{\begin{bmatrix} #1\\#2\\#3\\#4\\#5 \end{bmatrix} }

\newcommand{\comment}[1]{\marginpar{\scriptsize\color{gray}$\bullet$\,#1}}
\newcommand{\highlight}[1]{\color{blue}#1\color{black}}

\renewcommand{\labelenumi}{\theenumi}
\renewcommand{\theenumi}{(\alph{enumi})}%
\renewcommand{\labelenumii}{\theenumii}
\renewcommand{\theenumii}{(\roman{enumii})}%
\newcommand{\due}[1]{\vspace{-0.2in}\begin{center}\textsc{due at the beginning of class \underline{#1}} \end{center}\medskip }


%%%
%%% Theorem Styles
%%%
\newtheorem{Proposition}{Proposition}
\newtheorem{Corollary}{Corollary}
\newtheorem{Theorem}{Theorem}
\newtheorem*{Thm}{Theorem}
\newtheorem{Postulate}{Postulate}
\newtheorem{Lemma}{Lemma}
\theoremstyle{definition}
\newtheorem*{Definition}{Definition}
\newtheorem*{Example}{Example}
\newtheorem*{Remark}{Remark}
\newtheorem{problem}{Exercise}
\newtheorem*{Question}{Question}

\let\oldenumerate=\enumerate
\def\enumerate{
	\oldenumerate
	\setlength{\itemsep}{1pt}
}
\let\olditemize=\itemize
\def\itemize{
	\olditemize
	\setlength{\itemsep}{5pt}
}

\allowdisplaybreaks
\begin{document}
	\centerline{\textbf{\Large{Quiz Problems}}}
	
	\vspace{0.2in}
	
	\begin{enumerate}
	\item[1.] Consider the function
	\begin{align*}
		f(x) = \frac{1}{\sqrt{8 \pi}} e^{- \frac{(x - 1)^2}{8}}
	\end{align*}
	
	Given that the first and second derivatives are:
	\begin{align*}
		f'(x) &= \frac{1}{8 \sqrt{2 \pi}} (x - 1) e^{- \frac{(x - 1)^2}{8}} \\
		f''(x) &= \frac{1}{32 \sqrt{2 \pi}} (x^2 - 2x - 3) e^{- \frac{(x - 1)^2}{8}}
	\end{align*}
	
	\begin{enumerate}
		\item What are the critical points of $f(x)$. Is there a local max or min ? (Make sure to give the actual point(s) not just the $x$-values). 
		\item What are the points of inflection ? (Make sure to give the actual points, not just the $x$ values). Intervals of concavity? 
		\item Sketch the graph of $f(x)$, making sure to label all the points you found above. 
		\item Bonus question: $f(x)$ is actually a special type of function. What special function is it? 
		
	\end{enumerate}
	\item[2.] Find the arc length of $y = \frac{1}{3}(x^2 + 2)^{3/2}$ on $[0, 1]$
	\item[3.] Find the volume of the solid of revolution created by rotating the region bounded by $y = 1, y = \sqrt{\sin x}$, when $0 \leq x \leq \frac{\pi}{2}$ about the $x$-axis.
	\item[4.]  Let  $f(x) = x^{3/2}$ and $g(x) = x^{2/3}$. Find the area between these two curves.
	\item[5.] Find the area between $f(x) = x^2$ and $g(x) = x^4 - x^2$ when $x$ is negative
	\item[6.] Evaluate the following integrals using substitution -
	\begin{enumerate}
		\item[a.] $$ \int \frac{x^2}{\sqrt{x +  2}} \; dx$$ 
		%\vspace{8cm}
		\item[b.] $$\int_{2}^{3} x\sqrt{x^2 - 4} \; dx$$
	\end{enumerate}
	\item[7.] Evaluate the following definite integrals by using geometry.
	\begin{enumerate}
		\item[a.] $$ \int_{-4}^{2} (2x + 4) \; dx$$ 
		%\vspace{8cm}
		\item[b.] $$\int_{-1}^{3} \sqrt{4 - (x - 1)^2} \; dx$$
	\end{enumerate}
	\item[8.] Evaluate the following integrals 
	\begin{enumerate}
		\item[a.] \begin{align*}
			\int \frac{12 t^8 - t}{t^{3/2}} \; dt 
		\end{align*}
		\item[b.] $$ \int \sec \theta (\tan \theta + \sec \theta + \sin 2\theta) \; d\theta$$ (Hint: $\sin 2\theta$ can be written as what using double angle formulas?)
	\end{enumerate}
	\item[9.] Compute the linearization $L(x)$ of $f(x) = \sqrt{1 + x}$ at $x = 8$ and use this to estimate the value of $\sqrt{9.2}$. 
	\item[10.] Consider the function $$f(x) = \frac{x^2}{x^2 + 9}.$$ Find the intervals of increase/decrease as well as the intervals on which $f$ is concave up and concave down. Then using the test of your choice find where $f$ has local extreme values.
	\item[11.] Find the absolute extreme values, so both absolute maximum and absolute minimum, of $f(x) = x + 2 \sin x \cos x$ on the interval $[0, \frac{\pi}{2}]$.
	\item[12.] Find the equation of the tangent line to the curve $(x^2 + y^2)^2 = \frac{25}{4}xy^2$ at the point (1,2)
	\item[13.] Sand is poured onto a surface at 15 $\textnormal{cm}^3 / \textnormal{sec}$, forming a conical pile whose base diameter is always equal to its altitude ($h = d$). How fast is the altitude of the pile increasing when the pile is 3 cm high? (Hint: the volume of a cone was mentioned in class. You will need to use the fact that $h = d$ and the relationship between diameter and radius of a circle). 
	\item[14.]  Differentiate the following functions. Simplify your answers.
	\begin{enumerate}
		\item[a.] $y = \frac{4u^2 + u}{8u + 1}$ 
		%\vspace{12cm}
		\item[b.] $ y = 5 x^3 - 3 x + \sqrt{x} - \frac{4}{x^2 \sqrt{x}}$
	\end{enumerate}  
	\item[15.] Let \[y = \frac{f(x)}{g(x)}\]. Show that \[y' = \frac{f'(x) g(x) - f(x) g'(x)}{[g(x)]^2}\]. \textbf{Do not use the quotient rule!}\textit{Hint:} Rewrite $y$ as a product and then differentiate. 
	\item[16.] Find the derivative function using the definition of $f(x) = \frac{1}{x}$ and find the equation of the tangent line at $a = -5$
	\item[17.] Find the point at which the tangent line to the curve given by $y = x^3 - 2x^2 + x + 8$ has the smallest slope. In other words, find the point at which the slope of the tangent line is at a minimum. What is the slope at that point? 
	
	\end{enumerate}
	
	



	
	
	
	
\end{document}