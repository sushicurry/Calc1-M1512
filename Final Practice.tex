\documentclass[11pt,reqno]{article}
%\input{17F-137-HomeworkTemplate}
\title{Homework \#2}
%\maketitle
%\due{Tuesday 9/12}

\headheight -10pt

%%\voffset 0.3 cm

\textheight 26cm% 25.5cm

\textwidth 18cm

\topmargin -2cm

\parindent 0pt

\oddsidemargin -1.2cm \columnsep 18pt

\usepackage{amsmath}
\usepackage{amsthm}
\usepackage{amssymb}
\usepackage{amsfonts}
\usepackage{graphicx}
\usepackage{latexsym}
\usepackage{times}
\usepackage{fancyhdr}
\usepackage{url}
\usepackage{multicol}
\usepackage{cite}
\usepackage{hyperref}
\usepackage{mathrsfs}
\usepackage[usenames]{color}
\usepackage{enumitem}
%\usepackage{verbatim}

\usepackage{subcaption}
\usepackage{setspace}


\usepackage{tikz}
\usetikzlibrary{decorations.pathreplacing}
\usetikzlibrary{positioning}
\usetikzlibrary{calc}
\usetikzlibrary{arrows,shapes,backgrounds,3d}
\usepackage{fix-cm}
\usetikzlibrary{decorations.pathreplacing,shapes,snakes}

%%
\newcommand{\coursenumber}{Math 137}
\newcommand{\coursename}{Real and Functional Analysis}


\renewcommand{\Re}[1]{\operatorname{Re} #1 }
\renewcommand{\Im}[1]{\operatorname{Im} #1}
\newcommand{\diam}{\operatorname{diam}}

%% New Commands
\newcommand{\D}{\mathbb{D}}
\newcommand{\Z}{\mathbb{Z}}
\newcommand{\R}{\mathbb{R}}
\newcommand{\Q}{\mathbb{Q}}
\newcommand{\C}{\mathbb{C}}
\newcommand{\F}{\mathbb{F}}
\newcommand{\G}{\mathcal{G}}
\newcommand{\M}{\mathcal{M}}
\newcommand{\K}{\mathbb{K}}
\newcommand{\U}{\mathcal{U}}
\newcommand{\UU}{\mathscr{U}}
\newcommand{\s}{\mathscr{S}}
\newcommand{\A}{\mathcal{A}}
\renewcommand{\L}{\mathfrak{L}}
\newcommand{\B}{\mathcal{B}}
\renewcommand{\P}{\mathcal{P}}
\newcommand{\N}{\mathbb{N}}

\DeclareMathOperator{\Aut}{Aut}



\renewcommand{\ker}{\operatorname{kernel}}
\newcommand{\ran}{\operatorname{range}}

\newcommand{\diag}{\operatorname{diag}}
\newcommand{\norm}[1]{\| #1 \|}
\newcommand{\inner}[1]{\langle #1 \rangle}
\newcommand{\E}{\mathcal{E}}
\newcommand{\V}{\mathcal{V}}
\newcommand{\W}{\mathcal{W}}
\newcommand{\WW}{\mathscr{W}}
\newcommand{\T}{\mathbb{T}}
\newcommand{\vecspan}{\operatorname{span}}
\newcommand{\interior}{\operatorname{int}}
\newcommand{\tr}{\operatorname{tr}}
\newcommand{\rank}{\operatorname{rank}}
\newcommand{\nullity}{\operatorname{nullity}}

\newcommand{\colspace}{\operatorname{colspace}}
\newcommand{\rowspace}{\operatorname{rowspace}}
\newcommand{\nullspace}{\operatorname{nullspace}}

\linespread{1}
\setlength{\parskip}{0.5ex plus 0.5ex minus 0.2ex}

%%  Matrices
\newcommand{\minimatrix}[4]{\begin{bmatrix} #1 & #2 \\ #3 & #4 \end{bmatrix}  }
\newcommand{\megamatrix}[9]{\begin{bmatrix} #1 & #2 & #3 \\ #4 & #5 & #6 \\ #7 & #8 & #9\end{bmatrix}  }

\renewcommand{\vec}[1]{{\bf #1}}
\newcommand{\ovec}{\operatorname{vec}}
\renewcommand{\labelenumi}{(\roman{enumi})}
\renewcommand{\hat}{\widehat}

\newcommand{\tworowvector}[2]{[#1\,\,#2]}
\newcommand{\threerowvector}[3]{[#1\,\,#2\,\,#3]}
\newcommand{\fourrowvector}[4]{[#1\,\,#2\,\,#3\,\,#4]}
\newcommand{\fiverowvector}[5]{[#1\,\,#2\,\,#3\,\,#4\,\,#5]}
\newcommand{\sixrowvector}[6]{[#1\,\,#2\,\,#3\,\,#4\,\,#5\,\,#6]}

\newcommand{\twovector}[2]{\begin{bmatrix} #1\\#2 \end{bmatrix} }
\newcommand{\threevector}[3]{\begin{bmatrix} #1\\#2\\#3 \end{bmatrix} }
\newcommand{\fourvector}[4]{\begin{bmatrix} #1\\#2\\#3\\#4 \end{bmatrix} }
\newcommand{\fivevector}[5]{\begin{bmatrix} #1\\#2\\#3\\#4\\#5 \end{bmatrix} }

\newcommand{\comment}[1]{\marginpar{\scriptsize\color{gray}$\bullet$\,#1}}
\newcommand{\highlight}[1]{\color{blue}#1\color{black}}

\renewcommand{\labelenumi}{\theenumi}
\renewcommand{\theenumi}{(\alph{enumi})}%
\renewcommand{\labelenumii}{\theenumii}
\renewcommand{\theenumii}{(\roman{enumii})}%
\newcommand{\due}[1]{\vspace{-0.2in}\begin{center}\textsc{due at the beginning of class \underline{#1}} \end{center}\medskip }


%%%
%%% Theorem Styles
%%%
\newtheorem{Proposition}{Proposition}
\newtheorem{Corollary}{Corollary}
\newtheorem{Theorem}{Theorem}
\newtheorem*{Thm}{Theorem}
\newtheorem{Postulate}{Postulate}
\newtheorem{Lemma}{Lemma}
\theoremstyle{definition}
\newtheorem*{Definition}{Definition}
\newtheorem*{Example}{Example}
\newtheorem*{Remark}{Remark}
\newtheorem{problem}{Exercise}
\newtheorem*{Question}{Question}

\let\oldenumerate=\enumerate
\def\enumerate{
	\oldenumerate
	\setlength{\itemsep}{1pt}
}
\let\olditemize=\itemize
\def\itemize{
	\olditemize
	\setlength{\itemsep}{5pt}
}

\allowdisplaybreaks
\begin{document}
	\centerline{\textbf{\Large{Final Practice}}}
	
	\begin{itemize}
		\item[1.] Find the equation of the tangent line to the curve $y = \sqrt{2x^2 + 3}$ at $x = -1$ using the limit definition of the derivative. 
		\item[2.] Where is the function $f(x) =  |\sin x |$ differentiable on the interval $0 \leq x \leq 2 \pi$. 
		\item[3.] Find $f'(x)$ and $f''(x)$ when $f(x) = \sqrt[5]{x + x^3}$. 
		\item[4.] (\textbf{Challenge})Find $f'(x)$ and $f''(x)$ when $f(x) = \frac{\sqrt{625 - \sec (x^2)}}{ x^{20} }$.
		\item[5.] Use implicit differentiation to find the tangent line to $x^4 = y^2 + x^2$ at $(2, \sqrt{12})$. 
		\item[6.] Use differentials to estimate the amount of paint needed to apply a coat of paint 0.03 cm thick to a sphere with diameter 40 meters. 
		\item[7.] A police helicopter is flying at 200 kilometers per hour at a constant altitude of 1 km above a straight road. The pilot uses radar to determine that an oncoming car is at a distance of exactly 2 kilometers from the helicopter, and that this distance is decreasing at 250 kph. Find the speed of the car.
		\item[8.] The strength of a rectangular beam is proportional to the product of its width $w$ times the square of its depth $d$. Find the dimensions of the strongest beam that can be cut from a cylindrical log of radius r.
		\item[9.] Consider the function
		\begin{align*}
		f(x) = \frac{1}{\sqrt{8 \pi}} e^{- \frac{(x - 1)^2}{8}}
		\end{align*}
		
		Given that the first and second derivatives are:
		\begin{align*}
		f'(x) &= \frac{1}{8 \sqrt{2 \pi}} (x - 1) e^{- \frac{(x - 1)^2}{8}} \\
		f''(x) &= \frac{1}{32 \sqrt{2 \pi}} (x^2 - 2x - 3) e^{- \frac{(x - 1)^2}{8}}
		\end{align*}
		
		\begin{enumerate}
			\item What are the critical points of $f(x)$. Is there a local max or min ? (Make sure to give the actual point(s) not just the $x$-values). 
			\item What are the points of inflection ? (Make sure to give the actual points, not just the $x$ values). Intervals of concavity? 
			\item Sketch the graph of $f(x)$, making sure to label all the points you found above. 
			\item Bonus question: $f(x)$ is actually a special type of function. What special function is it? 
		\end{enumerate}
	
		\item[10.] Find the derivative of $$f(x) = \int_{-\sqrt{x}}^{e^{x^2 }} \sec(\pi(t^2 + 2t - 1)) \; dt$$
		\item[11.] Evaluate the following integrals: 
		\begin{enumerate}
			\item $$\int_{0}^{4} \sqrt{2x + 1} \; dx$$
			\item $$\int \sin^6 2x \; dx$$ 
			\item $$\int_{-1}^{1} \frac{x^2}{\sqrt[3]{1 - x^3}} \; dx$$. 
		\end{enumerate}
		\item[12.] Find the area between $y = x^2 - 2x$ and $y = x - 2$. 
		\item[13.] Find the average value of $f(x) = \frac{x^2 + 2}{(6x + x^3)^2}$ over the interval $[1, 3]$. 
		\item[14.] Find the volume of the region bounded by $y = x - x^2$ and the $x$-axis rotated about the $x$-axis.
		\item[15.] Suppose that a water tank is shaped like a right circular cone with the tip at the bottom, and has height 10 meters and radius 2 meters at the top. If the tank is full, how much work is required to pump all the water out over the top? 
		\item[16.] Find the arc length of $f(x) = x^{3/2}$ on $[1,4]$. 
		
	\end{itemize}

	
	
\end{document}